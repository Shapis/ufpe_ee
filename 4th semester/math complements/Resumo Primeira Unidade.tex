\documentclass[12pt,twoside, a4paper, twocolumn]{article}
\usepackage[utf8]{inputenc}
\usepackage[brazil]{babel}
\usepackage[margin = 0.5in]{geometry}
\usepackage{amsmath}
\usepackage{amsthm}
\usepackage{amssymb}
\usepackage{amsthm}
\usepackage{setspace}
\usepackage{circuitikz}
\usepackage{lipsum}
\usepackage{pgfplots}



\title{Resumo Complementos de Matematica Primeira Unidade}
\author{Henrique da Silva \\ hpsilva@proton.me}
\date{\today}
\pgfplotsset{width = 10cm, compat = 1.9}


\begin{document}
\maketitle
\pagenumbering{gobble}
\newpage
%pagenumbering{roman}
\tableofcontents
\newpage

\newcommand\deriv[2]{\frac{\mathrm d #1}{\mathrm d #2}}

\section{Potencias de i}
\paragraph{As potencias de i sao periodicas em 4. Da seguinte maneira:
}
\begin{center}
    \begin{tabular}{ |ccc| }
        \hline
        $i^0$ & $=$      & 1            \\
        $i^1$ & $=$      & $i$          \\
        $i^2$ & $=$      & -1           \\
        $i^3$ & $=$      & $-i$         \\
        $i^4$ & $=$      & 1            \\
        $i^5$ & $=$      & $i$          \\
        $i^6$ & $=$      & $-1$         \\
              & $\vdots$ &              \\
        $i^n$ & $=$      & $i^{n \% 4}$ \\
        \hline
    \end{tabular}
\end{center}
\paragraph{Com "\%" sendo resto da divisao inteira
}
% \begin{equation}
%     \oint \vec{E} \, * \, \vec{dA} = \frac{q}{\epsilon_0}
% \end{equation}

\section{Forma algebrica de um numero complexo}
\paragraph{A forma algebrica de um numero complexo eh:}
\begin{equation}
    Z = a + ib
\end{equation}
\paragraph{Onde $a$ eh a componente real de $Z$ e pode ser chamado de $Re(Z)$ e $b$ eh a componente imaginaria e pode ser chamado de $Im(Z)$ }

\paragraph*{Podemos dizer que os numeros $\Re$ sao um subconjunto de $\mathbb{C}$, exceto que no caso de um numero $\Re$ a parte imaginaria $b$ seria 0, alguns exemplos:}
\begin{center}
    \begin{tabular}{ |c|c|c| }
        \hline
        $5 + i$    & $a = 5$  & $b = 1$  \\
        $4 - 3i^2$ & $a = 4$  & $b = -3$ \\
        $12$       & $a = 12$ & $b = 0$  \\
        $7i^3$     & $a = 0$  & $b = 7$  \\
        \hline
    \end{tabular}
\end{center}
\paragraph{Dois numeros complexos sao iguais se seus componentes reais e imaginarios forem iguais}

\section{Operacoes na forma algebrica}

\paragraph*{Nos exemplos a seguir: $Z_n = a_n + ib_n$}

\subsection{Adicao $\&$ Subtracao}
\subparagraph{Para subtrair e adicionar basta subtrair e adicionar as partes imaginarias dos numeros complexos}

\begin{equation}
    Z_1 + Z_2 = (a_1+a_2) + (b_1+b_2)i
\end{equation}
\begin{equation}
    Z_1 - Z_2 = (a_1-a_2) + (b_1-b_2)i
\end{equation}

\subsection{Multiplicacao}
\subparagraph{Vamos utilizar a distributividade e o fato que $i^2 = -1$}
\begin{equation}
    \begin{aligned}
        Z_1 * Z_2 & = (a_1+b_1i)(a_2+b_2i)                    \\
        Z_1 * Z_2 & = a_1a_2 + a_1 b_2i + b_1a_2i + b_1b_2i^2 \\
    \end{aligned}
\end{equation}
\subparagraph*{Como $i^2 = -1$ podemos entao simplificar em:}
\begin{equation}
    Z_1 * Z_2  = (a_1a_2- b_1b_2) + (a_1 b_2 + b_1a_2)i
\end{equation}

\section*{Divisao}


\end{document}