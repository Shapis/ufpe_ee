\documentclass[12pt,twoside, a4paper, twocolumn]{article}
\usepackage[utf8]{inputenc}
\usepackage[brazil]{babel}
\usepackage[margin = 0.5in]{geometry}
\usepackage{amsmath}
\usepackage{amsthm}
\usepackage{amssymb}
\usepackage{amsthm}
\usepackage{setspace}
\usepackage[americanvoltages,fulldiodes,siunitx]{circuitikz}
\usepackage{lipsum}
\usepackage{pgfplots}
\usepackage{ifthen}
\usepackage{adjustbox}
\usepackage[section]{placeins}
\usepackage{hyperref}
\usepackage{graphicx}
\pgfplotsset{compat=newest}
\graphicspath{ {./images/} }
%  #1 color - optional #2 x_0 #3 y_0 #4 x_f #5 y_f #6 name - optional  #7 true if adding lines to axis
\newcommand{\drawvector} [9] [color=cyan] {
  \draw[line width=1.5pt,#1,-stealth](axis cs: #2, #3)--(axis cs: #4, #5) node[anchor=south west]{$#6$};
 \ifthenelse{\equal{#7}{true}}{
  \draw[line width=1pt,#1, dashed](axis cs: #4, #5)--(axis cs: #4, 0) node[anchor= north west]{$#8$};
  \draw[line width=1pt,#1, dashed](axis cs: #4, #5)--(axis cs: 0, #5) node[anchor=south east]{$#9$};
  }
  {}
}
\newcommand\deriv[2]{\frac{\mathrm d #1}{\mathrm d #2}}
\title{Quinto Relatório de Física Experimental 2}
\author{Henrique da Silva \\ hpsilva@proton.me}
\date{\today}
\pgfplotsset{width = 10cm, compat = 1.9}
\begin{document}
\maketitle
\pagenumbering{gobble}
\newpage
%pagenumbering{roman}
\tableofcontents
\newpage

\section{Introdução}

\paragraph*{Neste relatório, vamos discutir e confirmar a lei de inducao de Faraday.}

\paragraph*{Todos arquivos utilizados para criar este relatório, e o relatório em si estão em:  \url{https://github.com/Shapis/ufpe_ee/tree/main/4th semester/}}

\section{Montando o experimento}

\subparagraph*{Analisaremos um sistema de duas bobinas proximas uma da outra. Passaremos uma corrente em um dos dois circuitos e analisaremos a tensao que esta sendo induzida no outro circuito.}

\begin{center}
    \begin{circuitikz}
        % \draw (0,2)
        % node[ocirc,  label=180:$V_{1}$]{};
        % \draw (2.5,0)
        % node[ocirc,  label=90:$V_{a}$]{};
        % \draw (0,0)
        % node[ocirc,  label=180:$V_{2}$]{};
        % \draw (6,-0.5)
        % node[ocirc,  label=2:$V_{0}$]{};
        \draw (0,2) -- (1,2) to[resistor=$R_1$] (3,2) to[inductor=$L_2$] (3,0) -- (0,0);
        \draw (1,2) to[battery2] (1,0);
        \draw (2,0) -- (3,0) -- (3,2) to[resistor=$R_3$] (5,2) -- (5,-0.5) -- (6,-0.5);
        \draw (2,-1) node[above]{$v_i$} to[short, o-] ++(1,0)
        node[op amp, noinv input down, anchor=+](OA){\texttt{}}
        ;
        \draw (2,-1) -- (2,-1.5);
        \draw (2,-1.5)
        node[rground]{};

    \end{circuitikz}
\end{center}

\subsection{Comparando as ondas geradas com a visualização no osciloscópio}

\subparagraph*{Fizemos isto e observamos o comportamento senoidal e quadratico respectivamente das ondas na tela do osciloscópio.}



\end{document}