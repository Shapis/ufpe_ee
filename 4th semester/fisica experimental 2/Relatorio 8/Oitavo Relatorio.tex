\documentclass[12pt,twoside, a4paper, twocolumn]{article}
\usepackage[utf8]{inputenc}
\usepackage[brazil]{babel}
\usepackage[margin = 0.5in]{geometry}
\usepackage{amsmath}
\usepackage{amsthm}
\usepackage{amssymb}
\usepackage{amsthm}
\usepackage{setspace}
\usepackage[americanvoltages,fulldiodes,siunitx]{circuitikz}
\usepackage{lipsum}
\usepackage{pgfplots}
\usepackage{ifthen}
\usepackage{adjustbox}
\usepackage[section]{placeins}
\usepackage{hyperref}
\usepackage{graphicx}
\usepackage{adjustbox}
\pgfplotsset{compat=newest}
\graphicspath{ {./images/} }
%  #1 color - optional #2 x_0 #3 y_0 #4 x_f #5 y_f #6 name - optional  #7 true if adding lines to axis
\newcommand{\drawvector} [9] [color=cyan] {
\draw[line width=1.5pt,#1,-stealth](axis cs: #2, #3)--(axis cs: #4, #5) node[anchor=south west]{$#6$};
\ifthenelse{\equal{#7}{true}}{
\draw[line width=1pt,#1, dashed](axis cs: #4, #5)--(axis cs: #4, 0) node[anchor= north west]{$#8$};
\draw[line width=1pt,#1, dashed](axis cs: #4, #5)--(axis cs: 0, #5) node[anchor=south east]{$#9$};
}
{}
}
\newcommand\deriv[2]{\frac{\mathrm d #1}{\mathrm d #2}}
\title{Sétimo  Relatório de Física Experimental 2}
\author{Henrique da Silva \\ hpsilva@proton.me}
\date{\today}
\pgfplotsset{width = 10cm, compat = 1.9}
\begin{document}
\maketitle
\pagenumbering{gobble}
\newpage
%pagenumbering{roman}
\tableofcontents
\newpage

\section{Introdução}

\paragraph*{Neste relatório, vamos discutir a refração da luz, e sua polarização após incidencia sobre superfícies.}

\paragraph*{Também discutiremos alguns circuitos retificadores com diodos.}

\paragraph*{Todos arquivos utilizados para criar este relatório, e o relatório em si estão em:  \url{https://github.com/Shapis/ufpe_ee/tree/main/4th semester/}}


\section{Reflexão interna total}

\subsection{Tabela de dados}

\begin{center}
  \begin{tabular}{ |c|c|c|c|c| }
    \hline
    $\theta_1$   & $\theta_2$     & $-\theta_2$ & $\theta_{2-m}$  & $n$ \\
                 &                &             &                 &     \\
    $10 \pm 0.5$ & $7.0 \pm 0.5
    $            & $6.5 \pm 0.5$  & $7 \pm 1$   & $1.48 \pm 0.05$       \\
    $20 \pm 0.5$ & $13.0 \pm 0.5
    $            & $13 \pm 0.5$   & $13 \pm 1$  & $1.52 \pm 0.05$       \\
    $30 \pm 0.5$ & $19.5 \pm 0.5
    $            & $19.5 \pm 0.5$ & $ 20 \pm 1$ & $1.50 \pm 0.05$       \\
    $40 \pm 0.5$ & $25.5 \pm 0.5
    $            & $25.5 \pm 0.5$ & $25 \pm 1$  & $1.49 \pm 0.05$       \\
    $50 \pm 0.5$ & $30.5 \pm 0.5
    $            & $30.5 \pm 0.5$ & $ 30 \pm 1$ & $1.51 \pm 0.05$       \\
    $60 \pm 0.5$ & $35.5 \pm 0.5
    $            & $35 \pm 0.5$   & $ 35 \pm 1$ & $1.50 \pm 0.05$       \\
    $70 \pm 0.5$ & $39.0 \pm 0.5
    $            & $38.5 \pm 0.5$ & $ 39 \pm 1$ & $1.50 \pm 0.05$       \\
    $80 \pm 0.5$ & $41.0 \pm 0.5
    $            & $40.5 \pm 0.5$ & $ 41 \pm 1$ & $1.51 \pm 0.05$       \\
    \hline
  \end{tabular}
\end{center}

\subsection{Media dos erros}

\subparagraph*{Obtivemos o $n$ medio e seu respectivo erro da seguinte maneira:}

\begin{equation*}
  \begin{aligned}
     & n_m = \frac{(1.48 + 1.52 + 1.49 + 1.51 + 1.5 + 1.5 + 1.51)}{8} \\
     & n_m = 1.5
  \end{aligned}
\end{equation*}

\begin{equation*}
  \begin{aligned}
     & \varDelta_n = \sum{\sqrt{\frac{1.5 - n_i}{8}}} = 0.01
  \end{aligned}
\end{equation*}

\subparagraph*{Que nos da $n = 1.50 \pm 0.01$}

\subsection{Angulo critico}

\subparagraph*{Obtivemos o ângulo crítico $+43 \pm 1$ e $-43 \pm 1$ que nos dá a média $43 \pm 2$}

\subparagraph*{Para conseguirmos o ângulo de refração lembramos da \emph{Lei de Snell}, e a utilizando obtemos: $\sin{43} = \frac{1}{n_2}$}

\subparagraph*{Que nos da $n_2 = 1.47 \pm 0.08$}

\subsection{Erro absoluto e relativo}

\subparagraph*{Temos que o erro absoluto foi de $0.03$ e o relativo de $2\%$}

\section{Polarização da luz}


\subsection{Verificacao do angulo de Brewster}
\subparagraph*{Obtivemos $\theta_1 = 56$ e $\theta_2 = 34$, e por lei de refração $\theta_1 = \theta_b$. Logo:}

\begin{equation}
  \theta_b + \theta_2 = 90
\end{equation}

\subsection{Sentido da polarização}

\subparagraph*{A luz refletida no ângulo de Brewster é polarizada perpendicularmente ao plano de incidência.}

\subsection{Obtendo o coeficiente de refracao}

\subparagraph*{Partirei da equação (1), e aplicarei a \emph{Lei de Snell}, utilizando $1$ como índice de refração do ar.}

\begin{equation*}
  \begin{aligned}
     & \sin{\theta_b} = n \sin{\theta_2}                              \\
     & \sin{\theta_b} = n \sin{\left(\frac{\pi}{2} - \theta_b\right)} \\
     & \tan{\theta_b} = n                                             \\
     & \arctan{n} = 56                                                \\
     & n = 1.48                                                       \\
  \end{aligned}
\end{equation*}

\subparagraph*{Observamos que o valor encontrado está coerente com os encontrados anteriormente.}

\subsection{Polarizadores em serie}

\subparagraph*{A luz que passa pelo nosso polarizador fica polarizada em uma direção específica. Se aplicarmos um polarizador ortogonal a este em série, este segundo bloqueará toda luz do primeiro e não observaremos nada após o segundo. }

\subparagraph*{Porém se utilizarmos um terceiro polarizador entre estes dois com ângulo de $pi/4$, ou seja um ângulo médio entre eles. Permitiremos alguma luz do primeiro passar pelo segundo, e alguma luz do segundo passar pelo terceiro.}

\subparagraph*{O efeito interessante é que se tivermos infinitos polarizadores em série, variando ângulo infinitesimalmente entre si, toda luz passará por todos eles. Porque a variação infinitesimal não bloqueia luz alguma em nenhum deles.}


\end{document}