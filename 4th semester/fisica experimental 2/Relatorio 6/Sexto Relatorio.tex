\documentclass[12pt,twoside, a4paper, twocolumn]{article}
\usepackage[utf8]{inputenc}
\usepackage[brazil]{babel}
\usepackage[margin = 0.5in]{geometry}
\usepackage{amsmath}
\usepackage{amsthm}
\usepackage{amssymb}
\usepackage{amsthm}
\usepackage{setspace}
\usepackage[americanvoltages,fulldiodes,siunitx]{circuitikz}
\usepackage{lipsum}
\usepackage{pgfplots}
\usepackage{ifthen}
\usepackage{adjustbox}
\usepackage[section]{placeins}
\usepackage{hyperref}
\usepackage{graphicx}
\usepackage{adjustbox}
\pgfplotsset{compat=newest}
\graphicspath{ {./images/} }
%  #1 color - optional #2 x_0 #3 y_0 #4 x_f #5 y_f #6 name - optional  #7 true if adding lines to axis
\newcommand{\drawvector} [9] [color=cyan] {
 \draw[line width=1.5pt,#1,-stealth](axis cs: #2, #3)--(axis cs: #4, #5) node[anchor=south west]{$#6$};
\ifthenelse{\equal{#7}{true}}{
 \draw[line width=1pt,#1, dashed](axis cs: #4, #5)--(axis cs: #4, 0) node[anchor= north west]{$#8$};
 \draw[line width=1pt,#1, dashed](axis cs: #4, #5)--(axis cs: 0, #5) node[anchor=south east]{$#9$};
 }
 {}
}
\newcommand\deriv[2]{\frac{\mathrm d #1}{\mathrm d #2}}
\title{Sexto Relatório de Física Experimental 2}
\author{Henrique da Silva \\ hpsilva@proton.me}
\date{\today}
\pgfplotsset{width = 10cm, compat = 1.9}
\begin{document}
\maketitle
\pagenumbering{gobble}
\newpage
%pagenumbering{roman}
\tableofcontents
\newpage

\section{Introdução}

\paragraph*{Neste relatório, vamos discutir o funcionamento de um circuito $RC$, em particular suas funcoes como filtro de sinais e integrador de sinais.}

\paragraph*{Todos arquivos utilizados para criar este relatório, e o relatório em si estão em:  \url{https://github.com/Shapis/ufpe_ee/tree/main/4th semester/}}


\section{Lei da inducao de Faraday}

\subsection{Inducao de corrente por campo magnetico}

\end{document}