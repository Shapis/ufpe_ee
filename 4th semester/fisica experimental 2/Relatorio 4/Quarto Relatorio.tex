\documentclass[12pt,twoside, a4paper, twocolumn]{article}
\usepackage[utf8]{inputenc}
\usepackage[brazil]{babel}
\usepackage[margin = 0.5in]{geometry}
\usepackage{amsmath}
\usepackage{amsthm}
\usepackage{amssymb}
\usepackage{amsthm}
\usepackage{setspace}
\usepackage[americanvoltages,fulldiodes,siunitx]{circuitikz}
\usepackage{lipsum}
\usepackage{pgfplots}
\usepackage{ifthen}
\usepackage{adjustbox}
\usepackage[section]{placeins}
\usepackage{hyperref}
\usepackage{graphicx}
\pgfplotsset{compat=newest}
\graphicspath{ {./images/} }
 
 
%  #1 color - optional #2 x_0 #3 y_0 #4 x_f #5 y_f #6 name - optional  #7 true if adding lines to axis
 
\newcommand{\drawvector} [9] [color=cyan] {
   \draw[line width=1.5pt,#1,-stealth](axis cs: #2, #3)--(axis cs: #4, #5) node[anchor=south west]{$#6$};
 
  
 
\ifthenelse{\equal{#7}{true}}{
   \draw[line width=1pt,#1, dashed](axis cs: #4, #5)--(axis cs: #4, 0) node[anchor= north west]{$#8$};
   \draw[line width=1pt,#1, dashed](axis cs: #4, #5)--(axis cs: 0, #5) node[anchor=south east]{$#9$};
   }
   {}
}
 
\newcommand\deriv[2]{\frac{\mathrm d #1}{\mathrm d #2}}
 
 
\title{Terceiro Relatório de Física Experimental 2}
\author{Henrique da Silva \\ hpsilva@proton.me}
\date{\today}
\pgfplotsset{width = 10cm, compat = 1.9}
 
 
\begin{document}
\maketitle
\pagenumbering{gobble}
\newpage
%pagenumbering{roman}
\tableofcontents
\newpage

\section{Introdução}

\paragraph*{Neste relatório, vamos discutir o capacitor. E como ele se se comporta sobre acao de correntes diretas e alternadas.}

\paragraph*{Todos arquivos utilizados para criar este relatório, e o relatório em si estão em:  \url{https://github.com/Shapis/ufpe_ee/tree/main/4th semester/}}


\section{Funcionamento basico de um osciloscopio}

\subsection{Comparando as ondas geradas com a visualizacao no osciloscopio}

\subparagraph*{Fizemos isto e observamos o comportamento senoidal e quadratico respectivamente das ondas na tela do osciloscopio.}

\subsection{Graficos das ondas observadas}

\subsubsection{Grafico da tensao $V_{ad}$ pelo tempo em milisegundos no acoplamento AC}

\begin{tikzpicture}
    \begin{axis}[
            ylabel={Tensao $V$},
            xlabel={Tempo $ms$},
            xmin = 0, xmax = 10,
            ymin = -4, ymax = 5.0,
            xtick distance = 1,
            ytick distance = 1,
            grid = both,
            minor tick num = 1,
            major grid style = {lightgray},
            minor grid style = {lightgray!25},
            width = \textwidth,
            height = 0.5\textwidth]
        \addplot[
            domain = 0:10,
            samples = 400,
            smooth,
            thick,
            blue,
        ] {1 + 3 * cos(deg(2*pi*0.5* x))};
    \end{axis}
\end{tikzpicture}

\subsubsection{Grafico da tensao $V_{ad}$ pelo tempo em milisegundos no acoplamento DC}

\begin{tikzpicture}
    \begin{axis}[
            ylabel={Tensao $V$},
            xlabel={Tempo $ms$},
            xmin = 0, xmax = 10,
            ymin = -4, ymax = 5.0,
            xtick distance = 1,
            ytick distance = 1,
            grid = both,
            minor tick num = 1,
            major grid style = {lightgray},
            minor grid style = {lightgray!25},
            width = \textwidth,
            height = 0.5\textwidth]
        \addplot[
            domain = 0:10,
            samples = 400,
            smooth,
            thick,
            blue,
        ] {1 };
    \end{axis}
\end{tikzpicture}


\clearpage
\subsubsection{Grafico da tensao $V_{bd}$ pelo tempo em milisegundos no acoplamento AC}

\begin{tikzpicture}
    \begin{axis}[
            ylabel={Tensao $V$},
            xlabel={Tempo $ms$},
            xmin = 0, xmax = 10,
            ymin = -4, ymax = 5.0,
            xtick distance = 1,
            ytick distance = 1,
            grid = both,
            minor tick num = 1,
            major grid style = {lightgray},
            minor grid style = {lightgray!25},
            width = \textwidth,
            height = 0.5\textwidth]
        \addplot[
            domain = 0:10,
            samples = 400,
            smooth,
            thick,
            blue,
        ] {1 + 2.1 * cos(deg(2*pi*0.5* x))};
    \end{axis}
\end{tikzpicture}

\subsubsection{Grafico da tensao $V_{bd}$ pelo tempo em milisegundos no acoplamento DC}

\begin{tikzpicture}
    \begin{axis}[
            ylabel={Tensao $V$},
            xlabel={Tempo $ms$},
            xmin = 0, xmax = 10,
            ymin = -4, ymax = 5.0,
            xtick distance = 1,
            ytick distance = 1,
            grid = both,
            minor tick num = 1,
            major grid style = {lightgray},
            minor grid style = {lightgray!25},
            width = \textwidth,
            height = 0.5\textwidth]
        \addplot[
            domain = 0:10,
            samples = 400,
            smooth,
            thick,
            blue,
        ] {1};
    \end{axis}
\end{tikzpicture}

\clearpage
\subsubsection{Grafico da tensao $V_{cd}$ pelo tempo em milisegundos no acoplamento AC}

\begin{tikzpicture}
    \begin{axis}[
            ylabel={Tensao $V$},
            xlabel={Tempo $ms$},
            xmin = 0, xmax = 10,
            ymin = -4, ymax = 5.0,
            xtick distance = 1,
            ytick distance = 1,
            grid = both,
            minor tick num = 1,
            major grid style = {lightgray},
            minor grid style = {lightgray!25},            width = \textwidth,
            height = 0.5\textwidth]
        \addplot[
            domain = 0:10,
            samples = 400,
            smooth,
            thick,
            blue,
        ] { 1.25 * cos(deg(2*pi*0.5* x))};
    \end{axis}
\end{tikzpicture}

\subsubsection{Grafico da tensao $V_{cd}$ pelo tempo em milisegundos no acoplamento DC}

\begin{tikzpicture}
    \begin{axis}[
            ylabel={Tensao $V$},
            xlabel={Tempo $ms$},
            xmin = 0, xmax = 10,
            ymin = -4, ymax = 5.0,
            xtick distance = 1,
            ytick distance = 1,
            grid = both,
            minor tick num = 1,
            major grid style = {lightgray},
            minor grid style = {lightgray!25},
            width = \textwidth,
            height = 0.5\textwidth]
        \addplot[
            domain = 0:10,
            samples = 400,
            smooth,
            thick,
            blue,
        ] {0};
    \end{axis}
\end{tikzpicture}

\clearpage

\subsection{Medindo $V_{ab}$ , $V_{bd}$, e $V_{cd}$}

\subparagraph*{Nao podemos fazer estas medicoes diretamente pois estariamos alterando o circuito se encaixassemos o osciloscopio nos pontos AB, BD, e CD respectivamente.}

\subsection{Papel do capacitor}

\subparagraph*{Este esta "bloqueando" a passagem da corrente direta. Isto acontece porque a medida que a corrente direta carrega o capacitor, a tensao nos terminais do capacitor se iguala. }
\subparagraph*{Quando o capacitor esta completamente carregando, as tensoes nos seus terminais fica iguai, e nao ha passagem de corrente.}

\subsection{Equacoes das tensoes}

\begin{center}
    \begin{tabular}{ |ccc| }
        \hline
        $AC: V_{ad}$ & $\rightarrow$ & V = 1 + 3cos(2pi*500*t)   \\
        $DC: V_{ad}$ & $\rightarrow$ & V = 1                     \\
        $AC: V_{bd}$ & $\rightarrow$ & V = 1 + 2.1cos(2pi*500*t) \\
        $DC: V_{bd}$ & $\rightarrow$ & V = 1                     \\
        $AC: V_{cd}$ & $\rightarrow$ & V = 1.25cos(2pi*500*t)    \\
        $DC: V_{cd}$ & $\rightarrow$ & V = 0                     \\
        \hline
    \end{tabular}
\end{center}




\subsection{Medicoes no multimetro}

\subparagraph*{$DC = 0.855V$ e $AC = 2.069V$}

\subparagraph*{Indicando que estamos lidando com medicoes rms}

\subsection{Diferenca entre valores medios e RMS}

\subparagraph*{Valores RMS nos levamos em consideracao a raiz dos quadrados de todos valores. O que faz com que correntes alternadas somem ao valor. }

\subparagraph*{Ja valor medio, o caso da corrente alternada somaria como 0. Ja que ha o mesmo numero de valores positivos que negativos}

\subparagraph*{O caso no qual rms = valor medio sera o caso no qual nao ha componente de corrente alternada no sistema.}

\clearpage

\section{Carga e descarga de um capacitor}

\begin{tikzpicture}
    \begin{axis}[
            ylabel={Tensao $V$},
            xlabel={Constante de tempo $\tau$ em ms},
            xmin = 0, xmax = 10,
            ymin = -4, ymax = 5.0,
            xtick distance = 1,
            ytick distance = 1,
            grid = both,
            minor tick num = 1,
            major grid style = {lightgray},
            minor grid style = {lightgray!25},
            width = \textwidth,
            height = 0.5\textwidth]
        \addplot[
            domain = 0:2,
            samples = 400,
            smooth,
            thick,
            blue,
        ] {4*(1-e^-x/0.4)};
        \addplot[
            domain = 2:4,
            samples = 400,
            smooth,
            thick,
            blue,
        ] {4*e^-(x-2)/0.4};
        \addplot[
            domain = 4:6,
            samples = 400,
            smooth,
            thick,
            blue,
        ] {4*(1-e^-(x-4)/0.4)};
        \addplot[
            domain = 6:8,
            samples = 400,
            smooth,
            thick,
            blue,
        ] {4*e^-(x-6)/0.4};
        \addplot[
            domain = 8:10,
            samples = 400,
            smooth,
            thick,
            blue,
        ] {4*(1-e^-(x-8)/0.4)};
    \end{axis}
\end{tikzpicture}


\end{document}
