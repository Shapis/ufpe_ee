\documentclass[12pt,twoside, a4paper, twocolumn]{article}
\usepackage[utf8]{inputenc}
\usepackage[brazil]{babel}
\usepackage[margin = 0.5in]{geometry}
\usepackage{amsmath}
\usepackage{amsthm}
\usepackage{amssymb}
\usepackage{amsthm}
\usepackage{setspace}
\usepackage[americanvoltages,fulldiodes,siunitx]{circuitikz}
\usepackage{lipsum}
\usepackage{pgfplots}
\usepackage{ifthen}
\usepackage{adjustbox}
\usepackage[section]{placeins}
\usepackage{hyperref}

\pgfplotsset{compat=newest}



%  #1 color - optional #2 x_0 #3 y_0 #4 x_f #5 y_f #6 name - optional  #7 true if adding lines to axis

\newcommand{\drawvector} [9] [color=cyan] {
    \draw[line width=1.5pt,#1,-stealth](axis cs: #2, #3)--(axis cs: #4, #5) node[anchor=south west]{$#6$};

    

\ifthenelse{\equal{#7}{true}}{
    \draw[line width=1pt,#1, dashed](axis cs: #4, #5)--(axis cs: #4, 0) node[anchor= north west]{$#8$};
    \draw[line width=1pt,#1, dashed](axis cs: #4, #5)--(axis cs: 0, #5) node[anchor=south east]{$#9$};
    }
    {}
}

\newcommand\deriv[2]{\frac{\mathrm d #1}{\mathrm d #2}}


\title{Primeiro Relatório de Lab de Circuitos}
\author{Henrique da Silva \\ hpsilva@proton.me}
\date{\today}
\pgfplotsset{width = 10cm, compat = 1.9}


\begin{document}
\maketitle
\pagenumbering{gobble}
\newpage
%pagenumbering{roman}
\tableofcontents
\newpage

\section{Introdução}

\paragraph*{Neste relatório, vamos discutir a ponte de Wheatstone e um método experimental para obter uma Resistencia desconhecida a partir de um circuito ja conhecido }

\paragraph*{Todos arquivos utilizados para criar este relatorio, e o relatorio em si estão em:  \url{https://github.com/Shapis/ufpe_ee/tree/main/4th semester/lab circuitos}}

\subsection{A ponte de Wheatstone}
\subparagraph*{}
\begin{center}
    \begin{circuitikz}
        \draw
        (0,0) to[battery1=$V_{cc}$,  invert] (0,4) % l=5<\milli\volt>
        -- (2,4) to[resistor=$R_1$] (2,2) to [resistor=$R_3$] (2,0)
        (2,4) -- (4,4) to [resistor=$R_2$] (4,2) to [resistor=$R_4$] (4,0)
        -- (0,0)
        ;
        \draw (0,-0.05)
        node[rground]{};
        \draw (2,2)
        node[ocirc,  label=2:$V_{a}$]{};
        \draw (4,2)
        node[ocirc,  label=2:$V_{b}$]{};
    \end{circuitikz}
\end{center}

\subparagraph*{Esta tem como função principal determinar uma resistência desconhecida $R_4$ a partir de três resistências e uma corrente previamente conhecidas, que vamos chamar aqui de $V_{cc}$ e $R_1$, $R_2$, e $R_3$.}

\subsection{Obtendo $R_4$}

\subparagraph*{Para obter essa resistência desconhecida, o que faremos é inicialmente determinar as tensões $V_a$ e $V_b$ em função das resistências e da tensão $Vcc$. E a partir dessas determinar uma expressão para $R_4$ }

\subparagraph*{Montando o sistema e equações e lembrando da soma de resistores em série e em paralelos teremos: }

\begin{equation}
    \begin{aligned}
        V_{a} & = \frac{R_3}{R_1+R_3}V_{cc} \\
        V_{b} & = \frac{R_4}{R_2+R_4}V_{cc} \\
    \end{aligned}
\end{equation}

\subparagraph*{Daí tiramos que o nosso $V_{ab}$ sendo este $V_a - V_b$ será:}

\begin{equation}
    \begin{aligned}
        V_{ab} & = V_{a} - V_{b} = \left(\frac{R_3}{R_1+R_3} - \frac{R_4}{R_2+R_4}\right)V_{cc} \\
    \end{aligned}
\end{equation}

\subparagraph*{Resolvendo isolando o $R_4$ teremos:}

\begin{equation}
    \begin{aligned}
        R_4 = \frac{R2(R3(V_{cc} - V_{ab}) - R_1V_{ab})}{R_1 (V_{cc} + V_{ab}) +R_3V_{ab}} \\
    \end{aligned}
\end{equation}

\subparagraph*{Com isso conseguimos facilmente isolar nossa resistência desconhecida $R_4$ a partir de valores conhecidos do sistema}

\subsection{Resultados preliminares}


\subparagraph*{ Inicialmente montarei o sistema no simulador de circuitos online Falstad. \href{https://www.falstad.com/circuit/circuitjs.html?ctz=CQAgjCAMB0l3BWcMBMcUHYMGZIA4UA2ATmIxAUgoqoQFMBaMMAKADcQUAWKlPPTjxDFCUMULBUpUaAhYB3QVREhsCFMNGQFq9ZpBcuAldoBOBo-sMCwhaVwzwd18HYsCU3KC3MvPXd05+MUknRT9g7l5gs10NFTUNMAwtcEo4HUTwFLjOL1is-yUggV40DIAHXKK-LwgpFiqXWypagPrvIA}{Clique aqui para acessar}}

\subparagraph*{Para o exemplo preliminar com o seguintes valores iniciais: }
\subparagraph*{$V_{cc} = 10V$ , $R_1 = 15k\varOmega$ , $R_2 = 47k\varOmega$, $R_3 = 22k\varOmega$ e $R_4 = 10k\varOmega$}

\subparagraph*{Resolvendo em python \href{https://www.online-python.com/AkfwEsRGU9}{(clique aqui para acessar)} as equações (1) e (2) teremos o seguinte valores para $V_{a}$  $V_b$ e $V_{ab}$:}

\begin{equation*}
    \begin{aligned}
        V_{a}  & = 5.946V \\
        V_{b}  & = 1.754V \\
        V_{ab} & = 4.191V \\
    \end{aligned}
\end{equation*}


\section{Descrição da prática}

\paragraph*{Nesta prática montei o circuito descrito em (1.1). }

\paragraph*{Coletei medições deste sistema com todos resistores conhecidos, e apos, com um desconhecido.}
\paragraph*{Fiz uma analise comparando os resultados experimentais com os resultados experimentais.}



\section{Resultados}

\subsection{Medições do sistema conhecido}

\subparagraph*{Abaixo estão os valores experimentais dos elementos do sistema.}

\subsubsection{Resistores}

\begin{center}
    \begin{tabular}{ |ccc| }
        \hline
        $R_1$ & $\rightarrow$ & $14.907\, m\varOmega$ \\
        $R_2$ & $\rightarrow$ & $21.930\, m\varOmega$ \\
        $R_3$ & $\rightarrow$ & $48.600\, m\varOmega$ \\
        $R_4$ & $\rightarrow$ & $9.835\, m\varOmega$  \\
        \hline
    \end{tabular}
\end{center}

\subsubsection{Fontes de tensao}
\subparagraph*{Abaixo estão os valores experimentais das fontes de tensão, e o modulo da diferença $d$ entre os valores experimentais e os esperados teóricos.}

\begin{center}
    \begin{tabular}{ |ccccc| }
        \hline
        $V_{cc}$   & $\rightarrow$ & $10.000\,V$ & $\rightarrow$ & $0\,V$      \\
        $V_{a}$    & $\rightarrow$ & $5.945\, V$ & $\rightarrow$ & $0.001 \,V$ \\
        $V_{b}\,$  & $\rightarrow$ & $1.681\, V$ & $\rightarrow$ & $0.073\,V$  \\
        $V_{ab}\,$ & $\rightarrow$ & $4.262\, V$ & $\rightarrow$ & $0.071\,V$  \\
        \hline
    \end{tabular}
\end{center}

\subsection{Medições do sistema desconhecido}

\subparagraph{Abaixo estao os $V_{cc}$ e o $V_{ab}$ experimentais, o modulo da diferença $d$ entre o $V_{ab}$ experimental e o teórico, e o $R_4$ conseguido a partir das medições experimentais.}


\begin{center}
    \begin{tabular}{ |ccccccc| }
        \hline
        $V_{cc}$   & $$            & $V_{ab}$    & $$            & $R_4$                & $\rightarrow$ & $d$ \\
        $0.516\,V$ & $\rightarrow$ & $0.007\, V$ & $\rightarrow$ & $66.392\,m\varOmega$ & $\rightarrow$ & $0$ \\
        $2.497\,V$ & $\rightarrow$ & $0.033\, V$ & $\rightarrow$ & $66.517\,m\varOmega$ & $\rightarrow$ & $0$ \\
        $4.503\,V$ & $\rightarrow$ & $0.060\, V$ & $\rightarrow$ & $66.478\,m\varOmega$ & $\rightarrow$ & $0$ \\
        $6.499\,V$ & $\rightarrow$ & $0.086\, V$ & $\rightarrow$ & $66.511\,m\varOmega$ & $\rightarrow$ & $0$ \\
        $8.500\,V$ & $\rightarrow$ & $0.113\, V$ & $\rightarrow$ & $66.489\,m\varOmega$ & $\rightarrow$ & $0$ \\
        $9.985\,V$ & $\rightarrow$ & $0.133\, V$ & $\rightarrow$ & $66.479\,m\varOmega$ & $\rightarrow$ & $0$ \\
        \hline
    \end{tabular}
\end{center}

\begin{equation}
    \overline{R_4} = 66.478\, m\varOmega
\end{equation}

\subparagraph*{Creio que meu $d$ deu $0$ porque para consegui-lo eu preciso do $R_4$ que inicialmente é desconhecido. Então eu o calculo de acordo com (3). E (3) utiliza meu $V_{ab}$ experimental e resistências que foram obtidas experimentalmente. Logo quando re-calculo o $V_{ab}$ com o agora conhecido valor de $R_4$, este fica atrelado ao $V_{ab}$ experimental.}
\subparagraph*{A alternativa seria utilizar resistências teóricas e $V_{cc}$ teóricos. Mas acho que isso não faz sentido porque o objetido do experimento em si é obter o $R_4$, entao a priori, eu não teria um valor teórico para o $R_4$}

\section{Conclusão}

\paragraph*{Utilizando um circuito de \emph{Wheatstone} posso medir pequenas alterações de $V_{ab}$ para descobrir uma resistência desconhecida com bastante precisão.}

\subparagraph*{Esse sistema é bastante robusto para diferentes valores de tensões de fonte. E tambem é significantemente resistente a erros aleatórios de medição.}

\end{document}

