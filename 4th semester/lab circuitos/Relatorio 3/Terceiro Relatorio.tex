\documentclass[12pt,twoside, a4paper, twocolumn]{article}
\usepackage[utf8]{inputenc}
\usepackage[brazil]{babel}
\usepackage[margin = 0.5in]{geometry}
\usepackage{amsmath}
\usepackage{amsthm}
\usepackage{amssymb}
\usepackage{amsthm}
\usepackage{setspace}
\usepackage[americanvoltages,fulldiodes,siunitx]{circuitikz}
\usepackage{lipsum}
\usepackage{pgfplots}
\usepackage{ifthen}
\usepackage{adjustbox}
\usepackage[section]{placeins}
\usepackage{hyperref}

\pgfplotsset{compat=newest}



%  #1 color - optional #2 x_0 #3 y_0 #4 x_f #5 y_f #6 name - optional  #7 true if adding lines to axis

\newcommand{\drawvector} [9] [color=cyan] {
    \draw[line width=1.5pt,#1,-stealth](axis cs: #2, #3)--(axis cs: #4, #5) node[anchor=south west]{$#6$};

    

\ifthenelse{\equal{#7}{true}}{
    \draw[line width=1pt,#1, dashed](axis cs: #4, #5)--(axis cs: #4, 0) node[anchor= north west]{$#8$};
    \draw[line width=1pt,#1, dashed](axis cs: #4, #5)--(axis cs: 0, #5) node[anchor=south east]{$#9$};
    }
    {}
}

\newcommand\deriv[2]{\frac{\mathrm d #1}{\mathrm d #2}}


\title{Terceiro Relatório de Lab de Circuitos}
\author{Henrique da Silva \\ hpsilva@proton.me}
\date{\today}
\pgfplotsset{width = 10cm, compat = 1.9}


\begin{document}
\maketitle
\pagenumbering{gobble}
\newpage
%pagenumbering{roman}
\tableofcontents
\newpage



\section{Introdução}


\subparagraph*{Neste relatório, vamos discutir amplificadores operacionais, e como controlar uma saida de corrente a partir de duas correntes de entrada.}

\subparagraph*{Todos arquivos utilizados para criar este relatorio, e o relatorio em si estão em:  \url{https://github.com/Shapis/ufpe_ee/tree/main/4th semester/lab circuitos}}




\subsection{O Amp Op}
\subparagraph*{}
\begin{center}
    \begin{circuitikz}
        \draw (0,2)
        node[ocirc,  label=180:$V_{1}$]{};
        \draw (2.5,0)
        node[ocirc,  label=90:$V_{a}$]{};
        \draw (0,0)
        node[ocirc,  label=180:$V_{2}$]{};
        \draw (6,-0.5)
        node[ocirc,  label=2:$V_{0}$]{};
        \draw (0,2) to[resistor=$R_1$] (2,2) -- (2,0) to[resistor=$R_2$] (0,0);
        \draw (2,0) -- (3,0) -- (3,2) to[resistor=$R_3$] (5,2) -- (5,-0.5) -- (6,-0.5);
        \draw (2,-1) node[above]{$v_i$} to[short, o-] ++(1,0)
        node[op amp, noinv input down, anchor=+](OA){\texttt{}}
        ;
        \draw (2,-1) -- (2,-1.5);
        \draw (2,-1.5)
        node[rground]{};

    \end{circuitikz}
\end{center}

\subparagraph*{Neste caso o amp op faria uma multiplicacao da corrente $V_a$ na saida $V_0$ de acordo com um fator de multiplicacao $A$}





\section{Analise nodal do circuito}


\subparagraph*{Primeiro vale lembrar que a resisencia de Thevenin e a de Norton sao iguais. Logo obtendo uma tambem obteremos a outra.
}

\subparagraph*{Neste caso, resolvendo o sistema vamos obter que esta resistencia eh igual a $R_{c}$}

\begin{equation}
    \begin{aligned}
         & \frac{V_a - V_1}{R_1} + \frac{V_a-V_0}{R_3} + \frac{V_a-V_2}{R_2} = 0 \\
         & V_0 = -A*V_a
    \end{aligned}
\end{equation}

\subparagraph*{Que nos da:}

\begin{equation}
    V_0 = -\frac{A R_1 R_3 V_2 + A R_2 R_3 V_1}{(R_2 + R_1)R_3 + (A + 1) R_1 R_2}
\end{equation}

\subparagraph*{E para o caso especifico do amp op ideal, fazemos $A$ tender a infinito e simplesmente temos:}

\begin{equation}
    \begin{aligned}
        V_0 & = -\frac{R_1 R_3 V_2 + R_2 R_3 V_1}{R_1 R_2} \\
        V_0 & = - \frac{R_3}{R_1}V_1 - \frac{R_3}{R_2}V_2  \\
    \end{aligned}
\end{equation}

\subparagraph*{Dai podemos juntar (1) com (3) e obter:}

\subparagraph*{Isto me da as seguintes equacoes:}

\begin{equation}
    \begin{aligned}
        A_{v_1} & = -\frac{R_3}{R_1} \\
        A_{v_2} & = -\frac{R_3}{R_2} \\
    \end{aligned}
\end{equation}


\subparagraph*{Tambem eh importante notar que as resistencias vistas de $V_1$ e $V_2$ sao as seguintes:}
\begin{equation}
    I_n = \frac{V_1-V_a}{R_n} \rightarrow R_{im_n} = \frac
    {V_n}{I_n} = R_n * \frac{V_n}{V_n - V_a} = R_n
\end{equation}

\section{Resultados preliminares}

\subparagraph*{Aqui vamos fazer uma analise utilizando a teoria demonstrada acima para saber como montar o circuito para termos um ganho $A_1 = -2$ e $A_2 = -4$}
\subsection{Montando o circuito}
\subparagraph*{Nos temos da equacao (4) como os ganhos se comportam a partir das resistencias do circuito. Entao, basta resolvermos este sistema utilizando valores de resistores comerciais.}



\begin{equation}
    \begin{aligned}
        A_{v_1} & = -\frac{R_3}{R_1} = -2 \\
        A_{v_2} & = -\frac{R_3}{R_2} = -4 \\
    \end{aligned}
\end{equation}

\subparagraph*{Podemos entao escolher resistores com aproxidamente os seguintes valores:}

\begin{equation}
    \begin{aligned}
        R_1 & \approx 100k \varOmega \\
        R_2 & \approx 47k \varOmega  \\
        R_3 & \approx 220k \varOmega \\
    \end{aligned}
\end{equation}

\subsection{Valores esperados}

\subparagraph*{Vamos analisar as seguintes combinacoes de tensoes em $V_1$ e $V_2$: ${-1,2 ; -0,6 ; 0 ; 0,6 ; 1,2}$}

\subparagraph*{A analise sera feita em $C\#$ e esta na pasta do arquivo}

\begin{center}
    \begin{tabular}{ |cccccc| }
        \hline
        $V_1 \rightarrow$ & $-1.2$ & $-0.6$ & $0.0$ & $0.6$ & $1.2$ \\
        $ V_2 \downarrow$ & $2$    & $2$    & $3$   & $4$   & $5$   \\
        $-1.2$            & $2$    & $2$    & $3$   & $4$   & $5$   \\
        $ -0.6$           & $2$    & $2$    & $3$   & $4$   & $5$   \\
        $0.0$             & $2$    & $2$    & $3$   & $4$   & $5$   \\
        $ 0.6$            & $2$    & $2$    & $3$   & $4$   & $5$   \\
        $ 1.2$            & $2$    & $2$    & $3$   & $4$   & $5$   \\
        \hline
    \end{tabular}
\end{center}

\end{document}

