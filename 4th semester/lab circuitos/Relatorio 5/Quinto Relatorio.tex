\documentclass[12pt,twoside, a4paper, twocolumn]{article}
\usepackage[utf8]{inputenc}
\usepackage[brazil]{babel}
\usepackage[margin = 0.5in]{geometry}
\usepackage{amsmath}
\usepackage{amsthm}
\usepackage{amssymb}
\usepackage{amsthm}
\usepackage{setspace}
\usepackage[americanvoltages,fulldiodes,siunitx]{circuitikz}
\usepackage{lipsum}
\usepackage{pgfplots}
\usepackage{ifthen}
\usepackage{adjustbox}
\usepackage[section]{placeins}
\usepackage{hyperref}
\usepackage{graphicx}
\usepackage{amsmath}
\usepackage{amsthm}
\usepackage{amssymb}
\usepackage{amsthm}
\usepackage{setspace}
\usepackage[americanvoltages,fulldiodes,siunitx]{circuitikz}
\usepackage{lipsum}
\usepackage{pgfplots}
\usepackage{ifthen}
\usepackage{adjustbox}
\usepackage[section]{placeins}
\usepackage{hyperref}
\usepackage{graphicx}
\usepackage{adjustbox}
\pgfplotsset{compat=newest}
\graphicspath{ {./images/} }
%  #1 color - optional #2 x_0 #3 y_0 #4 x_f #5 y_f #6 name - optional  #7 true if adding lines to axis
\newcommand{\drawvector} [9] [color=cyan] {
 \draw[line width=1.5pt,#1,-stealth](axis cs: #2, #3)--(axis cs: #4, #5) node[anchor=south west]{$#6$};
\ifthenelse{\equal{#7}{true}}{
 \draw[line width=1pt,#1, dashed](axis cs: #4, #5)--(axis cs: #4, 0) node[anchor= north west]{$#8$};
 \draw[line width=1pt,#1, dashed](axis cs: #4, #5)--(axis cs: 0, #5) node[anchor=south east]{$#9$};
 }
 {}
}
\newcommand\deriv[2]{\frac{\mathrm d #1}{\mathrm d #2}}
\title{Quinto Relatório de Lab de Circuitos}
\author{Henrique da Silva \\ hpsilva@proton.me}
\date{\today}
\pgfplotsset{width = 10cm, compat = 1.9}
\begin{document}
\maketitle
\pagenumbering{gobble}
\newpage
%pagenumbering{roman}
\tableofcontents
\newpage



\section{Introdução}

\subparagraph*{Neste relatório vamos discutir novamente o Amp Op. Desta vez em uma configuração que tenha um capacitor no circuito, e veremos que ele se comporta como um circuito RC.}

\subparagraph*{Todos arquivos utilizados para criar este relatório, é o relatorio em si estão em:  \url{https://github.com/Shapis/ufpe_ee/tree/main/4th semester/lab circuitos}}


\section{Análise dos circuitos}

\subsection*{Capacitor em paralelo na realimentação}

\subparagraph*{Podemos fazer a seguinte análise no nosso circuito:}

\begin{equation}
    \begin{aligned}
        V_c = Va - V_0
    \end{aligned}
\end{equation}

\subparagraph*{Porém $V_a$ está em curto virtual com o terra do circuito. Então $V_a = 0$}

\begin{equation}
    \begin{aligned}
        V_c = - V_0
    \end{aligned}
\end{equation}

\subparagraph*{Daqui faremos análise nodal para determinar a EDO que precisaremos resolver.}

\begin{equation}
    \begin{aligned}
         & \frac{-V_i}{R_1} + \frac{-V_0}{R_2} + C \deriv{V_c}{t} = 0              \\
         & \deriv{V_c}{t} + \frac{V_0}{R_2 C} = \frac{\frac{-V_i R_2}{R_1}}{C R_2} \\
         & V_0(t) = V_0 e^{\frac{-t}{A}} + B(1- e^{\frac{-t}{A}})                  \\
         & I_c = C \deriv{V_c}{t}                                                  \\
         & A = C R_2 , B = \frac{-V_i R_2}{R_1}                                    \\
    \end{aligned}
\end{equation}

\subparagraph*{Então por fim obtemos uma equação que rege o comportamento da tensão no capacitor:}

\begin{equation}
    V_0(t) = V_0 e^{\frac{-t}{C R_2}} + \frac{-V_i R_2}{R_1}(1- e^{\frac{-t}{C R_2}})
\end{equation}

\subsection{Segundo circuito}

\subparagraph*{Neste teremos um capacitor em série com a resistência de entrada $R_1$.}

\subparagraph*{Similarmente o amp op deixará o no $V_a$ em curto virtual então teremos:}

\begin{equation}
    \begin{aligned}
         & \frac{-V_0}{R_2} + I_c = 0                    \\
         & -V_0 = C \deriv{V_c}{t}                       \\
         & V_0(0) = R2 * I_c                             \\
         & I_c = \frac{V_r}{R_1} = \frac{V_i - V_c}{R_1} \\
         & V_i = R_1 C \deriv{V_c}{t} + V_c              \\
         & \deriv{V_0}{t} + \frac{V_0}{C R_1} = 0        \\
         & A = C R_1 , B = 0                             \\
    \end{aligned}
\end{equation}

\subparagraph*{Neste caso, muito convenientemente nosso B é zero, a equação que rege a tensão no tempo fica:}

\begin{equation}
    V_0(t) = V_0 e^{\frac{-t}{C R_1}}
\end{equation}


\section{Medições no laboratório}

\paragraph*{Utilizaremos as seguintes resistências e tensões na montagem dos circuitos:}

\begin{equation*}
    \begin{aligned}
         & R_1 = 32.37 k\varOmega \\
         & R_2 = 46.70 k\varOmega \\
         & V_{c0} = -7.25 V       \\
         & V_1 = 5 V              \\
         & C = 100nf              \\
    \end{aligned}
\end{equation*}

\subsection{Primeiro circuito}

\subparagraph*{Para o primeiro circuito haviamos concluído que a constante de tempo era $C R_2$, logo:}

\begin{equation}
    f = \frac{1}{15*C R_2} = \frac{1}{ 0.0705} = 14.2 Hz
\end{equation}


\paragraph*{Comportamento no carregamento:}
\begin{center}
    \begin{tabular}{ |ccc| }
        \hline
        $Porcentagem$ & $Tensao$  & $Tempo$  \\
        $90\%$        & $-6.525V$ & $10.5ms$ \\
        $36.8\%$      & $-2.668V$ & $2.0ms$  \\
        $10\%$        & $-0.725$  & $0.5ms$  \\
        \hline
    \end{tabular}
\end{center}

\paragraph*{Comportamento no descarregamento:}
\begin{center}
    \begin{tabular}{ |ccc| }
        \hline
        $Porcentagem$ & $Tensao$  & $Tempo$   \\
        $90\%$        & $-6.525V$ & $0.35ms$  \\
        $36.8\%$      & $-2.668V$ & $2.10ms$  \\
        $10\%$        & $-0.725$  & $10.50ms$ \\
        \hline
    \end{tabular}
\end{center}

\subsubsection{Grafico do comportamento do circuito}

\begin{adjustbox}{scale=0.5}

    \begin{tikzpicture}
        \begin{axis}[
                ylabel={Tensao $V$},
                xlabel={Tempo $10^2 ms$},
                xmin = 0, xmax = 10,
                ymin = -1, ymax = 9.0,
                xtick distance = 2,
                ytick distance = 2,
                grid = both,
                minor tick num = 1,
                major grid style = {lightgray},
                minor grid style = {lightgray!25},
                width = 1\textwidth,
                height = 1\textwidth]
            \addplot[
                domain = -1:1,
                samples = 400,
                smooth,
                thick,
                red,
            ] {(7.25*(e^-(x+1)/0.3))};
            \addplot[
                domain = 1:3,
                samples = 400,
                smooth,
                thick,
                red,
            ] {(7.25*(1-e^-(x-1)/0.3))};
            \addplot[
                domain = 3:5,
                samples = 400,
                smooth,
                thick,
                red,
            ] {(7.25*(0+e^-(x-3)/0.3))};
            \addplot[
                domain = 5:7,
                samples = 400,
                smooth,
                thick,
                red,
            ] {(7.25*(1-e^-(x-5)/0.3))};
            \addplot[
                domain = 7:9,
                samples = 400,
                smooth,
                thick,
                red,
            ] {(7.25*(e^-(x-7)/0.3))};
            \addplot[
                domain = 9:10,
                samples = 400,
                smooth,
                thick,
                red,
            ] {(6*(1-e^-(x-9)/0.3))};
        \end{axis}
    \end{tikzpicture}
\end{adjustbox}

\subsection{Segundo circuito}

\subparagraph*{Para o segundo circuito haviamos concluído que a constante de tempo era $C R_1$, logo seguindo a mesma lógica para achar a frequência que vamos utilizar, obtemos $20Hz$.}

\paragraph*{Comportamento no carregamento:}
\begin{center}
    \begin{tabular}{ |ccc| }
        \hline
        $Porcentagem$ & $Tensao$ & $Tempo$ \\
        $90\%$        & $4.5V$   & $7.4ms$ \\
        $36.8\%$      & $3.16V$  & $3.1ms$ \\
        $10\%$        & $0.5$    & $0.5ms$ \\
        \hline
    \end{tabular}
\end{center}

\paragraph*{Comportamento no descarregamento:}
\begin{center}
    \begin{tabular}{ |ccc| }
        \hline
        $Porcentagem$ & $Tensao$  & $Tempo$  \\
        $90\%$        & $-6.525V$ & $0.35ms$ \\
        $36.8\%$      & $-2.668V$ & $3.20ms$ \\
        $10\%$        & $-0.725$  & $7.45ms$ \\
        \hline
    \end{tabular}
\end{center}

\subsubsection{Grafico do comportamento do circuito}

\begin{adjustbox}{scale=0.5}

    \begin{tikzpicture}
        \begin{axis}[
                ylabel={Tensao $V$},
                xlabel={Tempo $10^2 ms$},
                xmin = 0, xmax = 10,
                ymin = -1, ymax = 9.0,
                xtick distance = 2,
                ytick distance = 2,
                grid = both,
                minor tick num = 1,
                major grid style = {lightgray},
                minor grid style = {lightgray!25},
                width = 1\textwidth,
                height = 1\textwidth]
            \addplot[
                domain = -1:1,
                samples = 400,
                smooth,
                thick,
                red,
            ] {(7.25*(e^-(x+1)/0.3))};
            \addplot[
                domain = 1:3,
                samples = 400,
                smooth,
                thick,
                red,
            ] {(7.25*(1-e^-(x-1)/0.3))};
            \addplot[
                domain = 3:5,
                samples = 400,
                smooth,
                thick,
                red,
            ] {(7.25*(0+e^-(x-3)/0.3))};
            \addplot[
                domain = 5:7,
                samples = 400,
                smooth,
                thick,
                red,
            ] {(7.25*(1-e^-(x-5)/0.3))};
            \addplot[
                domain = 7:9,
                samples = 400,
                smooth,
                thick,
                red,
            ] {(7.25*(e^-(x-7)/0.3))};
            \addplot[
                domain = 9:10,
                samples = 400,
                smooth,
                thick,
                red,
            ] {(6*(1-e^-(x-9)/0.3))};
        \end{axis}
    \end{tikzpicture}
\end{adjustbox}


\end{document}