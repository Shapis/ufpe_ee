\documentclass[12pt,twoside, a4paper, twocolumn]{article}
\usepackage[utf8]{inputenc}
\usepackage[brazil]{babel}
\usepackage[margin = 0.5in]{geometry}
\usepackage{amsmath}
\usepackage{amssymb}
\usepackage{amsthm}
\usepackage{setspace}
\usepackage{circuitikz}
\usepackage{lipsum}
\usepackage{pgfplots}
\usepackage{ifthen}
\usepackage{adjustbox}

\newcommand{\drawvector} [9] [color=cyan] {
    \draw[line width=1.5pt,#1,-stealth](axis cs: #2, #3)--(axis cs: #4, #5) node[anchor=south west]{$#6$};

    

\ifthenelse{\equal{#7}{true}}{
    \draw[line width=1pt,#1, dashed](axis cs: #4, #5)--(axis cs: #4, 0) node[anchor= north west]{$#8$};
    \draw[line width=1pt,#1, dashed](axis cs: #4, #5)--(axis cs: 0, #5) node[anchor=south east]{$#9$};
    }
    {}
}


\title{Resumo Fisica 4 Primeira Unidade}
\author{Henrique da Silva \\ hpsilva@proton.me}
\date{\today}
\pgfplotsset{width = 10cm, compat = 1.9}


\begin{document}
\maketitle
\pagenumbering{gobble}
\newpage
%pagenumbering{roman}
\tableofcontents
\newpage

\newcommand\deriv[2]{\frac{\mathrm d #1}{\mathrm d #2}}

\section{Equacoes de Maxwell}

\subsection{Primeira Equacao}
\subparagraph{Essa equacao vem da lei de Gauss e diz que o fluxo eletrico eh dado pela seguinte equacao:
}
\begin{equation}
    \oint \vec{E} \, * \, \vec{dA} = \frac{q}{\epsilon_0}
\end{equation}
\subparagraph*{Na qual $\vec{E}$ eh o campo eletrico, $q$ eh a quantidade de carga envolvida, $\epsilon_0$ eh a permeabilidade do espaco vacuo e $\vec{dA}$ eh a area da superficie.}
\subparagraph*{Se o campo eletrico for constante sobre a superficie entao podemos dizer que:}
\begin{equation}
    E \, * \, A = \frac{q}{\epsilon_0}
\end{equation}
\subparagraph*{E tambem que:}
\begin{equation}
    E = \frac{V}{d}
\end{equation}
\subparagraph*{Este ultimo eh especialmente importante em questoes de corrente de conducao}
\paragraph*{E tambem que:}
\begin{equation}
    Capacitancia = \frac{\epsilon_0 * A}{d}
\end{equation}
\subparagraph*{E tambem que:}
\begin{equation}
    E = \frac{\rho*I}{A}
\end{equation}

\subsection{Segunda equacao}
\subparagraph*{Essa tambem eh uma forma da lei de Gauss mas para o fluxo magnetico, e eh dada por:}
\begin{equation}
    \oint  \vec{B} \, * \, \vec{dA} = 0
\end{equation}
\subparagraph*{Na qual $\vec{B}$ eh o campo magnetico, $\vec{dA}$ eh a area da superficie.}
\subparagraph*{E tem que necessariamente ser igual a zero \emph{em superficies fechadas} ja que o fluxo magnetico deve sempre sair por um polo e inteiramente voltar pela outro}

\subparagraph*{Vale tambem lembrar que o fluxo magnetico:}
\begin{equation}
    \phi_B = B  *  A \cos{\theta}
\end{equation}
\subparagraph*{E tambem que o campo magnetico para fios carregando corrente eh dado por:}
\begin{equation}
    B = \frac{\mu_0 * I}{2 * \pi * r}
\end{equation}
\subparagraph*{Lembrando que a corrente $I$ eh simplesmente o fluxo eletrico variando no tempo:}
\begin{equation}
    I = \deriv{\phi_E}{t}
\end{equation}


\subsection{Terceira equacao}
\subparagraph*{Essa tem a ver com a lei de Faraday sobre inducao, porem um pouco diferente.}
\subparagraph*{A lei como tinhamos visto era dada por:}
\begin{equation}
    \varepsilon = - \deriv{\phi_B}{t}
\end{equation}
\subparagraph*{No qual $\varepsilon$ eh a forca eletromotriz, $d\phi_B$ eh a mudanca no fluxo magnetico, e $dt$ eh a mudanca no tempo}
\subparagraph*{A lei de Faraday diz que um campo magnetico que muda com o tempo vai induzir uma forca eletromotriz em um fio enrolado}
\subparagraph*{A versao de Maxwell eh mais geral, simplificando a lei de Faraday}
\begin{equation}
    \oint \vec{E} * \vec{ds} = - \deriv{\phi_B}{t}
\end{equation}
\subparagraph*{Na qual $\vec{E}$ eh o campo eletrico, $\vec{ds}$ eh um elemento infinitesimal do loop fechado, $d\phi_B$ eh a mudanca do fluxo magnetico, e $dt$ eh a mudanca no tempo}
\subparagraph*{Com essa equacao Maxwell mostra a relacao de um campo magnetico que muda no tempo e de uma forca eletrica induzida}

\subsection{Quarta equacao}
\subparagraph*{Essa tem a ver com a lei de Ampere que diz que uma corrente que passa por um fio induz um campo magnetico ao redor do caminho ao redor do fio. A lei de Ampere como tinhamos visto era dada por:}
\begin{equation}
    \oint \vec{B} * \vec{ds} = \mu_0 * I_{enc}
\end{equation}
\subparagraph*{No qual $\vec{B}$ eh o campo magnetico, $\vec{ds}$ um pedaco infinitesimal do elemento do loop fechado, $\mu_0$ eh a permeabilidade do espaco para campos magneticos e finalmente, $I$ eh a corrente}
\subparagraph*{Eh importante lembrar que $I_{enc}$ eh a corrente dentro do loop fechado $\vec{ds}$, entao se for dado o fluxo total de um capacitor por exemplo. O fluxo que vamos considerar eh apenas o fluxo que esta dentra da superficie definida pelo loop fechado $\vec{ds}$}
\subparagraph*{O problema eh que ha uma geracao de campo magnetico induzido por uma descarga entre capacitores, onde nao ha fio nenhum conectando-os}
\subparagraph*{Maxwell resolveu isso pensando em algo que chamou de $I_D$, uma corrente de deslocamento, que na verdade nao eh exatamente uma corrente eletrica, mas eh apenas a mudanca das cargas dos capacitores no tempo, assim obtendo:}
\begin{equation}
    \int \vec{B} * \vec{ds} = \mu_0 * \left(I +I_D\right)
\end{equation}
\subparagraph*{Com o $I_D$ sendo igual a mudanca do fluxo eletrico no tempo, ou seja, a carga que passa de um capacitor para o outro, e assim nosso $I_D$ eh:}
\begin{equation}
    I_D = \epsilon_0 * \deriv{\phi_E}{t}
\end{equation}
\subparagraph*{Com $\epsilon_0$ a permeabilidade do espaco para campos electricos, e $\deriv{\phi_E}{t}$ a mudanca do fluxo eletrico no tempo}
\subparagraph*{Que finalmente nos da a forma integral da quarta equacao de Maxwell:}
\begin{equation}
    \int \vec{B} * \vec{ds} = \mu_0 * I + \mu_0 * \epsilon_0 * \deriv{\phi_E}{t}
\end{equation}

\section{Ondas eletromagneticas}
\subsection{Velocidade de propagacao}
\begin{equation}
    v = \frac{E}{B} = c
\end{equation}
\subparagraph*{Na qual $E$ eh o campo eletrico, $B$ eh o campo magnetico, $v$ eh a velocidade de propagacao e $c$ eh a velocidade da luz}
\subparagraph*{Que pode ser simplificada em:}
\begin{equation}
    v = \frac{1}{\sqrt[]{\epsilon_0 * \mu_0}} = c
\end{equation}
\subparagraph*{com $\epsilon_0$ a permeabilidade do espaco para campos electricos, e $\mu_0$ a permeabilidade do espaco para campos magneticos, $v$ a velocidade de propagacao da onde, e c a velocidade da luz}

\subparagraph*{Tambem temos que ha uma relacao entre frequencia, comprimento de onda e velocidade de propagacao, e eh dado por:}

\begin{equation}
    v = f * \lambda = c
\end{equation}

\subparagraph*{Na qual $f$ eh a frequencia, $\lambda$ eh o comprimento de onda, $v$ eh a velocidade de propagacao, e $c$ eh a velocidade da luz}
\subparagraph*{Ou seja facilmante conseguimos achar a frequencia dado o comprimento de onda e vice versa}

\subsection{Energia da onda}
\subparagraph*{Para calcular a energia da onda precisamos da magnitude do campo eletrico e magnetico, e temos as seguintes equacoes:}

\subparagraph*{Densidade de energia do campo eletrico:}
\begin{equation}
    \mu_E = \frac{1}{2} * \epsilon_0 * E^2
\end{equation}

\subparagraph*{Densidade de energia do campo magnetico:}
\begin{equation}
    \mu_B = \frac{1}{2} * \frac{B^2}{\mu_0}
\end{equation}

\subparagraph{Nos quais $E$ eh o campo eletrico, $B$ eh o campo magnetico, $\mu_E$ a densidade de energia do campo eletrico, $\mu_B$ a densidade de energia do campo magnetico, $\epsilon_0$ a permeabilidade do espaco para campos electricos, $\mu_0$ a permeabilidade do espaco para campos magneticos}
\subparagraph{A energia da onda eh a soma das duas energias:}
\begin{equation}
    \mu = \mu_E + \mu_B
\end{equation}
\begin{equation}
    \mu = \frac{1}{2} * \epsilon_0 * E^2 + \frac{1}{2} * \frac{B^2}{\mu_0}
\end{equation}
\subparagraph{Eh importante de tirar disso que podemos substituir o $B$ e o $E$ ja que temos uma relacao direta entre os dois que eh dada pela velocidade nas equacoes $(10)$ e $(11)$}

\subsection*{Intensidade}
\begin{equation}
    I = \frac{\Delta{U}}{A * \Delta{t}}
\end{equation}
\subparagraph*{Na qual $\Delta{U}$ eh a energia de um elemento infinitesimal da onda, $A$ eh a area da superficie que a onda cobre, $\Delta{t}$ eh um elemento infinitesimal de tempo, e $I$ eh a intensidade}
\subparagraph*{Vale a pena lembrar que $\Delta{U} = \mu * \Delta{V}$, e temos maneiras simples de calcular este $\mu$ como vimos acima em $(10)$}
\subparagraph*{Vale lembrar que potencia eh justamente $\frac{\Delta{U}}{\Delta{t}}$. O que nos deixa re-escrever de forma mais simples como:}
\begin{equation}
    I = \frac{P}{A}
\end{equation}
\subparagraph*{E muito comumente se consideramos a origem da onda como uma fonte pontual podemos dizer que a area eh a area da esfera que a envolve, ou seja $A = 4\pi * r^2$}
\subparagraph*{Simplificando finalmente chegamos em:}
\begin{equation}
    I = \mu_0 * c * E^2
\end{equation}
\subparagraph*{Forca eletromagnetica por um corpo complemanente absorvente eh dada por:}
\begin{equation}
    F_{em} = \frac{I * A}{c}
\end{equation}


\subsection{Vetor de Poynting}
\subparagraph*{O vetor de Poynting aponta na direcao de propagacao da onda eletromagnetica, e eh dado por:}
\begin{equation}
    \vec{S} = \frac{\vec{E} \times \vec{B}}{\mu_0}
\end{equation}
\subparagraph*{Ou seja:}
\begin{equation}
    S = \frac{E * B * \sin{\theta}}{\mu_0}
\end{equation}
\subparagraph*{Sua magnitude varia no tempo, e atinge seu maximo no mesmo instante que $\vec{E}$ e $\vec{B}$ atingem seus maximos}

\subsection{Polarizacao}
\paragraph*{Polarizacao acontece quando um raio eletromagnetico atravessa um campo magnetico polarizado, apenas deixando passar a as componentes do  raio eletromagnetico que estava na mesma polarizacao, o efeito disso eh que apenas metade do raio eletromagnetico passa pelo campo magnetico polarizado e a outra metade eh absorvida pelo filtro}

\section{Interferencia, reflexao e difracao}
\subsection{Lei da reflexao}
\subparagraph*{O angulo de incidencia de um raio de luz eh igual ao angulo de reflexao}
\subsection{Refracao e lei de Snell}
\subparagraph*{Eh o fenomeno de um raio de luz mudar de angulo quando passa de um meio para outro}
\subparagraph*{O angulo do raio incidente eh relacionado com o angulo do raio da refracao pela lei de Snell}
\subparagraph*{}

\begin{adjustbox}{scale=0.9}
    \begin{tikzpicture}
        \begin{axis}[
            clip = false,
            xmin=0, xmax=4,
            ymin=-2, ymax=2,
            axis lines=center,
            xlabel = $x$, ylabel=$y$,
            title={Lei de Snell},
            xtick={},
            xticklabels={}
            ytick={},
            yticklabels={}
            % xticklabel style = {anchor=south west},
            % xmajorgrids=true,
            % ymajorgrids=true,
            % grid style=dashed,
            ]


            % \addplot[domain=0:2*pi,color=blue, samples=100]{sin(deg(x))}
            % node[anchor=west, pos =0.7] {$3120$};



            \drawvector{0}{1}{1}{0}{}{false}{a}{b};

            \draw[line width=0pt,gray, dashed](axis cs: 1, 2)--(axis cs: 1, -2)  node[anchor= south east, pos =0.5] {$\rho$};


            \draw [cyan](axis cs: 1, 0.4) arc[start angle=90, end angle=135, radius=40]
            node[anchor= south east, pos =0.4] {$\theta_0$};

            \drawvector{1}{0}{1.5}{-1}{}{false}{a}{b};

            \draw [cyan](axis cs: 1, -0.4) arc[start angle=270, end angle=292.5, radius=40]
            node[anchor= north, pos =0.7] {$\theta_1$};


        \end{axis}
    \end{tikzpicture}
\end{adjustbox}



\begin{equation}
    \begin{aligned}
        \theta_0          & > \theta_1          \\
        n_0\sin{\theta_0} & = n_1\sin{\theta_1}
    \end{aligned}
\end{equation}

\subparagraph*{O indice de refracao $n$ eh dado por:}

\begin{equation}
    \begin{aligned}
        n = \frac{c}{v}
    \end{aligned}
\end{equation}

\subparagraph*{Nos quais $c$ eh a velocidade da luz no vacuo e $v$ eh a velocidade da luz no meio}\

\subparagraph*{O que equacao acima nos diz, eh que quanto maior o indice de refracao, menor o angulo desta refracao}

\subparagraph*{Vale ressaltar, que essa mudanca ocorre em qualquer mudanca de meio com indice de refracao $n$ diferentes}

\subsection{Imagens virtuais vs imagens reais}
\subparagraph*{Uma imagem eh dita real se os raios saidos do objeto convergem em um local}

\section{Lentes}

\subsection{Lente convergente/convexa}

\subparagraph*{Distancia focal $f$ de uma lente convexa esta a metade do raio de curvatura da lente}

\subparagraph*{Poder $P$ da lente}

\begin{equation}
    \begin{aligned}
        P = \frac{1}{f}
    \end{aligned}
\end{equation}

\subparagraph*{A imagem que passa por uma lente convergente sempre converge do lado oposto do objeto, e fica de cabeca pra baixo}

\subparagraph*{Equacao da lente fina relaciona, a distancia focal $f$, distancia da imagem $d_i$ e a distancia do objeto $d_0$:}

\begin{equation}
    \begin{aligned}
        \frac{1}{d_0} + \frac{1}{d_i} = \frac{1}{f}
    \end{aligned}
\end{equation}

\subsection{Lente divergente/concava}

\subparagraph*{Esse tipo de lente gera uma imagem virtual}
\subparagraph*{Na lente concava a distancia focal esta do mesmo lado do objeto}


\subsection{Equacao da magnificacao, relaciona a altura $h_i$ e $h_0$ do objeto com a distancia da imagem $d_i$ e a distancia do objecto $d_0$}

\begin{equation}
    \begin{aligned}
        m = \frac{h_i}{h_0} = - \frac{d_i}{d_0}
    \end{aligned}
\end{equation}

\end{document}