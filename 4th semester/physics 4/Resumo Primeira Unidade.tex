\documentclass[12pt,twoside, a4paper, twocolumn]{article}
\usepackage[utf8]{inputenc}
\usepackage[brazil]{babel}
\usepackage[margin = 0.5in]{geometry}
\usepackage{amsmath}
\usepackage{amssymb}
\usepackage{amsthm}
\usepackage{setspace}
\usepackage{circuitikz}
\usepackage{lipsum}
\usepackage{pgfplots}



\title{Resumo Fisica 4 Primeira Unidade}
\author{Henrique da Silva \\ hpsilva@proton.me}
\date{\today}
\pgfplotsset{width = 10cm, compat = 1.9}


\begin{document}
\maketitle
\pagenumbering{gobble}
\newpage
%pagenumbering{roman}
\tableofcontents
\newpage

\newcommand\deriv[2]{\frac{\mathrm d #1}{\mathrm d #2}}

\section{Primeira Eq. de Maxwell}
\paragraph{Essa equacao vem da lei de Gauss e diz que o fluxo eletrico eh dado pela seguinte equacao:
}
\begin{equation}
    \oint \vec{E} \, * \, \vec{dA} = \frac{q}{\epsilon_0}
\end{equation}
\paragraph*{Na qual $\vec{E}$ eh o campo eletrico, $q$ eh a quantidade de carga envolvida, $\epsilon_0$ eh a permeabilidade do espaco vacuo e $\vec{dA}$ eh a area da superficie.}
\paragraph*{Se o campo eletrico for constante sobre a superficie entao podemos dizer que:}
\begin{equation}
    E \, * \, A = \frac{q}{\epsilon_0}
\end{equation}


\section{Segunda Eq. de Maxwell}
\paragraph*{Essa tambem eh uma forma da lei de Gauss mas para o fluxo magnetico, e eh dada por:}
\begin{equation}
    \oint  \vec{B} \, * \, \vec{dA} = 0
\end{equation}
\paragraph*{Na qual $\vec{B}$ eh o campo magnetico, $\vec{dA}$ eh a area da superficie.}
\paragraph*{E tem que necessariamente ser igual a zero \emph{em superficies fechadas} ja que o fluxo magnetico deve sempre sair por um polo e inteiramente voltar pela outro}

\paragraph*{Vale tambem lembrar que o fluxo magnetico:}
\begin{equation}
    \phi = B  *  A \cos{\theta}
\end{equation}
\paragraph*{E tambem que o campo magnetico para fios carregando corrente eh dado por:}
\begin{equation}
    B = \frac{\mu_0 * I}{2 * \pi * r}
\end{equation}


\section{Terceira lei de Maxwell}
\paragraph*{Essa tem a ver com a lei de Faraday sobre inducao, porem um pouco diferente.}
\paragraph*{A lei como tinhamos visto era dada por:}
\begin{equation}
    \varepsilon = - \deriv{\phi_B}{t}
\end{equation}
\paragraph*{No qual $\varepsilon$ eh a forca eletromotriz, $d\phi_B$ eh a mudanca no fluxo magnetico, e $dt$ eh a mudanca no tempo}
\paragraph*{A lei de Faraday diz que um campo magnetico que muda com o tempo vai induzir uma forca eletromotriz em um fio enrolado}
\paragraph*{A versao de Maxwell eh mais geral, simplificando a lei de Faraday}
\begin{equation}
    \oint \vec{E} * \vec{ds} = - \deriv{\phi_B}{t}
\end{equation}
\paragraph*{Na qual $\vec{E}$ eh o campo eletrico, $\vec{ds}$ eh um elemento infinitesimal do loop fechado, $d\phi_B$ eh a mudanca do fluxo magnetico, e $dt$ eh a mudanca no tempo}
\paragraph*{Com essa equacao Maxwell mostra a relacao de um campo magnetico que muda no tempo e de uma forca eletrica induzida}

\section{Quarta lei de Maxwell}
\paragraph*{Essa tem a ver com a lei de Ampere que diz que uma corrente que passa por um fio induz um campo magnetico ao redor do caminho ao redor do fio }
\paragraph*{A lei de Ampere como tinhamos visto era dada por:}
\begin{equation}
    \int \vec{B} * \vec{ds} = \mu_0 * I
\end{equation}
\paragraph*{No qual $\vec{B}$ eh o campo magnetico, $\vec{ds}$ um pedaco infinitesimal do elemento do loop fechado, $\mu_0$ eh a permeabilidade do espaco para campos magneticos e finalmente, $I$ eh a corrente}
\paragraph*{O problema eh que ha uma geracao de campo magnetico induzido por uma descarga entre capacitores, onde nao ha fio nenhum conectando-os}
\paragraph*{Maxwell resolveu isso pensando em algo que chamou de $I_D$, uma corrente de deslocamento, que na verdade nao eh exatamente uma corrente eletrica, mas eh apenas a mudanca das cargas dos capacitores no tempo, assim obtendo:}
\begin{equation}
    \int \vec{B} * \vec{ds} = \mu_0 * \left(I +I_D\right)
\end{equation}
\paragraph*{Com o $I_D$ sendo igual a mudanca do fluxo eletrico no tempo, ou seja, a carga que passa de um capacitor para o outro, e assim nosso $I_D$ eh:}
\begin{equation}
    I_D = \epsilon_0 * \deriv{\phi_E}{t}
\end{equation}
\paragraph*{Com $\epsilon_0$ a permeabilidade do espaco para campos electricos, e $\deriv{\phi_E}{t}$ a mudanca do fluxo eletrico no tempo}
\paragraph*{Que finalmente nos da a forma integral da quarta equacao de Maxwell:}
\begin{equation}
    \int \vec{B} * \vec{ds} = \mu_0 * I + \mu_0 * \epsilon_0 * \deriv{\phi_E}{t}
\end{equation}

\section{Velocidade de propagacao}
\begin{equation}
    v = \frac{E}{B} = c
\end{equation}
\paragraph*{Na qual $E$ eh o campo eletrico, $B$ eh o campo magnetico, $v$ eh a velocidade de propagacao e $c$ eh a velocidade da luz}
\paragraph*{Que pode ser simplificada em:}
\begin{equation}
    v = \frac{1}{\sqrt[]{\epsilon_0 * \mu_0}} = c
\end{equation}
\paragraph*{com $\epsilon_0$ a permeabilidade do espaco para campos electricos, e $\mu_0$ a permeabilidade do espaco para campos magneticos, $v$ a velocidade de propagacao da onde, e c a velocidade da luz}

\paragraph*{Tambem temos que ha uma relacao entre frequencia, comprimento de onda e velocidade de propagacao, e eh dado por:}

\begin{equation}
    v = f * \lambda = c
\end{equation}

\paragraph*{Na qual $f$ eh a frequencia, $\lambda$ eh o comprimento de onda, $v$ eh a velocidade de propagacao, e $c$ eh a velocidade da luz}
\paragraph*{Ou seja facilmante conseguimos achar a frequencia dado o comprimento de onda e vice versa}

\section{Energia da onda}
\paragraph*{Para calcular a energia da onda precisamos da magnitude do campo eletrico e magnetico, e temos as seguintes equacoes:}

\subparagraph*{Densidade de energia do campo eletrico:}
\begin{equation}
    \mu_E = \frac{1}{2} * \epsilon_0 * E^2
\end{equation}

\subparagraph*{Densidade de energia do campo magnetico:}
\begin{equation}
    \mu_B = \frac{1}{2} * \frac{B^2}{\mu_0}
\end{equation}

\paragraph{Nos quais $E$ eh o campo eletrico, $B$ eh o campo magnetico, $\mu_E$ a densidade de energia do campo eletrico, $\mu_B$ a densidade de energia do campo magnetico, $\epsilon_0$ a permeabilidade do espaco para campos electricos, $\mu_0$ a permeabilidade do espaco para campos magneticos}
\paragraph{A energia da onda eh a soma das duas energias:}
\begin{equation}
    \mu = \mu_E + \mu_B
\end{equation}
\begin{equation}
    \mu = \frac{1}{2} * \epsilon_0 * E^2 + \frac{1}{2} * \frac{B^2}{\mu_0}
\end{equation}
\paragraph{Eh importante de tirar disso que podemos substituir o $B$ e o $E$ ja que temos uma relacao direta entre os dois que eh dada pela velocidade nas equacoes $(10)$ e $(11)$}

\section*{Intensidade}
\begin{equation}
    I = \frac{\Delta{U}}{A * \Delta{t}}
\end{equation}
\paragraph*{Na qual $\Delta{U}$ eh a energia de um elemento infinitesimal da onda, $A$ eh a area da superficie que a onda cobre, $\Delta{t}$ eh um elemento infinitesimal de tempo, e $I$ eh a intensidade}
\paragraph*{Vale a pena lembrar que $\Delta{U} = \mu * \Delta{V}$, e temos maneiras simples de calcular este $\mu$ como vimos acima em $(10)$}
\paragraph*{Vale lembrar que potencia eh justamente $\frac{\Delta{U}}{\Delta{t}}$. O que nos deixa re-escrever de forma mais simples como:}
\begin{equation}
    I = \frac{P}{A}
\end{equation}
\paragraph*{E muito comumente se consideramos a origem da onda como uma fonte pontual podemos dizer que a area eh a area da esfera que a envolve, ou seja $A = 4\pi * r^2$}
\paragraph*{Simplificando finalmente chegamos em:}
\begin{equation}
    I = \mu_0 * c * E^2
\end{equation}






\end{document}