\documentclass[12pt,twoside, a4paper, twocolumn]{article}
\usepackage[utf8]{inputenc}
\usepackage[brazil]{babel}
\usepackage[margin = 0.5in]{geometry}
\usepackage{amsmath}
\usepackage{amsthm}
\usepackage{amssymb}
\usepackage{amsthm}
\usepackage{setspace}
\usepackage[americanvoltages,fulldiodes,siunitx]{circuitikz}
\usepackage{lipsum}
\usepackage{pgfplots}
\usepackage{ifthen}
\usepackage{adjustbox}
\usepackage[section]{placeins}
\usepackage{hyperref}
\usepackage{graphicx}
\usepackage{amsmath}
\usepackage{amsthm}
\usepackage{amssymb}
\usepackage{amsthm}
\usepackage{setspace}
\usepackage[americanvoltages,fulldiodes,siunitx]{circuitikz}
\usepackage{lipsum}
\usepackage{pgfplots}
\usepackage{ifthen}
\usepackage{adjustbox}
\usepackage[section]{placeins}
\usepackage{hyperref}
\usepackage{graphicx}
\usepackage{adjustbox}
\usepackage{indentfirst}

\pgfplotsset{compat=newest}
\graphicspath{ {./images/} }
%  #1 color - optional #2 x_0 #3 y_0 #4 x_f #5 y_f #6 name - optional  #7 true if adding lines to axis
\newcommand{\drawvector} [9] [color=cyan] {
\draw[line width=1.5pt,#1,-stealth](axis cs: #2, #3)--(axis cs: #4, #5) node[anchor=south west]{$#6$};
\ifthenelse{\equal{#7}{true}}{
\draw[line width=1pt,#1, dashed](axis cs: #4, #5)--(axis cs: #4, 0) node[anchor= north west]{$#8$};
\draw[line width=1pt,#1, dashed](axis cs: #4, #5)--(axis cs: 0, #5) node[anchor=south east]{$#9$};
}
{}
}
\newcommand\deriv[2]{\frac{\mathrm d #1}{\mathrm d #2}}
\title{Quinto Relatório de Lab de Circuitos II}
\author{Henrique da Silva \\ hpsilva@proton.me}
\date{\today}
\pgfplotsset{width = 10cm, compat = 1.9}
\begin{document}
\maketitle
\pagenumbering{gobble}
\newpage
%pagenumbering{roman}
\tableofcontents
\newpage


\section{Introdução}

Neste relatório, vamos discutir filtros de Butterworth, em particular, vamos projetar, montar e testar um filtro de Butterworth ativo passa-baixa de quarta ordem.

Todos arquivos utilizados para criar este relatório, e o relatorio em si estão em:  \url{https://github.com/Shapis/ufpe_ee/tree/main/5th semester/Circuits II/}

\section{Análise preliminar}

Utilizarei o Maxima para fazer a análise teórica do circuito antes de montá-lo fisicamente.

Após terminar as análises compararei os resultados obtidos nas análises numéricas e em laboratório para verificar sua coerência.

\subsection{O circuito}

\begin{figure}[h]
    \centering
    \includegraphics[width=1\columnwidth]{images/circuito.png}
    \caption{Filtro ativo Butterworth de segunda ordem.}
\end{figure}

\pagebreak

\subsection{Maxima}

Podemos realizar a análise do circuito utilizando analise nodal.


\begin{equation}
    \begin{aligned}
         & \frac{V_a - V_i}{R} + \left(V_a - V_o\right) s C1 + \frac{V_a - V_o}{R} = 0 \\
         & \frac{V_o - V_a}{R_2} + V_o s C_2 = 0
    \end{aligned}
\end{equation}

Resolvendo simbolicamente no Maxima obtemos o seguinte:

\begin{equation}
    \frac{\frac{1}{C_1 C_2 R^2}}{s^2 + \frac{2s}{C_1 R} + \frac{1}{C_1 C_2 R^2}}
\end{equation}

Daqui vemos que temos um circuito passa-baixa com os seguintes parametros:

\begin{equation}
    \begin{aligned}
         & \omega_c^2 = \frac{1}{C_1 C_2 R^2} \\
         & \beta = \frac{2}{C_1 R}
    \end{aligned}
\end{equation}

E ja que estamos tratando de um filtro Butterworth de quarta ordem, precisamos analisar as projecoes das raizes dos seus polos no eixo real.

No caso, so observaremos as do segundo quadrante, ja que as suas projecoes serao iguais as das raizes que se encontram no terceiro quadrante.

\begin{equation}
    \begin{aligned}
         & \beta_1 = 2 \left| \cos{\frac{5 \pi}{8}} \right| = 0.765367 \\
         & \beta_2 = 2 \left| \cos{\frac{7 \pi}{8}}\right| = 1.847759  \\
    \end{aligned}
\end{equation}

Com esta informacao podemos projetar os filtros prototipos que utilizaremos em serie para obter nosso filtro de quarta ordem.

\subsubsection{Filtro 1}

Neste filtro chamaremos seus resistores de $R_1$ seus capacitores de $C_1$ e $C_2$.

\begin{equation}
    \begin{aligned}
         & \beta_1 = 0.765367 = \frac{2}{C_1 R}   \\
         & \omega_c^2 = 1 = \frac{1}{C_1 C_2 R^2} \\
         & R = 1;
    \end{aligned}
\end{equation}

Resolvendo para essas equacoes teremos:

\begin{equation}
    \begin{aligned}
         & R_1 = 1 \varOmega \\
         & C_1 = 2.6131 F    \\
         & C_2 = 0.38268 F
    \end{aligned}
\end{equation}

Agora vamos realizar o escalonamento para $50Hz$, ja que o nosso circuito prototipo foi projetado para $1Hz$.

Utilizaremos as seguintes equacoes de escalonamento:

\begin{equation}
    \begin{aligned}
        R' = R  k_m \\
        C' = \frac{C}{k_m k_f}
    \end{aligned}
\end{equation}

Com $k_f = 50$ para escalonarmos a frequencia central para $50Hz$ e escolheremos um valor de $k_m = 8.3 * 10^4$ para utilizarmos apenas um unico resistor com valor comercial.

Entao obteremos os seguintes componentes que serao usados para montar o filtro:

\begin{equation}
    \begin{aligned}
         & R_1 = 83k \varOmega \\
         & C_1 = 100 nF        \\
         & C_2 = 14.7 nF
    \end{aligned}
\end{equation}


\subsubsection{Filtro 2}

Neste filtro chamaremos seus resistores de $R_2$ seus capacitores de $C_3$ e $C_4$.

\begin{equation}
    \begin{aligned}
         & \beta_2 = 1.847759 = \frac{2}{C_3 R}   \\
         & \omega_c^2 = 1 = \frac{1}{C_3 C_4 R^2} \\
         & R = 1;
    \end{aligned}
\end{equation}

Resolvendo para essas equacoes teremos:

\begin{equation}
    \begin{aligned}
         & R_2 = 1 \varOmega \\
         & C_3 = 1.08239 F   \\
         & C_4 = 0.92388 F
    \end{aligned}
\end{equation}

Agora vamos realizar o escalonamento para $50Hz$, ja que o nosso circuito prototipo foi projetado para $1Hz$.

Utilizaremos as seguintes equacoes de escalonamento:

\begin{equation}
    \begin{aligned}
        R' = R  k_m \\
        C' = \frac{C}{k_m k_f}
    \end{aligned}
\end{equation}

Com $k_f = 50$ para escalonarmos a frequencia central para $50Hz$ e escolheremos um valor de $k_m = 6.2 * 10^4$ para utilizarmos apenas um unico resistor com valor comercial.

Entao obteremos os seguintes componentes que serao usados para montar o filtro:

\begin{equation}
    \begin{aligned}
         & R_2 = 62k \varOmega \\
         & C_3 = 55.6 nF       \\
         & C_4 = 47.4 nF
    \end{aligned}
\end{equation}

\subsubsection{Filtro combinado}

Podemos entao combinar os dois filtros de Butterworth de segunda ordem que projetamos em serie para obter um filtro de quarta ordem com a seguinte funcao de transferencia:

\begin{equation}
    \begin{aligned}
         & H(s) = \frac{\omega_c^4}{\left( s^2 + \beta_1 s + \omega_c^2 \right) \left( s^2 + \beta_2 s + \omega_c^2 \right)} \\
         & \beta_1 = 0.765367                                                                                                \\
         & \beta_2 = 1.847759                                                                                                \\
         & \omega_c = 2 \pi 50 = 314.159265                                                                                  \\
    \end{aligned}
\end{equation}


\subsubsection{Grafico de Bode}

\begin{figure}[h]
    \centering
    \includegraphics[width=1\columnwidth]{images/bodegain.png}
    \caption{Magnitude de H(s) do filtro.}
\end{figure}

Temos $80dB$ de perda em $500Hz$ e $160dB$ de perda em $5000Hz$.

Cada filtro de Butterworth tem $40dB$ de perda por decada. Como temos dois em serie, perdemos $80dB$ por decada.

Em outras palavras. temos $20dB$ de perda por polo por decada. Como temos 4 polos, perdemos $80dB$ por decada.

Entao o que observamos eh coerente.

\pagebreak

\section{Medições em laboratório}


\subsection{Os Circuitos}

Inicialmente farei as medições dos componentes a serem usados.

Apos isso farei um breve teste em cada um dos filtros de segunda ordem individualmente.

Por fim, combinarei os dois filtros no nosso filtro de quarta  ordem que queremos analisar.

\subsection{Tabela de componentes}

\subsubsection*{Filtro 1}
\begin{equation}
    \begin{aligned}
        R_1 & = 81.1k \varOmega \\
        R_2 & = 81k \varOmega   \\
        C_1 & = 100 nF          \\
        C_2 & = 14.7 nF         \\
    \end{aligned}
\end{equation}

\subsubsection*{Filtro 2}
\begin{equation}
    \begin{aligned}
        R_3 & = 61.8k \varOmega \\
        R_4 & = 61.2k \varOmega \\
        C_3 & = 55.5 nF         \\
        C_4 & = 47.8 nF         \\
    \end{aligned}
\end{equation}

\subsection{Graficos de Bode dos filtros reais}

Substitui os valores reais no Maxima para obter graficos de Bode dos filtros reais. Para observar o comportamento real esperado.

\subsubsection{Filtro 1}

\begin{figure}[h]
    \centering
    \includegraphics[width=1\columnwidth]{images/bodegainH1.png}
    \caption{Magnitude de H(s) do filtro 1.}
\end{figure}

Notemos que ha um ganho consideravel antes de comecar a filtrar frequencias altas.

Achamos uma frequencia de corte de $72Hz$.

\subsubsection{Filtro 2}

\begin{figure}[h]
    \centering
    \includegraphics[width=1\columnwidth]{images/bodegainH2.png}
    \caption{Magnitude de H(s) do filtro 2.}
\end{figure}

Achamos uma frequencia de corte de $54Hz$.

\subsubsection{Filtro total}

\begin{figure}[h]
    \centering
    \includegraphics[width=1\columnwidth]{images/bodegainHtotal.png}
    \caption{Magnitude de H(s) do filtro total.}
\end{figure}

Achamos uma frequencia de corte de $59Hz$.


\pagebreak



\subsection{Comportamento na frequencia}

Em frequencias baixas constatamos que o ganho permanecia 1. E achamos sua frequencia de corte de fato em 59Hz. Como esperavamos da analise numerica com os valores reais.

Notamos tambem que ha um pequeno ganho antes de atingirmos a frequencia de corte. Este comportamento pode ser observado no grafico de Bode do filtro 1 que fizemos acima na secao (3.3.1).

As fotos do osciloscopio se encontram na pasta do relatorio em "images/osciloscopio/*".


\subsection{Resultados das medidas}
\begin{center}
    \begin{tabular}{ |c|c|c|c| }
        \hline
        Múltiplos & Freq (Hz) & Entrada (V) & Saída (V) \\
        0.25      & $14.75$   & $5.11$      & $5.15$    \\
        0.5       & $29.5$    & $5.11$      & $5.47$    \\
        0.75      & $44.25$   & $5.11$      & $5.39$    \\
        1         & $59$      & $5.11$      & $3.62$    \\
        1.25      & $73.75$   & $5.11$      & $2.01$    \\
        1.6       & $94.4$    & $5.11$      & $0.97$    \\
        2.5       & $147.5$   & $5.11$      & $0.3$     \\
        \hline
    \end{tabular}
\end{center}


\newpage


\section{Pós-laboratorial}

As tabelas estao na secao 3.2 e 3.5.

\subsection{Funcao transferencia de todos filtros envolvidos.}

\subsubsection{Filtro 1}

\begin{figure}[h]
    \centering
    \includegraphics[width=1\columnwidth]{images/hsfiltro1.png}
    \caption{H(s) do filtro 1 com valores reais.}
\end{figure}

\begin{figure}[h]
    \centering
    \includegraphics[width=1\columnwidth]{images/zoomH1.png}
    \caption{Zoom do grafico de Bode do filtro 1 com valores reais.}
\end{figure}

Daqui tiramos uma frequencia de corte de $71.1Hz$ a partir do grafico de Bode desta funcao.


\subsubsection{Filtro 2}

\begin{figure}[h]
    \centering
    \includegraphics[width=1\columnwidth]{images/hsfiltro2.png}
    \caption{H(s) do filtro 2 com valores reais.}
\end{figure}

\begin{figure}[h]
    \centering
    \includegraphics[width=1\columnwidth]{images/zoomH2.png}
    \caption{Zoom do grafico de Bode do filtro 2 com valores reais.}
\end{figure}

Daqui tiramos uma frequencia de corte de $35.9Hz$ a partir do grafico de Bode desta funcao.


\pagebreak
\subsubsection{Filtro total}

\begin{figure}[h]
    \centering
    \includegraphics[width=1\columnwidth]{images/hsfiltrototal.png}
    \caption{H(s) do filtro total com valores reais.}
\end{figure}

\begin{figure}[h]
    \centering
    \includegraphics[width=1\columnwidth]{images/zoomHtotal.png}
    \caption{Zoom do grafico de Bode do filtro total com valores reais.}
\end{figure}

Daqui tiramos uma frequencia de corte de $50.5Hz$ a partir do grafico de Bode desta funcao.


\pagebreak

\subsection{Ganho do filtro}

Observamos um ganho unitario, que ocorre quando $s = jw$ com $w$ tendendo a $0$.

Que eh o resultado esperado para filtro passa-baixa.


\subsection{Ganho do filtro}

\begin{center}
    \begin{tabular}{ |c|c|c|c| }
        \hline
        Múltiplos & Freq (Hz) & $\left| H(jw) \right|$ \\
        0.25      & $14.75$   & $1.0078$               \\
        0.5       & $29.5$    & $1.0704$               \\
        0.75      & $44.25$   & $1.0547$               \\
        1         & $59$      & $0.7084$               \\
        1.25      & $73.75$   & $0.3933$               \\
        1.6       & $94.4$    & $0.1898$               \\
        2.5       & $147.5$   & $0.0587$               \\
        \hline
    \end{tabular}
\end{center}

\begin{figure}[h]
    \centering
    \includegraphics[width=1\columnwidth]{images/magnitudepontosreais.png}
    \caption{Grafico da curva esperada pelos pontos encontrados.}
\end{figure}

Ha uma discrepancia, que ja haviamos detectado. Em 4.1.3 vimos que a frequencia de corte esperada com os valores reais dos componentes eh de $50.5Hz$, e a que encontramos experimentalmente foi de $59Hz$.

Eu nao tenho certeza de pro que houve essa discrepancia. Creio que seja por conta de erro de medicao dos valores dos componentes.

\newpage


\section{Conclusões}

Conseguimos com sucesso fazer a análise numérica pelo Maxima, e comparamos os resultados com os obtidos experimentalmente.

Nos resultados práticos, a magnitude da função transferência e as frequências de corte foram coerentes com os resultados esperados.

Porem, houve o erro evidenciado em 4.3, de que a frequencia de corte esperada pelos valores de componentes medidos era de $50.5Hz$, e a que de fato achamos foi de $59Hz$.

Creio que por conta de erro de medicao dos valores dos componentes.

Os gráficos que geramos a partir dos resultados experimentais foram coerentes com os gráficos gerados numericamente.

Em suma creio que tivemos sucesso em nos familiarizar com as ferramentas de análise de circuitos elétricos numéricos.

\end{document}