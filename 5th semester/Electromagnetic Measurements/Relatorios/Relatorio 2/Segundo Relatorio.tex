\documentclass[12pt,twoside, a4paper, twocolumn]{article}
\usepackage[utf8]{inputenc}
\usepackage[brazil]{babel}
\usepackage[margin = 0.5in]{geometry}
\usepackage{amsmath}
\usepackage{amsthm}
\usepackage{amssymb}
\usepackage{amsthm}
\usepackage{setspace}
\usepackage[americanvoltages,fulldiodes,siunitx]{circuitikz}
\usepackage{lipsum}
\usepackage{pgfplots}
\usepackage{ifthen}
\usepackage{adjustbox}
\usepackage[section]{placeins}
\usepackage{hyperref}
\usepackage{graphicx}
\usepackage{amsmath}
\usepackage{amsthm}
\usepackage{amssymb}
\usepackage{amsthm}
\usepackage{setspace}
\usepackage[americanvoltages,fulldiodes,siunitx]{circuitikz}
\usepackage{lipsum}
\usepackage{pgfplots}
\usepackage{ifthen}
\usepackage{adjustbox}
\usepackage[section]{placeins}
\usepackage{hyperref}
\usepackage{graphicx}
\usepackage{adjustbox}
 
\pgfplotsset{compat=newest}
 
\graphicspath{ {./images/} }
 
%  #1 color - optional #2 x_0 #3 y_0 #4 x_f #5 y_f #6 name - optional  #7 true if adding lines to axis
 
\newcommand{\drawvector} [9] [color=cyan] {
   \draw[line width=1.5pt,#1,-stealth](axis cs: #2, #3)--(axis cs: #4, #5) node[anchor=south west]{$#6$};
 
  
 
\ifthenelse{\equal{#7}{true}}{
   \draw[line width=1pt,#1, dashed](axis cs: #4, #5)--(axis cs: #4, 0) node[anchor= north west]{$#8$};
   \draw[line width=1pt,#1, dashed](axis cs: #4, #5)--(axis cs: 0, #5) node[anchor=south east]{$#9$};
   }
   {}
}
 
\newcommand\deriv[2]{\frac{\mathrm d #1}{\mathrm d #2}}
 
 
\title{Segundo Relatório de Medidas Eletromagnéticas}
\author{Gabriel Soares \\ Henrique da Silva}
\date{\today}
\pgfplotsset{width = 10cm, compat = 1.9}
 
 
\begin{document}
\maketitle
\pagenumbering{gobble}
\newpage
%pagenumbering{roman}
\tableofcontents
\newpage



\section{Introdução}


\subparagraph*{Neste relatório, vamos medir os valores de resistência $\varOmega$ e capacitância $F$ de resistores e capacitores, a fim de compararmos com os valores verdadeiros convencionais, e calcularemos alguns de seus parâmetros estatísticos.}

\subparagraph*{Todos arquivos utilizados para criar este relatorio, e o relatorio em si estão em:  \url{https://github.com/Shapis/ufpe_ee/tree/main/5th semester/Electromagnetic Measurements/Relatorios}}




\subsection{Análise preliminar}
\subparagraph*{}


\subparagraph*{Utilizaremos um multímetro para medir as capacitâncias e resistências de alguns componentes.}

\subparagraph*{Faremos $20$ medições em cada componente, e calcularemos a média, desvio padrão, tendência e correção de cada um deles.}

\subparagraph*{Após isso discutiremos os nossos resultados.}

\section{Resultados esperados}

\subsection{Resistor}

\subparagraph*{Esperamos resultados consistentes entre as medidas, porém também esperamos que a resistência seja diferente da resistência de fábrica.}

\subparagraph*{Isso ocorreu por desgaste dos componentes devido a seu uso de laboratório, e também pela qualidade dos componentes.}

\subparagraph*{Muito provavelmente estamos fora dos padrões de confiabilidades de fábrica. Mas precisaríamos ver o \emph{datasheet} dos resistores em específico para confirmar isso.}

\subsection{Capacitor}

\subparagraph*{Tudo o que falamos acima se aplica aos capacitores, mas com dois diferenciais.}

\subparagraph*{O primeiro é que estes sao mais sensíveis ao uso, logo esperaremos discrepâncias maiores entre os valores de fábrica e os de fato.}

\subparagraph*{E também que, durante as medidas, os carregaremos e descarregaremos, o que implica também em um erro sistemático adicional.}


\section{Medições no laboratório}

\subparagraph*{Para reduzir erros, encaixaremos todos componentes em um \emph{protoboard}.}

\subparagraph*{Antes de fazer as medidas dos capacitores, vamos criar um circuito com um capacitor e um resistor em série para descarregá-los. Apos alguns segundos com esse circuito formado, o desconectaremos e faremos a medição da capacitância.}

\subsection{Tabelas de medições}

\subsubsection{Resistores}

\subparagraph*{Mediremos três resistores com valores de fabrica respectivamente de: $R_1 = 10k\varOmega$, $R_2 = 22k\varOmega$, $R_3 = 15k\varOmega$.}
\begin{center}
    \begin{tabular}{ |c|c|c| }
        \hline
        $R_1$ $10k\varOmega$ & $R_2$ $22k\varOmega$ & $R_3$ $15k\varOmega$ \\
        10,370 k$\varOmega$  & 21,932 k$\varOmega$  & 14,848 k$\varOmega$  \\
        10,370 k$\varOmega$  & 21,932 k$\varOmega$  & 14,849 k$\varOmega$  \\
        10,380 k$\varOmega$  & 21,932 k$\varOmega$  & 14,850 k$\varOmega$  \\
        10,380 k$\varOmega$  & 21,932 k$\varOmega$  & 14,849 k$\varOmega$  \\
        10,380 k$\varOmega$  & 21,932 k$\varOmega$  & 14,850 k$\varOmega$  \\
        10,370 k$\varOmega$  & 21,933 k$\varOmega$  & 14,849 k$\varOmega$  \\
        10,370 k$\varOmega$  & 21,933 k$\varOmega$  & 14,849 k$\varOmega$  \\
        10,370 k$\varOmega$  & 21,931 k$\varOmega$  & 14,850 k$\varOmega$  \\
        10,370 k$\varOmega$  & 21,931 k$\varOmega$  & 14,850 k$\varOmega$  \\
        10,370 k$\varOmega$  & 21,930 k$\varOmega$  & 14,848 k$\varOmega$  \\
        10,360 k$\varOmega$  & 21,932 k$\varOmega$  & 14,849 k$\varOmega$  \\
        10,370 k$\varOmega$  & 21,932 k$\varOmega$  & 14,849 k$\varOmega$  \\
        10,370 k$\varOmega$  & 21,932 k$\varOmega$  & 14,849 k$\varOmega$  \\
        10,370 k$\varOmega$  & 21,932 k$\varOmega$  & 14,849 k$\varOmega$  \\
        10,380 k$\varOmega$  & 21,934 k$\varOmega$  & 14,849 k$\varOmega$  \\
        10,360 k$\varOmega$  & 21,934 k$\varOmega$  & 14,850 k$\varOmega$  \\
        10,360 k$\varOmega$  & 21,934 k$\varOmega$  & 14,849 k$\varOmega$  \\
        10,370 k$\varOmega$  & 21,933 k$\varOmega$  & 14,849 k$\varOmega$  \\
        10,360 k$\varOmega$  & 21,934 k$\varOmega$  & 14,849 k$\varOmega$  \\
        10,360 k$\varOmega$  & 21,932 k$\varOmega$  & 14,848 k$\varOmega$  \\

        \hline
    \end{tabular}
\end{center}

\begin{center}
    \begin{tabular}{ |c|c|c|c| }
        \hline
                      & $R_1$ $10k\varOmega$ & $R_2$ $22k\varOmega$ & $R_3$ $15k\varOmega$ \\
        Média         & 10,37 k$\varOmega$   & 21,932 k$\varOmega$  & 14,849 k$\varOmega$  \\
        Desvio padrão & 0,00069 k$\varOmega$ & 0,0011 k$\varOmega$  & 0,00064 k$\varOmega$ \\
        Tendência     & 0,37 k$\varOmega$    & -0,068 k$\varOmega$  & -0,151 k$\varOmega$  \\
        Correção      & -0,37 k$\varOmega$   & 0,068 k$\varOmega$   & 0,151 k$\varOmega$   \\


        \hline
    \end{tabular}
\end{center}

\subsubsection{Capacitores}

\subparagraph*{Mediremos três capacitores com valores de fábrica respectivamente de: $C_1 = 100 nF$, $C_2 = 47 nF$, $R_3 = 10 nF$.}
\begin{center}
    \begin{tabular}{ |c|c|c| }
        \hline
        $C_1 = 100 nF$ & $C_2 = 47 nF$ & $R_3 = 10 nF$ \\
        46,31 $nF$     & 55,92 $nF$    & 12,74 $nF$    \\
        46,45 $nF$     & 55,70 $nF$    & 12,72 $nF$    \\
        46,34 $nF$     & 55,66 $nF$    & 12,77 $nF$    \\
        46,34 $nF$     & 55,87 $nF$    & 12,76 $nF$    \\
        46,25 $nF$     & 56,09 $nF$    & 12,78 $nF$    \\
        46,36 $nF$     & 55,85 $nF$    & 12,77 $nF$    \\
        46,21 $nF$     & 55,90 $nF$    & 12,74 $nF$    \\
        46,32 $nF$     & 55,76 $nF$    & 12,80 $nF$    \\
        46,30 $nF$     & 55,94 $nF$    & 12,83 $nF$    \\
        46,54 $nF$     & 55,72 $nF$    & 12,84 $nF$    \\
        46,54 $nF$     & 55,69 $nF$    & 12,79 $nF$    \\
        47,01 $nF$     & 55,78 $nF$    & 12,81 $nF$    \\
        46,70 $nF$     & 55,75 $nF$    & 12,78 $nF$    \\
        46,82 $nF$     & 55,85 $nF$    & 12,80 $nF$    \\
        46,75 $nF$     & 55,82 $nF$    & 12,81 $nF$    \\
        46,64 $nF$     & 55,43 $nF$    & 12,79 $nF$    \\
        46,71 $nF$     & 55,40 $nF$    & 12,76 $nF$    \\
        46,76 $nF$     & 55,39 $nF$    & 12,73 $nF$    \\
        46,85 $nF$     & 55,64 $nF$    & 12,69 $nF$    \\
        46,81 $nF$     & 55,68 $nF$    & 12,68 $nF$    \\

        \hline
    \end{tabular}
\end{center}

\begin{center}
    \begin{tabular}{ |c|c|c|c| }
        \hline
                      & $C_1 100 nF$ & $C_2 47 nF$ & $C_3 10 nF$ \\
        Média         & 46,55 $nF$   & 55,74 $nF$  & 12,77 $nF$  \\
        Desvio padrão & 0,2401 $nF$  & 0,1819 $nF$ & 0,0430 $nF$ \\
        Tendência     & -53,45 $nF$  & 8,742 $nF$  & 2,770 $nF$  \\
        Correção      & 53,45 $nF$   & -8,742 $nF$ & -2,770 $nF$ \\
        \hline
    \end{tabular}
\end{center}

\section{Conclusões}

\subparagraph*{Obtivemos desvios padrões baixos para nossos componentes. Porém, especificamente no caso dos capacitores, as tendências foram bastante elevadas, o que indica que uma calibração é necessária.}

\subparagraph*{A realização de sucessivas medições e obtenção de parâmetros como média e desvio padrão é de grande interesse para maximização da confiança na obtenção de grandezas. Com os valores verdadeiros convencionais delas, é possível obter também tendência e correção.}


\end{document}
