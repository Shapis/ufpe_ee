\documentclass[12pt,twoside, a4paper, twocolumn]{article}
\usepackage[utf8]{inputenc}
\usepackage[brazil]{babel}
\usepackage[margin = 0.5in]{geometry}
\usepackage{amsmath}
\usepackage{amsthm}
\usepackage{amssymb}
\usepackage{amsthm}
\usepackage{setspace}
\usepackage[americanvoltages,fulldiodes,siunitx]{circuitikz}
\usepackage{lipsum}
\usepackage{pgfplots}
\usepackage{ifthen}
\usepackage{adjustbox}
\usepackage[section]{placeins}
\usepackage{hyperref}
\usepackage{graphicx}
\usepackage{amsmath}
\usepackage{amsthm}
\usepackage{amssymb}
\usepackage{amsthm}
\usepackage{setspace}
\usepackage[americanvoltages,fulldiodes,siunitx]{circuitikz}
\usepackage{lipsum}
\usepackage{pgfplots}
\usepackage{ifthen}
\usepackage{adjustbox}
\usepackage[section]{placeins}
\usepackage{hyperref}
\usepackage{graphicx}
\usepackage{adjustbox}
 
\pgfplotsset{compat=newest}
 
\graphicspath{ {./images/} }
 
%  #1 color - optional #2 x_0 #3 y_0 #4 x_f #5 y_f #6 name - optional  #7 true if adding lines to axis
 
\newcommand{\drawvector} [9] [color=cyan] {
   \draw[line width=1.5pt,#1,-stealth](axis cs: #2, #3)--(axis cs: #4, #5) node[anchor=south west]{$#6$};
 
  
 
\ifthenelse{\equal{#7}{true}}{
   \draw[line width=1pt,#1, dashed](axis cs: #4, #5)--(axis cs: #4, 0) node[anchor= north west]{$#8$};
   \draw[line width=1pt,#1, dashed](axis cs: #4, #5)--(axis cs: 0, #5) node[anchor=south east]{$#9$};
   }
   {}
}
 
\newcommand\deriv[2]{\frac{\mathrm d #1}{\mathrm d #2}}
 
 
\title{Primeiro Relatório de Medidas Eletromagneticas}
\author{Gabriel Soares \\ Henrique da Silva}
\date{\today}
\pgfplotsset{width = 10cm, compat = 1.9}
 
 
\begin{document}
\maketitle
\pagenumbering{gobble}
\newpage
%pagenumbering{roman}
\tableofcontents
\newpage



\section{Introdução}


\subparagraph*{Neste relatório, vamos discutir o comportamento de um multímetro, como ele induz erros para certas bandas de frequência e o porquê.}

%\subparagraph*{Todos arquivos utilizados para criar este relatorio, e o relatorio em si estão em:  \url{https://github.com/Shapis/ufpe_ee/tree/main/5th semester/lab circuitos}}




\subsection{Análise preliminar}

\subparagraph*{Analisaremos a maneira que o multímetro mede tensões.}

\subparagraph*{Especificamente, mediremos uma tensão conhecida de $5 V_{pp}$ e analisaremos os erros absoluto e percentual da medição em função da frequência provinda do gerador de sinais.}

\subparagraph*{Faremos isso para dois tipos de onda de entrada, senoidal e serra.}

\section{Resultados esperados}

\subsection{Onda senoidal}

\subparagraph*{Para a onda senoidal, esperamos que o erro seja mais alto para frequências baixas e altas.}

\subparagraph*{Isso ocorre porque o multímetro tem uma banda de confiança. Quando nos afastamos dela, perdemos a certeza nas medidas.}

\subsection{Onda dente de serra}

\subparagraph*{Neste caso temos que lembrar que podemos decompor a onda em senoidais por série de Fourier. E como vimos anteriormente, as decomposições que tiverem frequencia alta ou baixa serão problemáticas.}

\subparagraph*{Mas esperamos que os erros sejam mais distribuídos ao longo da banda inteira que testarmos.}

\section{Medições no laboratório}

\subparagraph*{Vamos utilizar o osciloscópio para medir uma tensão de saída conhecida do osciloscópio, esta de $5 V_{pp}$, e registraremos os erros absoluto e relativo entre nossas medidas e a esperada de $5 V_{pp}$.}

\subsection{Tabela de medições}

\begin{center}
    \begin{tabular}{ |c|c|c| }
        \hline
        Freq (Hz) & Erro absoluto (V) & Erro percentual \\
        10        & 0.1818            & 3.58\%          \\
        15        & 0.1267            & 2.50\%          \\
        60        & 0.0930            & 1.83\%          \\
        120       & 0.0916            & 1.80\%          \\
        300       & 0.0913            & 1.80\%          \\
        600       & 0.0924            & 1.82\%          \\
        1000      & 0.0941            & 1.85\%          \\
        10000     & 0.1139            & 2.25\%          \\
        20000     & 0.1388            & 2.74\%          \\
        30000     & 0.1219            & 2.40\%          \\
        40000     & 0.1114            & 2.19\%          \\
        50000     & 0.0766            & 1.51\%          \\
        60000     & 0.0373            & 0.73\%          \\
        70000     & 0.0050            & 0.10\%          \\
        80000     & 0.0162            & 0.32\%          \\
        90000     & 0.0247            & 0.49\%          \\
        100000    & 0.0213            & 0.42\%          \\
        110000    & 0.0068            & 0.13\%          \\
        120000    & 0.0172            & 0.34\%          \\
        130000    & 0.0497            & 0.98\%          \\
        140000    & 0.0893            & 1.76\%          \\
        150000    & 0.1349            & 2.66\%          \\
        160000    & 0.1858            & 3.66\%          \\
        170000    & 0.2401            & 4.73\%          \\
        180000    & 0.2995            & 5.90\%          \\
        190000    & 0.3606            & 7.10\%          \\
        200000    & 0.4253            & 8.38\%          \\
        250000    & 0.7718            & 15.21\%         \\
        300000    & 1.1342            & 22.35\%         \\
        330000    & 1.1342            & 26.11\%         \\
        \hline
    \end{tabular}
\end{center}

\subsection{Gráficos dos dados}

\subsubsection{Erro absoluto por frequência}

\begin{adjustbox}{scale=0.30}
    \includegraphics{Grafico1.png}
\end{adjustbox}

\subsubsection{Erro percentual por frequência}

\begin{adjustbox}{scale=0.30}
    \includegraphics{Grafico2.png}
\end{adjustbox}

\subsection{Análise da onda dente de serra}

\subparagraph*{Quando analisamos esse tipo de onda, vimos erros distribuídos ao longo de toda banda de testes.}

\subparagraph*{Isso ocorreu porque a função dente de serra pode ser decomposta em senoides, e estas míltiplas senoides obedecem ao erro de acordo com os gráficos acima na seção \emph{3.2}.}


\subparagraph*{Logo, as senoides decompostas de alta frequência e baixa nos deram um certo erro considerável, porém distribuído em toda banda de testes.}

\newpage
\section{Conclusões}

\subparagraph*{Vemos que o multímetro tem bastante confiança em uma faixa intermediaria, mas fora desta a confiança é reduzida significantemente.}

\subparagraph*{Precisamos levar em consideração também o formato da onda de entrada e sua decomposicao por série de Fourier.}

\subparagraph*{Outro ponto que não abordamos nesta pratica foi o aspecto da calibração do multimetro. Esta pode afetar tanto a banda de frequência de confiança quanto a confiança em todos pontos da banda.}


\end{document}
