
\documentclass[12pt,twoside, a4paper, twocolumn]{article}
\usepackage[utf8]{inputenc}
\usepackage[brazil]{babel}
\usepackage[margin = 0.5in]{geometry}
\usepackage{amsmath}
\usepackage{amsthm}
\usepackage{amssymb}
\usepackage{amsthm}
\usepackage{setspace}
\usepackage[americanvoltages,fulldiodes,siunitx]{circuitikz}
\usepackage{lipsum}
\usepackage{pgfplots}
\usepackage{ifthen}
\usepackage{adjustbox}
\usepackage[section]{placeins}
\usepackage{hyperref}
\usepackage{graphicx}
\usepackage{amsmath}
\usepackage{amsthm}
\usepackage{amssymb}
\usepackage{amsthm}
\usepackage{setspace}
\usepackage[americanvoltages,fulldiodes,siunitx]{circuitikz}
\usepackage{lipsum}
\usepackage{pgfplots}
\usepackage{ifthen}
\usepackage{adjustbox}
\usepackage[section]{placeins}
\usepackage{hyperref}
\usepackage{graphicx}
\usepackage{adjustbox}
\pgfplotsset{compat=newest}
\graphicspath{ {./images/} }
%  #1 color - optional #2 x_0 #3 y_0 #4 x_f #5 y_f #6 name - optional  #7 true if adding lines to axis
\newcommand{\drawvector} [9] [color=cyan] {
  \draw[line width=1.5pt,#1,-stealth](axis cs: #2, #3)--(axis cs: #4, #5) node[anchor=south west]{$#6$};
 \ifthenelse{\equal{#7}{true}}{
  \draw[line width=1pt,#1, dashed](axis cs: #4, #5)--(axis cs: #4, 0) node[anchor= north west]{$#8$};
  \draw[line width=1pt,#1, dashed](axis cs: #4, #5)--(axis cs: 0, #5) node[anchor=south east]{$#9$};
  }
  {}
}
\newcommand\deriv[2]{\frac{\mathrm d #1}{\mathrm d #2}}
\title{Quarto Relatório de Medidas Eletromagnéticas}
\author{Gabriel Soares \\ Henrique da Silva}
\date{\today}
\pgfplotsset{width = 10cm, compat = 1.9}
\begin{document}
\maketitle
\pagenumbering{gobble}
\newpage
%pagenumbering{roman}
\tableofcontents
\newpage






\section{Introdução}




\subparagraph*{Neste relatório, vamos tentar obter uma indutância de um indutor a partir de medições de capacitância e frequência conhecidas.}


\subparagraph*{Todos arquivos utilizados para criar este relatório, é o relatorio em si estão em:  \url{https://github.com/Shapis/ufpe_ee/tree/main/5th semester/Electromagnetic Measurements/Relatorios}}








\section{Análise preliminar}


\subparagraph*{Para isso construímos um circuito $RLC$ e mediremos as diferenças de fase entre as entradas e as saídas.}


\subparagraph*{Utilizaremos da seguinte relação para medir a indutância:}


\begin{equation}
    f = \frac{1}{2 \pi \sqrt[]{LC}}
\end{equation}


\subparagraph*{Buscaremos valores de $C$ e $R$ que nos deem uma diferença de fase entre $1ms$ e $1\mu s$ entre o sinal de entrada e o de saída.}


\subparagraph*{Com isto utilizaremos da relação que o período é o inverso da frequência e obteremos a frequência a ser utilizada.}


\begin{equation}
    f = \frac{1}{T}
\end{equation}


\section{Resultados esperados}


\subparagraph*{Como nós vamos utilizar capacitores e resistores de magnitude arbitrária, há algumas coisas que vamos esperar que sejam verdade.}


\subparagraph*{A frequência não será alterada se dividirmos o valor de $C$ e de $L$ por uma mesma constante.}


\subparagraph*{O ganho não será alterado se multiplicarmos o valor de $C$ e de $R$ por uma mesma constante.}
\subparagraph*{}
\begin{equation}
    \begin{aligned}
        R' & = k_m R             \\
        L' & = \frac{k_m}{k_f} L \\
        C' & = \frac{C}{k_m k_f}
    \end{aligned}
\end{equation}


\subparagraph*{Ou seja, o que deveríamos ter feito, é fixado um $R$ e alterado o $C$ até que o período desejado fosse obtido.}


\subparagraph*{Porém. Não sabíamos desta relação no dia da aula. Só aprendemos na aula de Circuitos II seguinte. Então, o que de fato fizemos, foi alterar o $R$ e $C$ arbitrariamente até obter uma frequência que nos desse resultados legíveis no osciloscópio.}


\section{Medições no laboratório}


\subparagraph*{Vamos utilizar o osciloscópio para gerar uma onda quadrada pulsada que passará por um circuito $RLC$. Mediremos a tensão no indutor para fazermos a análise.}


\subparagraph*{Faremos isso cinco vezes para cada indutor para obter a média e desvio padrão das medidas.}


\subparagraph*{Com estes em mãos, determinaremos a indutância do nosso indutor.}


\subsection{Componentes utilizados}


\begin{equation}
    \begin{aligned}
        C & = 4.88nF          \\
        R & = 138.8 \varOmega \\
    \end{aligned}
\end{equation}


\subsection{Medições dos circuitos}

\subsubsection{Circuito 1}


\begin{center}
    \begin{tabular}{ |c|c|c|}
        \hline
                      & $f (Hertz)$ & $ L (Henry) $         \\
        Medida $1$    & $490000$    & $21.6186 *  10^{-6} $ \\
        Medida $2$    & $490000$    & $21.6186 *  10^{-6} $ \\
        Medida $3$    & $490000$    & $21.6186 *  10^{-6} $ \\
        Medida $4$    & $480000$    & $22.5288 *  10^{-6} $ \\
        Medida $5$    & $490000$    & $21.6186 *  10^{-6} $ \\
        Media         & $488000$    & $21.8006 *  10^{-6} $ \\
        Desvio Padrão & $4472.136$  & $4.07 * 10^{-7} $     \\
        \hline
    \end{tabular}
\end{center}


\subsubsection{Circuito 2}


\begin{center}
    \begin{tabular}{ |c|c|c|}
        \hline
                      & $f (Hertz)$ & $ L (Henry) $ \\
        Medida $1$    & $$          & $$            \\
        Medida $2$    & $$          & $$            \\
        Medida $3$    & $$          & $$            \\
        Medida $4$    & $$          & $$            \\
        Medida $5$    & $$          & $$            \\
        Media         & $$          & $$            \\
        Desvio Padrão & $$          & $$            \\
        \hline
    \end{tabular}
\end{center}




\subparagraph*{Esquecemos de fazer as medições do segundo circuito. Mas a ideia e procedimento são os mesmos.}


\section{Conclusões}


\subparagraph*{Conseguimos determinar a indutância }








\end{document}