
\documentclass[12pt,twoside, a4paper, twocolumn]{article}
\usepackage[utf8]{inputenc}
\usepackage[brazil]{babel}
\usepackage[margin = 0.5in]{geometry}
\usepackage{amsmath}
\usepackage{amsthm}
\usepackage{amssymb}
\usepackage{amsthm}
\usepackage{setspace}
\usepackage[americanvoltages,fulldiodes,siunitx]{circuitikz}
\usepackage{lipsum}
\usepackage{pgfplots}
\usepackage{ifthen}
\usepackage{adjustbox}
\usepackage[section]{placeins}
\usepackage{hyperref}
\usepackage{graphicx}
\usepackage{amsmath}
\usepackage{amsthm}
\usepackage{amssymb}
\usepackage{amsthm}
\usepackage{setspace}
\usepackage[americanvoltages,fulldiodes,siunitx]{circuitikz}
\usepackage{lipsum}
\usepackage{pgfplots}
\usepackage{ifthen}
\usepackage{adjustbox}
\usepackage[section]{placeins}
\usepackage{hyperref}
\usepackage{graphicx}
\usepackage{adjustbox}
\pgfplotsset{compat=newest}
\graphicspath{ {./images/} }
%  #1 color - optional #2 x_0 #3 y_0 #4 x_f #5 y_f #6 name - optional  #7 true if adding lines to axis
\newcommand{\drawvector} [9] [color=cyan] {
  \draw[line width=1.5pt,#1,-stealth](axis cs: #2, #3)--(axis cs: #4, #5) node[anchor=south west]{$#6$};
 \ifthenelse{\equal{#7}{true}}{
  \draw[line width=1pt,#1, dashed](axis cs: #4, #5)--(axis cs: #4, 0) node[anchor= north west]{$#8$};
  \draw[line width=1pt,#1, dashed](axis cs: #4, #5)--(axis cs: 0, #5) node[anchor=south east]{$#9$};
  }
  {}
}
\newcommand\deriv[2]{\frac{\mathrm d #1}{\mathrm d #2}}
\title{Quarto Relatório de Medidas Eletromagnéticas}
\author{Gabriel Soares \\ Henrique da Silva}
\date{\today}
\pgfplotsset{width = 10cm, compat = 1.9}
\begin{document}
\maketitle
\pagenumbering{gobble}
\newpage
%pagenumbering{roman}
\tableofcontents
\newpage






\section{Introdução}




\subparagraph*{Neste relatório, uma indutância será obtida em um filtro RLC passa-faixa a partir de medições de capacitância e frequência conhecidas.}


\subparagraph*{Todos arquivos utilizados para criar este relatório, é o relatorio em si estão em:  \url{https://github.com/Shapis/ufpe_ee/tree/main/5th semester/Electromagnetic Measurements/Relatorios}}








\section{Análise preliminar}


\subparagraph*{Constrói-se o circuito $RLC$ e observa-se os sinais das entradas e das saídas no osciloscópio. Em certa faixa de frequências, o sinal da saída é alto. Fora dela, esse mesmo sinal é baixo, considerando o sinal da entrada sempre constante.}


\subparagraph*{Utiliza-se a seguinte relação para medir a indutância:}


\begin{equation}
    f = \frac{1}{2 \pi \sqrt[]{LC}}
\end{equation}


\subparagraph*{Buscam-se valores de $C$ e $R$ que fornecem um alto sinal de saída para um período entre $1ms$ e $1\mu s$, enquanto para os demais períodos o sinal é baixo.}


\subparagraph*{Com isso, ao utilizar a relação que diz que o período é o inverso da frequência, obtém-se a frequência central.}


\begin{equation}
    f = \frac{1}{T}
\end{equation}


\section{Resultados esperados}


\subparagraph*{Como são utilizados capacitores e resistores de magnitude arbitrária, há algumas coisas que esperam-se que sejam verdade.}


\subparagraph*{A frequência não é alterada ao dividir os valores de $C$ e de $L$ por uma mesma constante.}


\subparagraph*{O ganho não é alterado ao multiplicar o valores de $C$ e de $R$ por uma mesma constante.}
\subparagraph*{}
\begin{equation}
    \begin{aligned}
        R' & = k_m R             \\
        L' & = \frac{k_m}{k_f} L \\
        C' & = \frac{C}{k_m k_f}
    \end{aligned}
\end{equation}


\subparagraph*{Porém, o que deveria ter sido feito é fixar uma resistência $R$ e alterar a capacitância $C$ até que o período desejado fosse obtido.}


\subparagraph*{Os autores deste relatório não sabiam dessa relação no dia da aula. Só a aprenderam na aula de Circuitos II no dia seguinte. O que de fato foi feito foi alterar o $R$ e $C$ arbitrariamente até obter uma frequência que desse resultados legíveis no osciloscópio.}


\section{Medições no laboratório}


\subparagraph*{Utilizou-se o osciloscópio para gerar uma onda quadrada pulsada que passou pelo circuito $RLC$. Mediu-se a tensão no indutor para fazer a análise.}


\subparagraph*{Fez-se isso cinco vezes para o indutor para obter a média e desvio padrão das medidas.}


\subparagraph*{Com estes em mãos, determinou-se a indutância do componente. A mesma lógica será futuramente aplicada para um outro circuito de mesma configuração, mas com outro valor de indutância. Chama-se esse circuito de 2, enquanto o já trabalhado de circuito 1. As medições do circuito 2 não foram feitas devido a esquecimento. Pensou-se que deveria-se escolher um dos indutores fornecidos em sala de aula, não trabalhar com os dois.}


\subsection{Componentes utilizados}


\begin{equation}
    \begin{aligned}
        C & = 4.88nF          \\
        R & = 138.8 \varOmega \\
    \end{aligned}
\end{equation}


\subsection{Medições dos circuitos}

\subsubsection{Circuito 1}


\begin{center}
    \begin{tabular}{ |c|c|c|}
        \hline
                      & $f (Hertz)$ & $ L (Henry) $         \\
        Medida $1$    & $490000$    & $21.6186 *  10^{-6} $ \\
        Medida $2$    & $490000$    & $21.6186 *  10^{-6} $ \\
        Medida $3$    & $490000$    & $21.6186 *  10^{-6} $ \\
        Medida $4$    & $480000$    & $22.5288 *  10^{-6} $ \\
        Medida $5$    & $490000$    & $21.6186 *  10^{-6} $ \\
        Média         & $488000$    & $21.8006 *  10^{-6} $ \\
        Desvio Padrão & $4472.136$  & $4.07 * 10^{-7} $     \\
        \hline
    \end{tabular}
\end{center}


\subsubsection{Circuito 2}


\begin{center}
    \begin{tabular}{ |c|c|c|}
        \hline
                      & $f (Hertz)$ & $ L (Henry) $ \\
        Medida $1$    & $$          & $$            \\
        Medida $2$    & $$          & $$            \\
        Medida $3$    & $$          & $$            \\
        Medida $4$    & $$          & $$            \\
        Medida $5$    & $$          & $$            \\
        Média         & $$          & $$            \\
        Desvio Padrão & $$          & $$            \\
        \hline
    \end{tabular}
\end{center}

\section{Conclusões}


\subparagraph*{A indutância foi determinada com satisfatória precisão. O erro relativo percentual entre o valor encontrado e o valor nominal é de aproximadamente 0,91\%.}

\subparagraph*{A tabela referente ao circuito 2 será no futuro preenchida.}

\end{document}
