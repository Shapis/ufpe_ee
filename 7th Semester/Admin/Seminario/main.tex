
%----------------------------------------------------------------------------------------
%	PACKAGES AND THEMES
%----------------------------------------------------------------------------------------
\documentclass[aspectratio=169,xcolor=dvipsnames]{beamer}
\usetheme{SimplePlusAIC}

\usepackage[brazil]{babel}
\usepackage{hyperref}
\usepackage{graphicx} % Allows including images
\usepackage{booktabs} % Allows the use of \toprule, \midrule and  \bottomrule in tables
\usepackage{svg} %allows using svg figures
\usepackage{tikz}
\usepackage{makecell}
\newcommand*{\defeq}{\stackrel{\text{def}}{=}}

%Select the Epilogue font (requires luaLatex or XeLaTex compilers)
\usepackage{fontspec}
\setsansfont{Epilogue}[
    Path=./epilogueFont/,
    Scale=0.9,
    Extension = .ttf,
    UprightFont=*-Regular,
    BoldFont=*-Bold,
    ItalicFont=*-Italic,
    BoldItalicFont=*-BoldItalic
    ]

%----------------------------------------------------------------------------------------
%	TITLE PAGE
%----------------------------------------------------------------------------------------

\title[short title]{Liderança Adminstrativa} % The short title appears at the bottom of every slide, the full title is only on the title page
\subtitle{Desafios e Estratégias da Liderança Moderna}

\author[Surname]{Bruno França // Gabriela Leite \linebreak 
Henrique Silva // Letícia Pandorf \linebreak 
Letícia Roxo // Rodrigo Bezerra \linebreak }
\institute[UFPE]{Centro de Tecnologia e Geociências \newline Curso de Engenharia Eletrônica\newline Universidade Federal de Pernambuco}
% Your institution as it will appear on the bottom of every slide, maybe shorthand to save space


\date{23/02/2023} % Date, can be changed to a custom date
%----------------------------------------------------------------------------------------
%	PRESENTATION SLIDES
%----------------------------------------------------------------------------------------

\begin{document}

\begin{frame}[plain]
	% Print the title page as the first slide
	\titlepage
\end{frame}

\begin{frame}{Sumário}
	% Throughout your presentation, if you choose to use \section{} and \subsection{} commands, these will automatically be printed on this slide as an overview of your presentation
	\tableofcontents
\end{frame}

%------------------------------------------------
\section{Definição de Liderança em Administração}
%------------------------------------------------

\begin{frame}{Definição de Liderança em Administração}
	\begin{itemize}
		\item Perspectivas Humanistas:
		      \begin{itemize}
			      \item Como dinâmica de interação interpessoal: Influência
			            \begin{itemize}
				            \item Controle;
				            \item Poder;
				            \item Autoridade;
			            \end{itemize}
			      \item Como processo de redução de incerteza
			            \begin{itemize}
				            \item Tomada de decisão;
			            \end{itemize}
			      \item Como uma relação funcional entre o líder e subordinados
			            \begin{itemize}
				            \item Ex: Líder comunitário de uma área urbana carente;
			            \end{itemize}
			      \item Como processo em função do líder, subordinados e variáveis da situação
			            \begin{itemize}
				            \item $L = f(l,v,s)$;
				            \item Abordagem Situacional;
			            \end{itemize}
		      \end{itemize}
	\end{itemize}
\end{frame}

%------------------------------------------------
\section{Teorias de Liderança}
%------------------------------------------------

\begin{frame}{Teorias de Liderança}
	\begin{itemize}
		\item Teoria dos traços de Personalidades
		      \begin{itemize}
			      \item Traço físico: aparência física, estatura e peso;
			      \item Traço Intelectual: entusiasmo, autoconfiança e adaptabilidade;
			      \item Traço Social: cooperação, habilidade interpessoal e administrativa;
			      \item Traço Relacionado com a tarefa: persistência e iniciativa;
		      \end{itemize}
		\item Teoria sobre estilo de Liderança
		      \begin{itemize}
			      \item Foco nas ações do Líder;
			      \item Liderança Autocrática;
			      \item Liderança Liberal;
			      \item Liderança Democrática;
		      \end{itemize}
		\item Teoria sobre Liderança Situacional
	\end{itemize}
\end{frame}

%------------------------------------------------
\section{Atributos do Líder Eficaz}
%------------------------------------------------

\begin{frame}{Atributos do Líder Eficaz}
	\begin{itemize}
		\item Habilidades de comunicação interpessoal
		\item Tomada de decisões baseada em dados e análises
		\item Empatia e inteligência emocional como atributos essenciais

	\end{itemize}
\end{frame}

%------------------------------------------------
\section{Desafios da Liderança}
%------------------------------------------------

\begin{frame}{Desafios da Liderança}
	\begin{itemize}
		\item Adaptabilidade a mudanças organizacionais
		\item Gestão de conflitos dentro da equipe
		\item Lidar com a pressão e o estresse da liderança

	\end{itemize}
\end{frame}

%------------------------------------------------
\section{Proposições inferidas da abordagem situacional}
%------------------------------------------------

\begin{frame}{Proposições inferidas da abordagem situacional}
	\begin{itemize}
		\item Tarefas rotineiras e repetitivas ---> total grau de autoridade
		\item Liderança diferente para cada membro da equipe
		\item Liderança diferente para o mesmo membro da equipe
	\end{itemize}
\end{frame}

%------------------------------------------------
\section{Habilidades de Liderança}
%------------------------------------------------

\begin{frame}{Habilidades de Liderança}
	\begin{itemize}
		\item Programas de treinamento e desenvolvimento de liderança
		\item Mentoria e coaching para líderes em ascensão
		\item A importância do feedback na melhoria contínua

	\end{itemize}
\end{frame}

%------------------------------------------------
\section{Ética na Liderança}
%------------------------------------------------

\begin{frame}{Ética na Liderança}
	\begin{itemize}
		\item Tomada de decisões éticas em situações desafiadoras
		\item Construção de uma cultura organizacional ética
		\item Responsabilidade social corporativa e papel do líder

	\end{itemize}
\end{frame}

%------------------------------------------------
\section{Liderança e Motivação}
%------------------------------------------------

\begin{frame}{Liderança e Motivação}
	\begin{itemize}
		\item Teorias de motivação, como a hierarquia de necessidades de Maslow
		\item Reconhecimento e recompensas como ferramentas motivacionais
		\item Construção de um ambiente de trabalho positivo

	\end{itemize}
\end{frame}

%------------------------------------------------
\section{Liderança e Tomada de Decisão}
%------------------------------------------------

\begin{frame}{Liderança e Tomada de Decisão}
	\begin{itemize}
		\item Processo de tomada de decisão em etapas
		\item Consideração de riscos e benefícios nas decisões
		\item Envolvimento da equipe na tomada de decisões

	\end{itemize}
\end{frame}

%------------------------------------------------
\section{Liderança Transformacional}
%------------------------------------------------

\begin{frame}{Liderança Transformacional}
	\begin{itemize}
		\item Inspirar a inovação e a criatividade
		\item Compartilhamento de visão e valores com a equipe
		\item Desenvolvimento de uma cultura organizacional voltada para o futuro

	\end{itemize}
\end{frame}

%------------------------------------------------
\section{Liderança Feminina na Administração}
%------------------------------------------------

\begin{frame}{Liderança Feminina na Administração}
	\begin{itemize}
		\item Desafios específicos enfrentados por mulheres em cargos de liderança
		\item Vantagens da diversidade de gênero na liderança
		\item Estratégias para promover a igualdade de oportunidades

	\end{itemize}
\end{frame}

%------------------------------------------------
\section{Liderança Digital}
%------------------------------------------------

\begin{frame}{Liderança Digital}
	\begin{itemize}
		\item Adaptação a ferramentas de comunicação digital
		\item Desenvolvimento de habilidades tecnológicas para líderes
		\item Gestão eficaz de equipes remotas e trabalho à distância

	\end{itemize}
\end{frame}

% \begin{frame}{Bibliografia}
% 	\begin{itemize}
% 		\item Livro 1
% 		\item Livro 2
% 		\item Livro 3

% 	\end{itemize}
% \end{frame}

\end{document}

