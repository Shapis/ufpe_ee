\section{Análise dos resultados}

Na análise, foca-se nas frequências de corte para compreender a importância de cada capacitor em sua determinação. Isso permite não apenas quantificar a influência de cada componente, mas também compreender a complexa interação entre eles.

\subsection{Frequências de corte}

Para o circuito com seus componentes inciais temos a seguinte frequência de corte:

\subsubsection{Circuito original}

\begin{equation}
    F_L = 880 Hz
\end{equation}

\subsubsection{Circuito com $C_1$ 10 vezes menor}

\begin{equation}
    F_L = 1.1 kHz
\end{equation}

\subsubsection{Circuito com $C_1$ 10 vezes maior}

\begin{equation}
    F_L = 880 Hz
\end{equation}

\subsubsection{Circuito com $C_2$ 10 vezes menor}

\begin{equation}
    F_L = 8.3 kHz
\end{equation}

\subsubsection{Circuito com $C_2$ 10 vezes maior}

\begin{equation}
    F_L = 480 Hz
\end{equation}

\subsubsection{Circuito com $C_3$ 10 vezes menor}

\begin{equation}
    F_L = 999 Hz
\end{equation}

\subsubsection{Circuito com $C_3$ 10 vezes maior}

\begin{equation}
    F_L = 870 Hz
\end{equation}

\subsection{Capacitor dominante}

Ao dispor dessas frequências, torna-se evidente que o capacitor $C_2$ desempenha um papel proeminente e determinante nas características de corte do circuito. Sua influência preponderante sugere que as propriedades específicas desse componente exercem um impacto significativo na resposta do circuito em diferentes faixas de frequência.