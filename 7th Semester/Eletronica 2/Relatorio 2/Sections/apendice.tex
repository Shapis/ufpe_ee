\section{Apêndice}
\label{ap:apendice}

% Abaixo se encontra o código utilizado para a análise simbólica e numérica do circuito.

Abaixo se encontra o código utilizado para a análise numérica do circuito.

\begin{python}
    import matplotlib.pyplot as plt
    import sympy as smp
    from sympy import *
    import numpy as np

    # j = smp.symbols('j', imaginary=True)

    # Vo, Vi, Vm, V_C0, V_D0, R_m, C, t, R, Vc = smp.symbols(
    #     'V_o V_i V_m V_C0 V_D0 R_m C t R V_c', real=True)


    def updateEqs(Vm, V_C0, V_D0, R_m, C, t, R):
    Vo1 = ((V_C0 + Vm + V_D0) * np.exp(-t/(R_m * C))) - V_D0
    Vc1 = (V_C0 + Vm + V_D0) * np.exp(-t/(R_m * C)) - Vm - V_D0
    Vo2 = (V_C0 + Vm) * np.exp(-t/(R * C))
    Vc2 = (V_C0 + Vm) * np.exp(-t/(R * C)) - Vm
    return Vo1, Vc1, Vo2, Vc2


    # Vo1, Vc1, Vo2, Vc2 = updateEqs(Vm, V_C0, V_D0, R_m, C, t, R)


    # estado 1: Vm + Vc0 < - Vd0
    # estado 2: Vm + Vc0 > - Vd0

    # Exemplo 2

    R = float("inf")
    # R = 4.7E3
    Vm = 5

    # Dados
    V_D0 = 0.5
    C = 100E-9
    R_m = 330
    V_C0 = 0

    Ts = 500E-6
    step = 1E-7
    temp = 0

    intervalos_tempo = np.arange(0, 10*Ts + step, step)
    valores_de_Vo = []
    valores_de_Vc = []

    for t in intervalos_tempo:
    temp += step
    if temp >= Ts/2:
    Vm = -Vm
    temp = 0

    Vo1, Vc1, Vo2, Vc2 = updateEqs(Vm, V_C0, V_D0, R_m, C, step, R)

    if (Vm + V_C0) < -V_D0:  # estado 1
    Vo = Vo1
    Vc = Vc1
    else:                   # estado 2
    Vo = Vo2
    Vc = Vc2

    valores_de_Vo.append(Vo)
    valores_de_Vc.append(Vc)
    V_C0 = Vc


    # smp.pprint(Vo)

    # Plotando os graficos


    plotH1 = valores_de_Vo
    plotH2 = valores_de_Vc


    fig, ax = plt.subplots()

    ax.plot(intervalos_tempo, plotH1, color='blue', label='Exemplo 1')
    ax.plot(intervalos_tempo, plotH2, color='orange', label='Exemplo 2')
    ax.legend(['Vo', 'Vc'])
    plt.xlabel('Tempo (s)')
    plt.ylabel('Tensao (V)')
    plt.title('Grazfico de $V_o(t)$ no intervalo de 0 a 10*Ts')
    plt.show()


    print(min(valores_de_Vo))
    print(max(valores_de_Vo))

\end{python}