\section{Análise dos resultados}

Na análise, busca-se a comparação entre os valores simulados e práticos, identificando convergências e discrepâncias. Isso proporciona insights valiosos para ajustar modelos teóricos e aprimorar técnicas experimentais, contribuindo para o desenvolvimento mais preciso do estudo.

\subsection{Ganhos e Frequências de Corte}

\begin{table}[h]
    \centering
    \begin{tabular}{|c|c|c|}
        \hline
        \textbf{-} & \textbf{Teórico} & \textbf{Real} \\
        \hline
        $F_L$      & $75.7Hz$         & $71Hz$        \\
        \hline
        $F_H$      & $95.26MHz$       & $10.3 MHz$    \\
        \hline
        $A_f$      & $4.06$           & $4.415$       \\
        \hline
        $A_{f-50}$ & $3.64$           & $3.72$        \\
        \hline
        $A_{f+50}$ & $4.16$           & $4.75$        \\
        \hline
    \end{tabular}
\end{table}

Os valores obtidos foram próximos dos esperados, com exceção da frequência de corte alta, devido à precisão dos instrumentos e ao efeito de altas frequências nos componentes.

Interessante notar que uma variação brusca de $50\%$ em $R_C$ e $R_L$ apenas acarreta em uma variação de dez por cento para aumentos e vinte por cento para reduções nos valores dos resistores.

\subsection{Parametros do Circuito}

\begin{table}[h]
    \centering
    \begin{tabular}{|c|c|c|}
        \hline
        \textbf{-} & \textbf{Teórico}   & \textbf{Real}     \\
        \hline
        $A$        & $6.435$            & $7.41$            \\
        \hline
        $\beta$    & $0.09$             & $0.09$            \\
        \hline
        $R_i$      & $1172.7\varOmega$  & $1108.13$         \\
        \hline
        $R_o$      & $100\varOmega$     & $98.9 \varOmega$  \\
        \hline
        $R_{if}$   & $1858.75\varOmega$ & $1860$            \\
        \hline
        $R_{of}$   & $63\varOmega$      & $58.92 \varOmega$ \\
        \hline
    \end{tabular}
\end{table}

Os valores obtidos foram próximos. As discrepâncias se devem às diferenças nos valores dos componentes utilizados na simulação e na montagem do circuito real.