\section{Conclusões}

Conclui-se que, ao analisar os resultados obtidos, o primeiro exemplo, em que se utilizou um resistor $R$, apresentou um comportamento consistente com o de um grampeador de tensão. Isso foi evidenciado pela limitação da amplitude da tensão de saída mínima a um valor pré-determinado, independentemente das variações na tensão de entrada. Por outro lado, no segundo exemplo, em que se simulou uma resistência infinita, o circuito não se comportou como um grampeador de tensão, uma vez que não houve a limitação da amplitude mínima da tensão de saída. Essa análise reforça a importância do componente resistivo na operação de um grampeador de tensão e destaca a influência desse elemento na funcionalidade do circuito.

Também é importante mencionar que o experimento foi realizado com êxito, graças às ferramentas de análise disponíveis. As medições em laboratório e as simulações numéricas forneceram resultados confiáveis e contribuíram significativamente para a compreensão do comportamento do circuito.