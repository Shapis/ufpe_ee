\section{Introdução}

Os transistores bipolares de junção (TBJ) são componentes essenciais na eletrônica moderna. Eles desempenham um papel crucial na amplificação de sinais elétricos e são fundamentais em uma variedade de circuitos eletrônicos, desde dispositivos de áudio até sistemas de comunicação e eletrônica de potência. Uma característica distintiva dos TBJ é sua capacidade de operar em diferentes classes de amplificação, sendo as classes A, B e AB as mais proeminentes. Cada uma dessas classes de operação apresenta características específicas que atendem a diferentes requisitos de aplicação, tornando os TBJ um componente versátil e amplamente empregado na eletrônica moderna.

Todos arquivos utilizados para criar este relatório, e o relatório em si estão em:  \url{https://github.com/Shapis/ufpe_ee/tree/main/6th semester/Eletronica 1/}