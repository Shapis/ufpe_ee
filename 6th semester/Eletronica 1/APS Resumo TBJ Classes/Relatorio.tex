\documentclass[12pt,twoside, a4paper]{article}
\usepackage[utf8]{inputenc}
\usepackage[brazil]{babel}
\usepackage[margin = 0.5in]{geometry}
\usepackage{amsmath}
\usepackage{amsthm}
\usepackage{amssymb}
\usepackage{amsthm}
\usepackage{setspace}
\usepackage[americanvoltages,fulldiodes,siunitx]{circuitikz}
\usepackage{lipsum}
\usepackage{pgfplots}
\usepackage{ifthen}
\usepackage{adjustbox}
\usepackage[section]{placeins}
\usepackage{hyperref}
\usepackage{graphicx}
\usepackage{amsmath}
\usepackage{amsthm}
\usepackage{amssymb}
\usepackage{amsthm}
\usepackage{setspace}
\usepackage[americanvoltages,fulldiodes,siunitx]{circuitikz}
\usepackage{lipsum}
\usepackage{pgfplots}
\usepackage{ifthen}
\usepackage{adjustbox}
\usepackage[section]{placeins}
\usepackage{hyperref}
\usepackage{graphicx}
\usepackage{adjustbox}
\usepackage{indentfirst}
\usepackage{float}
\usepackage{pythonhighlight}


\pgfplotsset{compat=newest}
\graphicspath{ {./images/} }
%  #1 color - optional #2 x_0 #3 y_0 #4 x_f #5 y_f #6 name - optional  #7 true if adding lines to axis
\newcommand{\drawvector} [9] [color=cyan] {
\draw[line width=1.5pt,#1,-stealth](axis cs: #2, #3)--(axis cs: #4, #5) node[anchor=south west]{$#6$};
\ifthenelse{\equal{#7}{true}}{
\draw[line width=1pt,#1, dashed](axis cs: #4, #5)--(axis cs: #4, 0) node[anchor= north west]{$#8$};
\draw[line width=1pt,#1, dashed](axis cs: #4, #5)--(axis cs: 0, #5) node[anchor=south east]{$#9$};
}
{}
}
\newcommand\deriv[2]{\frac{\mathrm d #1}{\mathrm d #2}}
\pgfplotsset{width = 10cm, compat = 1.9}

\begin{document}
\title{APS Sobre Classes de Circuitos Amplificadores -- Eletrônica 1}
\author{Henrique da Silva \\ henrique.pedro@ufpe.br}
\date{\today}

\maketitle
\pagenumbering{gobble}
\tableofcontents
\newpage



\input{Sections/intro}

\section{Classe A}

A classe A é conhecida por sua operação de amplificação linear precisa. Nessa classe, o transistor TBJ conduz continuamente durante todo o ciclo do sinal de entrada, ou seja, ele fica polarizado de modo que a corrente flua constantemente, independentemente do sinal de entrada. Isso resulta em uma resposta linear excepcional, tornando a classe A ideal para aplicações onde a distorção do sinal é inaceitável.

Em aplicações de áudio de alta qualidade, a classe A é frequentemente empregada, proporcionando uma reprodução precisa e natural do som. No entanto, a eficiência energética da classe A é relativamente baixa, pois o transistor está sempre em condução, o que gera uma quantidade significativa de calor. Portanto, a classe A é mais adequada para aplicações de baixa potência, onde a qualidade do sinal é prioritária em relação à eficiência energética.

\subsection{Características Principais}

A classe A é uma das categorias mais notáveis de operação para transistores bipolares de junção (TBJ) e é reconhecida por suas características distintivas que a tornam ideal para certas aplicações eletrônicas. Vamos explorar mais suas características principais.

\subsubsection{Operação Contínua}

Na classe A, os transistores TBJ são polarizados de maneira que a corrente flua constantemente, independentemente das flutuações no sinal de entrada. Isso significa que o transistor está sempre ligado e conduz durante todo o ciclo do sinal de entrada. Em outras palavras, ele não é desligado em nenhum momento, mesmo quando não há sinal de entrada aplicado. Esse comportamento é conhecido como operação contínua.

\subsubsection{Condução Durante o Ciclo Completo do Sinal de Entrada}

Uma das características mais marcantes da classe A é que o transistor TBJ conduz durante o ciclo completo do sinal de entrada. Isso resulta em uma amplificação completa e linear do sinal de entrada, pois a corrente flui continuamente por todo o transistor. Como resultado, a amplificação é precisa e não distorce o sinal de saída. O transistor permanece na região ativa, evitando regiões de corte ou saturação, onde a amplificação não seria linear.

\subsubsection{Baixa Eficiência de Energia}

Uma das desvantagens críticas da classe A é sua baixa eficiência energética. Devido ao fato de que o transistor TBJ está sempre conduzindo, ele consome energia constantemente, independentemente do sinal de entrada. Isso resulta na geração de uma quantidade considerável de calor, que deve ser dissipada por meio de sistemas de resfriamento. Como resultado, a eficiência energética geral do circuito é baixa, o que significa que uma parte significativa da energia é desperdiçada como calor.

Embora a classe A tenha uma eficiência energética limitada, é amplamente empregada em aplicações que valorizam a qualidade do sinal e a linearidade acima de tudo. Amplificadores de áudio de alta fidelidade e amplificadores de radiofrequência de baixa potência são exemplos típicos de dispositivos que fazem uso da classe A, onde a distorção do sinal é inaceitável e a eficiência energética é secundária. Portanto, embora a classe A possa não ser eficiente em termos de energia, seu desempenho na preservação da qualidade do sinal a torna uma escolha valiosa em várias aplicações eletrônicas de alta precisão.

\subsection{Uso comum}

A classe A é uma escolha amplamente adotada em muitas aplicações eletrônicas devido às suas características únicas. Vamos explorar em detalhes as aplicações mais comuns dessa classe de operação.

\subsubsection{Amplificadores de Áudio de Alta Qualidade}

Uma das aplicações mais notáveis da classe A é em amplificadores de áudio de alta qualidade. Nesse contexto, a classe A é altamente valorizada pela sua capacidade de fornecer uma amplificação precisa e linear do sinal de áudio. Isso significa que os amplificadores de classe A minimizam a distorção do sinal, reproduzindo fielmente o som original. Essa característica é crucial em sistemas de áudio de alta fidelidade, onde a qualidade do som é essencial. Músicos e audiófilos apreciam amplificadores de classe A devido à reprodução nítida e sem distorção de música e áudio.

\subsubsection{Amplificadores de RF de Baixa Potência}

Além do uso em áudio, a classe A também encontra aplicação em amplificadores de radiofrequência (RF) de baixa potência. Nessas aplicações, a classe A é escolhida quando é necessário amplificar sinais de RF com precisão e sem introduzir distorção significativa. Os amplificadores de classe A de baixa potência são comuns em sistemas de comunicação sem fio, receptores de rádio e transmissores de RF de baixa potência. Eles garantem que os sinais de RF sejam amplificados com a menor interferência possível.

\subsubsection{Polarização}

Embora a classe A ofereça vantagens notáveis em termos de qualidade de sinal, é importante mencionar que ela requer uma polarização extremamente precisa para operar de maneira eficaz e evitar distorções. A polarização cuidadosa envolve ajustar a corrente de base do transistor TBJ para garantir que ele permaneça na região ativa, evitando as regiões de corte e saturação. Qualquer desvio na polarização pode resultar em distorção do sinal, comprometendo a qualidade do áudio ou do sinal RF.

Para manter a polarização adequada, os amplificadores de classe A frequentemente incorporam circuitos de estabilização e monitoramento da corrente de base. Esses circuitos garantem que o transistor TBJ permaneça operando no ponto de polarização ideal, mesmo com variações na temperatura e nas características do dispositivo. Essa atenção aos detalhes é essencial para garantir o desempenho consistente e de alta qualidade dos amplificadores de classe A.

\subsection{Vantagens}

A classe A é uma categoria de operação de transistores bipolares de junção (TBJ) que se destaca por suas vantagens específicas, especialmente quando a qualidade do sinal é priorizada. Vamos aprofundar as principais vantagens dessa classe de operação.

\subsubsection{Baixa Distorção}

Uma das vantagens mais proeminentes da classe A é a baixa distorção do sinal. Isso ocorre porque o transistor TBJ na classe A conduz durante todo o ciclo do sinal de entrada, sem interrupções. Essa operação contínua elimina a distorção de crossover, que é comum em outras classes de operação, como a classe B. Portanto, os amplificadores de classe A são capazes de reproduzir com precisão o sinal de entrada, minimizando a distorção harmônica e intermodulação.

\subsubsection{Linearidade Excepcional}

A classe A é conhecida por sua linearidade excepcional. Isso significa que a relação entre a variação do sinal de entrada e a variação correspondente do sinal de saída é praticamente linear, dentro dos limites de operação do transistor. Essa linearidade é fundamental para a amplificação precisa de sinais, especialmente em aplicações de áudio de alta fidelidade, onde a reprodução fiel do som é essencial.

\subsubsection{Aplicações de Alta Fidelidade}

A classe A é amplamente empregada em amplificadores de áudio de alta fidelidade (Hi-Fi) e sistemas de alta qualidade de reprodução de som. Nestes sistemas, a fidelidade ao sinal original é crítica, e a classe A atende a essa necessidade com sua capacidade de amplificar os sinais sem introduzir distorção significativa. Amplificadores de classe A são apreciados por audiófilos e engenheiros de áudio devido à reprodução precisa dos detalhes sonoros e à capacidade de preservar a qualidade da música sem adicionar artefatos indesejados.

\subsubsection{Baixa Intermodulação}

Além da baixa distorção harmônica, a classe A também minimiza a distorção de intermodulação, que ocorre quando dois ou mais sinais diferentes são amplificados e suas frequências se misturam de maneira indesejada. Isso contribui ainda mais para a alta fidelidade da classe A, tornando-a adequada para sistemas de som de alta qualidade, onde a pureza do som é essencial.

\subsubsection{Qualidade de Som Superior}

A combinação de baixa distorção, linearidade excepcional e baixa intermodulação resulta em uma qualidade de som superior nos amplificadores de classe A. Essa qualidade de som é particularmente apreciada em estúdios de gravação, sistemas de cinema em casa e sistemas de som de alta fidelidade, onde a precisão na reprodução do som é crucial.

Em resumo, a classe A em transistores bipolares de junção (TBJ) oferece vantagens notáveis, incluindo baixa distorção, linearidade excepcional e adequação para aplicações de alta fidelidade. Essas características a tornam a escolha ideal quando a qualidade do sinal é a prioridade, mesmo que isso resulte em uma eficiência energética mais baixa e na geração de calor.

\subsection{Desvantagens}

Embora a classe A tenha suas vantagens notáveis em termos de qualidade de sinal, ela também apresenta algumas desvantagens importantes que a tornam inadequada para certas aplicações. Vamos explorar mais profundamente as principais desvantagens dessa classe de operação:

\subsubsection{Baixa Eficiência Energética}

Uma das desvantagens mais proeminentes da classe A é sua baixa eficiência energética. Isso ocorre porque, na classe A, o transistor TBJ está sempre conduzindo corrente, independentemente do sinal de entrada. Como resultado, uma quantidade significativa de energia é desperdiçada na forma de calor, mesmo quando não há sinal de saída. Isso torna os amplificadores de classe A ineficientes em termos de consumo de energia e requer sistemas de resfriamento eficazes para dissipar o calor gerado.

\subsubsection{Geração Significativa de Calor}

Devido à baixa eficiência energética, os amplificadores de classe A geram uma quantidade considerável de calor. Isso não apenas representa uma ineficiência em termos de consumo de energia, mas também pode ser problemático em ambientes onde o controle de temperatura é crítico. Em aplicações de áudio de alta potência, por exemplo, é necessário um resfriamento adequado para evitar o superaquecimento dos componentes eletrônicos.

\subsubsection{Amplificação de Alta Potência}

A classe A não é a escolha ideal para amplificação de alta potência devido à sua baixa eficiência energética. Amplificar grandes amplitudes de sinal requer mais corrente e, portanto, resulta em uma produção significativamente maior de calor. Além disso, amplificadores de classe A de alta potência exigiriam transistores TBJ de maior capacidade, o que aumentaria ainda mais o consumo de energia e a dissipação de calor.

\subsubsection{Complexidade da Polarização}

Para manter a operação de classe A, é necessário um circuito de polarização cuidadosamente projetado para garantir que o transistor TBJ permaneça na região ativa e não entre na região de corte ou saturação. Isso aumenta a complexidade do projeto e requer componentes adicionais, como resistores e fontes de tensão, para garantir que a polarização seja estável e precisa. A complexidade da polarização torna o projeto de amplificadores de classe A mais desafiador em comparação com outras classes de operação.

\subsubsection{Limitação de Potência}

Devido à baixa eficiência e à geração de calor, amplificadores de classe A são limitados em termos de potência de saída. Eles são mais adequados para aplicações de baixa a média potência, onde a qualidade do sinal é crítica, mas a eficiência energética é menos importante.


\section{Classe B}

A classe B é reconhecida por sua eficiência energética máxima, sendo uma escolha popular para amplificação de potência. Nessa classe, dois transistores TBJ operam em conjunto, onde cada um deles conduz apenas metade do ciclo do sinal de entrada. Isso resulta em uma eficiência significativamente maior em comparação com a classe A, uma vez que o transistor está desligado quando não está conduzindo corrente.

Apesar de sua alta eficiência energética, a classe B apresenta um desafio importante relacionado à distorção. Na transição entre os transistores, pode ocorrer uma distorção de sinal significativa, conhecida como distorção de crossover. Portanto, a classe B é mais adequada para aplicações onde a distorção não é crítica, como amplificadores de potência em sistemas de alto-falantes.

\subsection{Características Principais}

A classe B de operação de transistores bipolares de junção (TBJ) possui características distintivas que a tornam adequada para aplicações de alta potência, mas também apresenta desafios relacionados à distorção. Vamos explorar mais profundamente as características principais dessa classe de operação.

\subsubsection{Transistor Conduz Apenas Metade do Ciclo do Sinal}

A característica fundamental da classe B é que o transistor TBJ conduz apenas metade do ciclo do sinal de entrada. Em um amplificador de classe B, dois transistores são usados em conjunto: um transistor conduz durante a metade positiva do ciclo do sinal, enquanto o outro conduz durante a metade negativa. Isso significa que cada transistor está desligado durante metade do ciclo, resultando em uma operação eficiente em termos de consumo de energia. Essa característica é fundamental para alcançar alta eficiência energética.

\subsubsection{Elevada Eficiência Energética}

Uma das vantagens mais proeminentes da classe B é sua elevada eficiência energética. Como mencionado anteriormente, os transistores estão desligados durante metade do ciclo do sinal, o que significa que não há corrente de coletor fluindo quando não há sinal de entrada. Isso resulta em uma baixa dissipação de energia e, consequentemente, em uma eficiência energética máxima. A classe B é, portanto, uma escolha lógica quando a conservação de energia é crítica, especialmente em amplificadores de alta potência.

\subsubsection{Alto Potencial para Distorção na Região de Corte}

Apesar da alta eficiência energética, a classe B apresenta um desafio crítico relacionado à distorção. Na transição entre os transistores, quando um assume o controle do sinal e o outro é desligado, pode ocorrer uma distorção significativa. Essa distorção é conhecida como distorção de crossover e ocorre porque os transistores não estão perfeitamente emparelhados e há uma região de transição onde ambos os transistores operam em paralelo.

A distorção de crossover pode resultar em uma reprodução imprecisa do sinal de saída, introduzindo harmônicos indesejados. Para minimizar esse problema, é comum usar uma configuração de classe AB, que combina elementos das classes A e B, buscando um equilíbrio entre eficiência energética e linearidade.

\subsubsection{Aplicações de Alta Potência}

A classe B é frequentemente empregada em amplificadores de alta potência, onde a conservação de energia é essencial. Amplificadores de áudio de alta potência, sistemas de transmissão de RF de alta potência e amplificadores de potência para transdutores são exemplos comuns de aplicações que fazem uso da classe B. Nesses casos, a alta eficiência energética é priorizada em relação à distorção, e estratégias como a realimentação (feedback) são usadas para mitigar a distorção sempre que possível.

Em resumo, a classe B de operação em transistores bipolares de junção (TBJ) oferece alta eficiência energética, tornando-a ideal para aplicações de alta potência, mas também apresenta o desafio da distorção de crossover. A escolha da classe B ou de configurações relacionadas, como a classe AB, depende das necessidades específicas de aplicação, priorizando eficiência ou linearidade, conforme necessário.

\subsection{Uso Comum}

A classe B é uma classe de operação de transistores bipolares de junção (TBJ) que encontra amplas aplicações, principalmente em situações que exigem amplificação de alta potência e eficiência energética, mesmo que a distorção do sinal seja uma consideração secundária. Vamos explorar em detalhes os usos mais comuns da classe B.

\subsubsection{Amplificadores de Potência}

A aplicação mais proeminente da classe B é em amplificadores de potência. Quando a amplificação de sinais de alta potência é necessária, os amplificadores de classe B são frequentemente a escolha ideal devido à sua alta eficiência energética. Eles são usados em uma variedade de contextos, incluindo amplificadores de RF de alta potência, amplificadores de áudio de palco para música ao vivo, sistemas de som de grande porte e amplificadores para transdutores de alto-falantes em sistemas de PA (Público Alvo).

\subsubsection{Amplificação de Áudio}

Em certas aplicações de áudio, onde a distorção do sinal não é considerada crítica, a classe B também é usada. Isso ocorre porque, embora a classe B possa introduzir distorção na região de crossover, essa distorção pode ser aceitável em muitos contextos, como amplificação de som ambiente, intercomunicadores e sistemas de megafones. Nessas situações, a eficiência energética e a conservação de energia superam a necessidade de uma amplificação de áudio totalmente livre de distorção.

\subsubsection{Transmissores de RF de Alta Potência}

A classe B é comumente empregada em transmissores de radiofrequência (RF) de alta potência. Esses transmissores são usados em sistemas de comunicação, transmissão de televisão, rádio AM e FM, bem como em estações de rádio de grande porte. A alta eficiência energética da classe B é particularmente valiosa em transmissores de RF, pois ajuda a minimizar os custos operacionais e reduzir a dissipação de calor, o que é crítico em transmissões contínuas de alta potência.

\subsubsection{Conversores de Energia e Inversores}

Além dos setores de áudio e RF, a classe B é usada em conversores de energia e inversores em sistemas de alimentação elétrica. Nesses dispositivos, a eficiência energética é fundamental para a conversão eficaz de energia, e a classe B é uma escolha lógica para alcançar esse objetivo. Eles são usados em fontes de alimentação chaveadas, conversores DC-DC, inversores solares e sistemas de armazenamento de energia.

\subsubsection{Amplificadores de RF de Classe C}

Em algumas aplicações de RF, os amplificadores de classe B são usados em conjunto com amplificadores de classe C para amplificar sinais de alta frequência. Os amplificadores de classe C são altamente eficientes, mas não são lineares, e os amplificadores de classe B são usados para linearizar o sinal de saída quando necessário, como em sistemas de transmissão de alta potência.

Em resumo, a classe B de operação em transistores bipolares de junção (TBJ) é amplamente utilizada em amplificadores de potência, transmissores de RF de alta potência, conversores de energia e em situações de áudio onde a distorção do sinal não é uma preocupação crítica. Sua eficiência energética e capacidade de amplificar sinais de alta potência a tornam uma escolha valiosa em uma variedade de aplicações elétricas e eletrônicas.

\subsection{Vantagens}

A classe B de operação em transistores bipolares de junção (TBJ) apresenta várias vantagens significativas, tornando-a uma escolha ideal para aplicações de alta potência e eficiência energética. Vamos explorar mais profundamente as principais vantagens dessa classe de operação:

\subsubsection{Eficiência Energética Máxima}

Uma das características mais proeminentes da classe B é sua eficiência energética máxima. Isso ocorre devido ao fato de que, na classe B, cada transistor TBJ conduz apenas metade do ciclo do sinal de entrada. Em outras palavras, quando não há sinal de entrada ou o sinal está próximo de zero, ambos os transistores estão desligados. Isso resulta em uma baixa dissipação de energia quando não há demanda de amplificação, tornando a classe B altamente eficiente em termos de consumo de energia. Essa eficiência é especialmente importante em aplicações onde a conservação de energia é crítica, como em transmissões contínuas de alta potência.

\subsubsection{Aplicações de Alta Potência}

A classe B é particularmente adequada para aplicações de alta potência, onde a amplificação de sinais de grande amplitude é necessária. A capacidade de ligar e desligar cada transistor de forma alternada permite que a classe B amplifique sinais de alta potência com eficiência, minimizando a dissipação de calor. Essa característica é valiosa em amplificadores de potência, transmissores de radiofrequência de alta potência e sistemas de conversão de energia de alta capacidade.

\subsubsection{Dissipação de Calor}

Devido à sua operação eficiente, os amplificadores de classe B geram menos calor em comparação com amplificadores de classe A, onde os transistores estão sempre conduzindo corrente. Isso significa que, mesmo em amplificadores de alta potência, a dissipação de calor é relativamente baixa quando não há sinal de entrada ou quando o sinal é pequeno. Isso reduz a necessidade de sistemas de resfriamento complexos e caros.

\subsubsection{Conservação de Energia}

A eficiência energética da classe B a torna adequada para dispositivos portáteis alimentados por bateria, onde a vida útil da bateria é crítica. Amplificadores de áudio em smartphones, dispositivos de áudio Bluetooth e sistemas de comunicação sem fio frequentemente fazem uso da classe B para prolongar a vida útil da bateria, pois minimizam o consumo de energia quando não estão em uso.

\subsubsection{Aplicação em Conversores de Energia}

Além de amplificadores, a classe B é amplamente utilizada em conversores de energia, como inversores e fontes de alimentação chaveadas. Nessas aplicações, a eficiência energética é fundamental para a conversão eficaz de energia de uma forma para outra. A classe B é uma escolha lógica, pois permite a conversão de energia com perdas mínimas.

Em resumo, a classe B de operação em transistores bipolares de junção (TBJ) oferece eficiência energética máxima e é adequada para aplicações de alta potência, onde a conservação de energia é crítica. Suas vantagens tornam-na uma escolha valiosa em uma variedade de aplicações elétricas e eletrônicas, desde amplificadores de áudio de alta potência até sistemas de transmissão de RF de grande porte e dispositivos portáteis com alimentação por bateria.

\subsection{Desvantagens}

Apesar das suas vantagens em termos de eficiência energética e adequação para aplicações de alta potência, a classe B de operação em transistores bipolares de junção (TBJ) também apresenta algumas desvantagens importantes que limitam sua utilização em certos contextos. Vamos explorar mais detalhadamente as principais desvantagens dessa classe de operação:

\subsubsection{Distorção na Transição entre os Transistores}

A distorção na transição entre os transistores é uma das desvantagens mais significativas da classe B. Quando o sinal passa de um transistor para o outro durante a região de crossover, pode ocorrer distorção, resultando em harmônicos indesejados no sinal de saída. Isso é conhecido como "distorção de crossover". A distorção de crossover é mais evidente em amplificadores de áudio, onde a qualidade do som é crítica, e pode afetar negativamente a fidelidade do áudio amplificado. Para minimizar esse problema, é comum usar uma configuração de classe AB, que busca um equilíbrio entre eficiência e linearidade.

\subsubsection{Aplicações de Alta Fidelidade}

Devido à distorção de crossover, a classe B não é a escolha ideal para aplicações de alta fidelidade, onde a reprodução precisa e a fidelidade do sinal são essenciais. Amplificadores de alta fidelidade, sistemas de som de qualidade e estúdios de gravação procuram minimizar a distorção e, portanto, preferem configurações de classe A ou classe AB para obter um desempenho mais linear e uma reprodução precisa do áudio.

\subsubsection{Necessidade de Complementaridade}

Em um amplificador de classe B, são necessários dois transistores complementares para amplificar tanto a metade positiva quanto a metade negativa do ciclo do sinal de entrada. Isso significa que um transistor é do tipo NPN (transistor de junção positiva negativa) e o outro é do tipo PNP (transistor de junção negativa positiva). A necessidade de componentes complementares pode aumentar a complexidade do circuito e o custo de fabricação, especialmente em sistemas de alta potência.

\subsubsection{Sinais de Baixa Amplitude}

Embora a classe B seja altamente eficiente quando amplifica sinais de amplitude considerável, ela pode ser ineficaz quando lida com sinais de baixa amplitude. Quando o sinal de entrada é pequeno, a distorção de crossover se torna mais pronunciada, pois os transistores podem ser ligados e desligados com mais frequência. Isso pode resultar em distorção significativa e prejudicar a qualidade do sinal amplificado.

\subsubsection{Necessidade de Ajustes e Realimentação (Feedback)}

Para minimizar a distorção de crossover e melhorar a linearidade, muitos amplificadores de classe B incorporam circuitos de ajuste e realimentação (feedback). Esses circuitos adicionais aumentam a complexidade do projeto e podem exigir ajustes periódicos para manter o desempenho adequado. Isso torna os amplificadores de classe B menos simples de projetar e manter em comparação com amplificadores de classe A, que são mais autoestabilizantes.

\section{Classe AB}

A classe AB é uma classe de operação intermediária que busca combinar os benefícios da classe A (linearidade) e da classe B (eficiência energética). Nessa configuração, um dos transistores conduz mais do que meio ciclo do sinal de entrada, mas menos do que o ciclo completo. Isso ajuda a melhorar a eficiência energética em comparação com a classe A, enquanto ainda mantém uma boa linearidade, minimizando a distorção.

Os amplificadores classe AB são frequentemente usados em aplicações de áudio de alta qualidade, onde a eficiência energética é importante, mas a distorção precisa ser mantida em níveis aceitáveis. Eles são adequados para uma ampla gama de aplicações, incluindo amplificadores de potência de média potência e sistemas de comunicação.

\subsection{Características Principais}

A classe AB é uma categoria de operação de transistores bipolares de junção (TBJ) que combina elementos das classes A e B, resultando em características distintivas que a tornam amplamente utilizada em uma variedade de aplicações. Vamos explorar em detalhes as principais características dessa classe de operação.

\subsubsection{Combinação de Elementos das Classes A e B}

A característica mais notável da classe AB é a combinação de elementos das classes A e B. Enquanto na classe A o transistor TBJ conduz durante todo o ciclo do sinal e na classe B conduz apenas metade do ciclo, na classe AB, o transistor conduz mais do que meio ciclo, mas menos do que o ciclo completo. Isso significa que, na classe AB, há uma sobreposição controlada na condução dos transistores, minimizando a distorção de crossover que é típica da classe B.

\subsubsection{Melhor Eficiência Energética do que a Classe A}

Uma das vantagens significativas da classe AB é sua eficiência energética aprimorada em comparação com a classe A pura. Enquanto a classe A conduz constantemente e desperdiça energia mesmo quando não há sinal de entrada, a classe AB reduz a dissipação de energia durante os períodos em que o sinal é pequeno ou zero. Isso resulta em um consumo de energia menor em comparação com a classe A, tornando-a mais eficiente em termos energéticos.

\subsubsection{Redução da Distorção de Crossover}

A sobreposição controlada na condução dos transistores na classe AB ajuda a reduzir significativamente a distorção de crossover em comparação com a classe B. Isso melhora a linearidade da amplificação, tornando-a adequada para aplicações onde a fidelidade do sinal é importante, como em amplificadores de áudio de alta qualidade e sistemas de som de alta fidelidade. Embora a distorção não seja tão baixa quanto na classe A pura, a classe AB oferece um compromisso eficaz entre eficiência energética e qualidade de som.

\subsubsection{Ampla Aplicabilidade}

A classe AB é amplamente utilizada em uma variedade de aplicações, devido à sua flexibilidade e equilíbrio entre eficiência e linearidade. Ela é comum em amplificadores de áudio de alta potência, transmissores de RF de média potência, sistemas de som de alta qualidade, amplificadores de guitarra, equipamentos de áudio profissional e uma variedade de outros dispositivos eletrônicos.

\subsubsection{Circuitos de Polarização e Realimentação}

Para garantir que a operação da classe AB seja estável e linear, é comum incorporar circuitos de polarização e realimentação (feedback) nos amplificadores. Esses circuitos adicionais aumentam a complexidade do projeto, mas são necessários para controlar a condução dos transistores e minimizar a distorção. A realimentação é particularmente importante para manter a linearidade e a estabilidade em amplificadores de classe AB.

Em resumo, a classe AB de operação em transistores bipolares de junção (TBJ) combina elementos das classes A e B, resultando em uma eficiência energética aprimorada em comparação com a classe A pura, enquanto mantém uma redução significativa na distorção de crossover em relação à classe B. Essa combinação de eficiência e linearidade a torna uma escolha versátil para uma ampla gama de aplicações eletrônicas, especialmente em amplificação de áudio e sistemas de transmissão de média potência.

\subsection{Uso Comum}

A classe AB de operação em transistores bipolares de junção (TBJ) é amplamente empregada em uma variedade de aplicações devido ao seu equilíbrio entre eficiência energética e qualidade de sinal. Vamos explorar em detalhes os usos mais comuns da classe AB.

\subsubsection{Amplificação de Áudio de Alta Qualidade}

A amplificação de áudio de alta qualidade é uma das aplicações mais proeminentes da classe AB. A classe AB é escolhida em sistemas de áudio que exigem uma reprodução precisa e de alta fidelidade do som. Isso inclui amplificadores estéreo para sistemas de som doméstico, amplificadores de palco para músicos e bandas, amplificadores de potência para sistemas de som ao vivo e amplificadores de áudio de alta fidelidade (Hi-Fi). A capacidade da classe AB de oferecer uma qualidade de som superior em comparação com a classe B a torna a escolha ideal para essas aplicações.

\subsubsection{Amplificadores de RF de Média Potência}

A classe AB também é comumente utilizada em amplificadores de radiofrequência (RF) de média potência. Esses amplificadores são encontrados em sistemas de comunicação de média potência, como estações de rádio FM, transmissores de televisão UHF/VHF e sistemas de rádio móvel. A classe AB é preferida porque oferece eficiência energética suficiente para operações contínuas em transmissões RF moderadas, ao mesmo tempo em que mantém a linearidade necessária para evitar distorções indesejadas nos sinais de RF.

\subsubsection{Amplificadores de Guitarra}

Os amplificadores de guitarra também fazem uso frequente da classe AB. Esses amplificadores devem ser capazes de reproduzir com precisão as nuances do som da guitarra, incluindo sua distorção controlada. A classe AB é adequada para amplificadores de guitarra porque combina eficiência energética com a capacidade de fornecer uma resposta linear ao sinal de entrada, ao contrário da classe B, que seria inadequada devido à distorção de crossover.

\subsubsection{Equipamentos de Áudio Profissional}

Na indústria de áudio profissional, onde a qualidade do som é crítica, a classe AB é usada em amplificadores de potência para sistemas de PA (Público Alvo), mixers de áudio, processadores de sinal e amplificadores de monitoramento de estúdio. Esses equipamentos são usados em ambientes de produção musical, gravação de estúdio, eventos ao vivo e teatros, onde a fidelidade do som é essencial.

\subsubsection{Sistemas de Som de Alta Fidelidade}

Em sistemas de som de alta fidelidade em casa, como receivers estéreo e amplificadores integrados, a classe AB é amplamente utilizada. Os entusiastas de áudio doméstico valorizam a capacidade desses amplificadores de fornecer um som de alta qualidade e potência suficiente para preencher salas maiores com som cristalino.

\subsubsection{Amplificadores de Potência}

Em sistemas de som ao vivo, como concertos e eventos, amplificadores de potência de classe AB são usados para alimentar sistemas de alto-falantes de palco de alta potência. Esses amplificadores fornecem a potência necessária para amplificar o som de forma eficiente, mantendo a qualidade do áudio durante apresentações ao vivo.

Em resumo, a classe AB é altamente versátil e é escolhida para aplicações que exigem um equilíbrio entre eficiência energética e qualidade de sinal, com ênfase na amplificação de áudio de alta qualidade, amplificadores de RF de média potência e equipamentos de áudio profissional, atendendo a uma ampla gama de necessidades em comunicação, entretenimento e áudio de alta fidelidade.

\subsection{Vantagens}

A classe AB de operação em transistores bipolares de junção (TBJ) oferece várias vantagens que a tornam uma escolha popular em uma variedade de aplicações. Vamos explorar em detalhes as principais vantagens dessa classe de operação.

\subsubsection{Boa Linearidade}

Uma das vantagens mais notáveis da classe AB é sua boa linearidade. A linearidade se refere à capacidade do amplificador de reproduzir com precisão o sinal de entrada amplificado, sem introduzir distorção significativa. A classe AB oferece uma melhor linearidade em comparação com a classe B, devido à sobreposição controlada na condução dos transistores. Essa linearidade aprimorada torna a classe AB adequada para aplicações onde a fidelidade do sinal é crítica, como em amplificadores de áudio de alta qualidade e sistemas de som de alta fidelidade.

\subsubsection{Eficiência Energética Aprimorada}

A classe AB oferece uma eficiência energética significativamente melhor em comparação com a classe A pura. Enquanto os amplificadores de classe A conduzem continuamente, independentemente do nível de sinal de entrada, os amplificadores de classe AB reduzem a dissipação de energia durante os períodos em que o sinal é pequeno ou zero. Isso resulta em menor consumo de energia e menos geração de calor quando não há demanda de amplificação. A melhoria na eficiência energética torna a classe AB adequada para dispositivos alimentados por bateria e ajuda a reduzir os custos operacionais em sistemas de alta potência.

\subsubsection{Menor Geração de Calor}

Devido à sua operação eficiente, os amplificadores de classe AB geram menos calor em comparação com amplificadores de classe A, onde os transistores estão sempre conduzindo corrente. Isso é particularmente valioso em amplificadores de alta potência, onde a dissipação de calor excessiva pode causar problemas de resfriamento e reduzir a vida útil dos componentes eletrônicos. A menor geração de calor contribui para a confiabilidade e a longevidade dos dispositivos de classe AB.

\subsubsection{Aplicações em Amplificadores de Áudio}

A qualidade do som é crucial em amplificadores de áudio de alta fidelidade (Hi-Fi). A classe AB é frequentemente escolhida para esses amplificadores devido à sua combinação de boa linearidade e eficiência energética. Ela permite que os amplificadores de áudio de alta qualidade reproduzam com precisão os detalhes musicais, mantendo a distorção em níveis aceitáveis. Isso faz da classe AB uma escolha popular em sistemas de áudio de alta fidelidade em casa e em estúdios de gravação profissional.

\subsubsection{Amplificadores de Sinais de Média Potência}

A classe AB é adequada para amplificadores de sinais de média potência, onde a eficiência energética e a linearidade são importantes. Isso inclui amplificadores de RF de média potência usados em comunicações, transmissão de televisão e rádio FM, bem como amplificadores de potência em sistemas de som de palco e amplificadores de instrumentos musicais.

\subsection{Desvantagens}

Embora a classe AB de operação em transistores bipolares de junção (TBJ) ofereça muitas vantagens, também apresenta algumas desvantagens importantes que precisam ser consideradas em projetos e aplicações. Vamos explorar em detalhes as principais desvantagens dessa classe de operação.

\subsubsection{Distorção em Altas Potências}

Embora a classe AB tenha uma melhor linearidade em comparação com a classe B, a distorção ainda pode ser um problema, especialmente em altas potências. Durante a transição da condução de um transistor para o outro, a sobreposição controlada pode não ser suficiente para evitar completamente a distorção de crossover. Isso pode resultar em distorção harmônica indesejada no sinal de saída, afetando a qualidade do som ou do sinal amplificado. Para minimizar esse problema, é comum incorporar realimentação (feedback) e circuitos de correção na concepção de amplificadores de classe AB.

\subsubsection{Polarização, Corte e Saturação}

A operação adequada da classe AB requer uma polarização cuidadosa dos transistores para evitar problemas de corte e saturação. Se os transistores não estiverem polarizados corretamente, podem ocorrer situações em que um transistor está cortado (não conduzindo) enquanto o outro está saturado (conduzindo continuamente). Isso resulta em distorção grave e pode danificar os transistores. Portanto, a polarização precisa ser ajustada e controlada com precisão, o que pode aumentar a complexidade do projeto e exigir ajustes periódicos para garantir o desempenho adequado.

\subsubsection{Complexidade}

Para mitigar os problemas de polarização e distorção, os amplificadores de classe AB frequentemente incorporam circuitos de controle e correção, como diodos de polarização, circuitos de realimentação, e dispositivos de proteção. Esses circuitos adicionais aumentam a complexidade do projeto, bem como o custo de fabricação e manutenção. A complexidade também pode tornar os amplificadores de classe AB menos robustos em comparação com amplificadores de classe A, que são mais autoestabilizantes.

\subsubsection{Ineficiência em Sinais de Baixa Amplitude}

Assim como na classe B, os amplificadores de classe AB podem ser ineficazes quando amplificam sinais de baixa amplitude. Quando o sinal de entrada é pequeno, a distorção de crossover se torna mais pronunciada, o que pode resultar em distorção significativa. Para compensar isso, alguns amplificadores de classe AB incorporam uma configuração de "classe A até um certo ponto", onde operam em classe A para sinais de baixa amplitude e depois mudam para classe AB quando o sinal atinge um determinado nível.

\section{Conclusões}

Chegamos à conclusão de que o experimento foi conduzido com êxito, apresentando resultados que se aproximaram das expectativas inicialmente estabelecidas. A análise do circuito Darlington proporcionou uma compreensão mais aprofundada do comportamento dos transistores, bem como das análises necessárias para sua polarização e controle.

Foi possível realizar a montagem do projeto e empreender uma análise específica do componente preponderante no circuito, o capacitor $C_2$. Essa análise revelou a influência significativa de $C_2$ na frequência de corte do circuito, proporcionando uma visão mais apurada de sua contribuição para o desempenho global do sistema.

% \section{Apêndice}

Abaixo se encontra o código utilizado para a análise simbólica e numérica do circuito.

\begin{python}
    #from google.colab import drive
    #drive.mount('/content/drive')

    # Updating sympy to version 1.12 for faster inverse laplace transform
    # You need to restart the environment for the changes to take action
    # You need to run it before every session, or else the version will
    # be 1.11.1

    from google.colab import files

    %pip install -q --upgrade sympy

    import sympy
    sympy.__version__

    import matplotlib.pyplot as plt
    from sympy import *
    from IPython.core.interactiveshell import InteractiveShell

    # Allows multiple latex formatted lines
    InteractiveShell.ast_node_interactivity = 'all'

    # init_session prints the result in latex format, and has some useful presets,
    # more information at: https://docs.sympy.org/latest/modules/interactive.html
    init_session(quiet=True)
    # Allows the use of unicode characters
    init_printing(use_unicode=True)

\end{python}

\begin{python}
    import matplotlib.pyplot as plt
    from sympy import *

    # init_session prints the result in latex format, and has some useful presets,
    # more information at: https://docs.sympy.org/latest/modules/interactive.html
    init_session(quiet=True)
    # Allows the use of unicode characters
    init_printing(use_unicode=True)

    R0, R1, R2, R3, R4, R5, R6, R7, R8, R9, R10, R11, R12, R13 = \
    symbols('R0 R1 R2 R3 R4 R5 R6 R7 R8 R9 R10 R11 R12 R13')

    Va0, Va1, Va2, Va3, Va4, Va5, Va6, Va7, Va8, Va9, Va10 = \
    symbols('Va0 Va1 Va2 Va3 Va4 Va5 Va6 Va7 Va8 Va9 Va10')

    V0, V1, V2, V3, V4, V5, V6, V7, V8, V9, V10 = \
    symbols('V0 V1 V2 V3 V4 V5 V6 V7 V8 V9 V10')

    Vo0, Vo1, Vo2, Vo3, Vo4, Vo5, Vo6, Vo7, Vo8, Vo9, Vo10 = \
    symbols('Vo0 Vo1 Vo2 Vo3 Vo4 Vo5 Vo6 Vo7 Vo8 Vo9 Vo10')

    Vl0, Vl1, Vl2, Vl3, Vl4, Vl5, Vl6, Vl7, Vl8, Vl9, Vl10 = \
    symbols('Vl0 Vl1 Vl2 Vl3 Vl4 Vl5 Vl6 Vl7 Vl8 Vl9 Vl10')

    array_resistores = [R0, R1, R2, R3, R4, R5, R6, R7, R8, R9, R10, R11, R12, R13]

    Va = [Va0, Va1, Va2, Va3, Va4, Va5, Va6, Va7, Va8, Va9, Va10]

    V = [V0, V1, V2, V3, V4, V5, V6, V7, V8, V9, V10]

    Vo = [Vo0, Vo1, Vo2, Vo3, Vo4, Vo5, Vo6, Vo7, Vo8, Vo9, Vo10]

    Vl = [Vl0, Vl1, Vl2, Vl3, Vl4, Vl5, Vl6, Vl7, Vl8, Vl9, Vl10]

\end{python}

\begin{python}
    # Recebe um array de resistores e calcula a resistencia equivalente
    def paralelo(array_resistor, jump):
    sum = 0
    for i in range(len(array_resistor)):
    if i != jump:
    sum += 1/array_resistor[i]

    return 1/sum

    # Recebe um vetor de tensoes de entrada e um vetor de resistores ambos de mesmo tamanho e retorna o vetor com as tensoes de saida
    def divisorTensao(Vin, Vout ,resistores):
    for i in range(len(Vout)):
    Req = paralelo(resistores, i)
    Vout[i] = Vin[i]*Req/(resistores[i] + Req)
    return Vout
\end{python}

\begin{python}

    Va[0:4]=[0]*4
    Va[4:10] = divisorTensao(V[4:10], Va[4:10], array_resistores[4:10])
    Va[10] = 0
    Va

    for i in range(len(Vo)):
    if i<4:
    Vo[i]=0
    elif i<10:
    Vo[i] = Va[i]*(1 + array_resistores[11]/array_resistores[10])
    else:
    Vo[i] = -2.5*V[i]*array_resistores[11]/array_resistores[10]
    Vo

    for i in range(len(Vl)):
    if i<4:
    Vl[i] = -V[i]*array_resistores[13]/array_resistores[i]
    else:
    Vl[i] = -Vo[i]*(array_resistores[13]/array_resistores[12])
    Vl

    #K1 = R13
    #K2 = (R13/R12)(1+(R11/R10))
    #S10 = (2.5*R11*R13)/(R10*R12)

\end{python}

% \section{Anexos}

Código utilizado para geração de gráficos de bode, e análise utilizando as frequências de cortes obtidas experimentalmente.

\begin{python}
    import numpy as np
    import matplotlib.pyplot as plt
    from scipy.optimize import curve_fit



    fc1 = 210000
    fc2 = 11800


    freqs1 = np.array([0.002*fc1, 0.01*fc1, 0.05*fc1, 0.2*fc1, 0.5*fc1,
            0.8*fc1, fc1, 2*fc1, 4*fc1, 10*fc1, 20*fc1, 40*fc1])
    vin1 = np.array([0.2975, 0.297, 0.298, 0.297, 0.298,
            0.301, 0.297, 0.297, 0.3, 0.302, 0.301, 0.315])
    vout1 = np.array([1.404, 1.405, 1.403, 1.393, 1.306, 1.124,
            0.975, 0.54, 0.276, 0.114, 0.057, 0.031])


    freqs2 = np.array([0.05*fc2, 0.1*fc2, 0.2*fc2, 0.5*fc2, 0.8*fc2, fc2,
            2*fc2, 5*fc2, 20*fc2, 50*fc2, 200*fc2, 500*fc2, 1000*fc2])
    vin2 = np.array([0.1191, 0.1199, 0.12, 0.121, 0.122, 0.122,
            0.1218, 0.123, 0.123, 0.121, 0.121, 0.125, 0.151])
    vout2 = np.array([14.01, 13.95, 13.77, 12.61, 10.85, 9.8, 5.92,
            2.548, 0.645, 0.258, 0.0652, 0.0205, 0.0122])




    def mag_sqr_fun(f, K, fc):
    return (K*fc)**2/(f**2 + fc**2)


    def dB(m):
    return 20*np.log10(m)




    (K1, fc1), _ = curve_fit(lambda f, K, fc: dB(mag_sqr_fun(f, K, fc)),
    freqs1, 2*dB(vout1/vin1))


    print(f"""O ganho K eh {round(K1, 1)}
    A frequencia de corte eh {round(fc1, 1)} Hz""")

    f1 = np.logspace(np.log10(freqs1[0]) - 1, np.log10(freqs1[-1]) + 0.3)
    mag1 = dB(mag_sqr_fun(f1, K1, fc1))/2

    plt.semilogx(f1, mag1)
    plt.semilogx(freqs1, dB(vout1/vin1), "*")
    plt.xlabel("Freq (Hz)")
    plt.ylabel("Mag(H) (dB)")
    plt.title("Grafico de Bode de magnitude para o 1 circuito")
    plt.grid()
    plt.savefig("figura1.png")
    plt.show()



    (K2, fc2), _ = curve_fit(lambda f, K, fc: dB(mag_sqr_fun(f, K, fc)),
    freqs2, 2*dB(vout2/vin2))

    print("\n\nResultados para o 2 circuito:\n")
    print(f"""O ganho K eh {round(K2, 1)}
    A frequencia de corte eh {round(fc2, 1)} Hz""")

    f2 = np.logspace(np.log10(freqs2[0]) - 1, np.log10(freqs2[-1]) + 0.3)
    mag2 = dB(mag_sqr_fun(f2, K2, fc2))/2

    plt.semilogx(f2, mag2)
    plt.semilogx(freqs2, dB(vout2/vin2), "*")
    plt.xlabel("Freq (Hz)")
    plt.ylabel("Mag(H) (dB)")
    plt.title("Grafico de Bode de magnitude para o 2 circuito")
    plt.grid()
    plt.savefig("figura2.png")
    plt.show()




    plt.semilogx(f1, mag1)
    plt.semilogx(f2, mag2)
    plt.semilogx(freqs1, dB(vout1/vin1), "*")
    plt.semilogx(freqs2, dB(vout2/vin2), "*")
    plt.xlabel("Freq (Hz)")
    plt.ylabel("Mag(H) (dB)")
    plt.title("Grafico de Bode de magnitude para ambos os circuitos")
    plt.grid()
    plt.savefig("figura3.png")
    plt.show()

\end{python}


\end{document}