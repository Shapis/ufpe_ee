\section{Análise dos resultados}

Os principais resultados obtidos a partir das medições e simulações realizadas no experimento foram analisados, incluindo comparações com os resultados numéricos e as conclusões relevantes.

\subsection{Exemplo 1}

O exemplo 1 foi analisado sem a presença do resistor $R$ com o objetivo de simular a situação de uma resistência infinita.

\subsubsection{Gráficos}

Ao comparar os gráficos do exemplo 1 experimental (Figura \ref{fig:ex1}) com o exemplo 1 numérico (Figura \ref{fig:ex1_numerico}), observa-se notáveis semelhanças. Em ambos os casos, a tensão mínima é grampeada aproximadamente na tensão de polarização do diodo, enquanto a tensão máxima permanece fixa em cerca de $9.5V$, resultando em uma forma de onda quadrada. Essa concordância entre os resultados experimental e numérico indica que o circuito se comporta conforme o previsto, fortalecendo a validade das análises teóricas.

\subsubsection{Mínimos e máximos}

Ao comparar os valores de tensão mínima e máxima obtidos experimentalmente (Figura \ref{fig:minmax_ex1}) com os valores obtidos numericamente (\ref{eq:ex1_minmax}), é possível observar que os valores são bastante semelhantes, porém não exatamente iguais. Essa diferença pode ser atribuída a fatores como variações nos valores dos componentes utilizados no experimento em relação aos valores simulados, presença de ruídos no sinal de saída provenientes de interferências eletromagnéticas ou erros de medição. Além disso, é importante considerar que a tensão de polarização do diodo pode ter apresentado ligeiras variações ao longo do experimento, o que também contribui para a discrepância entre os valores de tensão máxima observados.

\subsection{Exemplo 2}

O exemplo 2 foi analisado com a presença do resistor $R = 4.7K \Omega$.

\subsubsection{Gráficos}

Ao comparar os gráficos do exemplo 2 experimental (Figura \ref{fig:ex2}) com o exemplo 1 numérico (Figura \ref{fig:ex2_numerico}), observa-se notáveis semelhanças. Em ambos os casos, a tensão mínima é grampeada aproximadamente na tensão de polarização do diodo, enquanto a tensão máxima permanece fixa em cerca de $9.5V$, resultando em uma forma de onda quadrada. Essa concordância entre os resultados experimental e numérico indica que o circuito se comporta conforme o previsto, fortalecendo a validade das análises teóricas.

\subsubsection{Mínimos e máximos}

Ao comparar os valores de tensão mínima e máxima obtidos experimentalmente (Figura \ref{fig:minmax_ex2}) com os valores obtidos numericamente (\ref{eq:ex2_minmax1} e \ref{eq:ex2_minmax2}), é possível observar que os valores são bastante semelhantes, porém não exatamente iguais. Essa diferença pode ser atribuída a fatores como variações nos valores dos componentes utilizados no experimento em relação aos valores simulados, presença de ruídos no sinal de saída provenientes de interferências eletromagnéticas ou erros de medição. Além disso, é importante considerar que a tensão de polarização do diodo pode ter apresentado ligeiras variações ao longo do experimento, o que também contribui para a discrepância entre os valores de tensão máxima observados.