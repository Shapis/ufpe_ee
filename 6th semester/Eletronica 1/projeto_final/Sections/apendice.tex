\section{Apêndice}

Abaixo se encontra o código utilizado para a análise simbólica e numérica do circuito.

\begin{python}
    #from google.colab import drive
    #drive.mount('/content/drive')

    # Updating sympy to version 1.12 for faster inverse laplace transform
    # You need to restart the environment for the changes to take action
    # You need to run it before every session, or else the version will
    # be 1.11.1

    from google.colab import files

    %pip install -q --upgrade sympy

    import sympy
    sympy.__version__

    import matplotlib.pyplot as plt
    from sympy import *
    from IPython.core.interactiveshell import InteractiveShell

    # Allows multiple latex formatted lines
    InteractiveShell.ast_node_interactivity = 'all'

    # init_session prints the result in latex format, and has some useful presets,
    # more information at: https://docs.sympy.org/latest/modules/interactive.html
    init_session(quiet=True)
    # Allows the use of unicode characters
    init_printing(use_unicode=True)

\end{python}

\begin{python}
    import matplotlib.pyplot as plt
    from sympy import *

    # init_session prints the result in latex format, and has some useful presets,
    # more information at: https://docs.sympy.org/latest/modules/interactive.html
    init_session(quiet=True)
    # Allows the use of unicode characters
    init_printing(use_unicode=True)

    R0, R1, R2, R3, R4, R5, R6, R7, R8, R9, R10, R11, R12, R13 = \
    symbols('R0 R1 R2 R3 R4 R5 R6 R7 R8 R9 R10 R11 R12 R13')

    Va0, Va1, Va2, Va3, Va4, Va5, Va6, Va7, Va8, Va9, Va10 = \
    symbols('Va0 Va1 Va2 Va3 Va4 Va5 Va6 Va7 Va8 Va9 Va10')

    V0, V1, V2, V3, V4, V5, V6, V7, V8, V9, V10 = \
    symbols('V0 V1 V2 V3 V4 V5 V6 V7 V8 V9 V10')

    Vo0, Vo1, Vo2, Vo3, Vo4, Vo5, Vo6, Vo7, Vo8, Vo9, Vo10 = \
    symbols('Vo0 Vo1 Vo2 Vo3 Vo4 Vo5 Vo6 Vo7 Vo8 Vo9 Vo10')

    Vl0, Vl1, Vl2, Vl3, Vl4, Vl5, Vl6, Vl7, Vl8, Vl9, Vl10 = \
    symbols('Vl0 Vl1 Vl2 Vl3 Vl4 Vl5 Vl6 Vl7 Vl8 Vl9 Vl10')

    array_resistores = [R0, R1, R2, R3, R4, R5, R6, R7, R8, R9, R10, R11, R12, R13]

    Va = [Va0, Va1, Va2, Va3, Va4, Va5, Va6, Va7, Va8, Va9, Va10]

    V = [V0, V1, V2, V3, V4, V5, V6, V7, V8, V9, V10]

    Vo = [Vo0, Vo1, Vo2, Vo3, Vo4, Vo5, Vo6, Vo7, Vo8, Vo9, Vo10]

    Vl = [Vl0, Vl1, Vl2, Vl3, Vl4, Vl5, Vl6, Vl7, Vl8, Vl9, Vl10]

\end{python}

\begin{python}
    # Recebe um array de resistores e calcula a resistencia equivalente
    def paralelo(array_resistor, jump):
    sum = 0
    for i in range(len(array_resistor)):
    if i != jump:
    sum += 1/array_resistor[i]

    return 1/sum

    # Recebe um vetor de tensoes de entrada e um vetor de resistores ambos de mesmo tamanho e retorna o vetor com as tensoes de saida
    def divisorTensao(Vin, Vout ,resistores):
    for i in range(len(Vout)):
    Req = paralelo(resistores, i)
    Vout[i] = Vin[i]*Req/(resistores[i] + Req)
    return Vout
\end{python}

\begin{python}

    Va[0:4]=[0]*4
    Va[4:10] = divisorTensao(V[4:10], Va[4:10], array_resistores[4:10])
    Va[10] = 0
    Va

    for i in range(len(Vo)):
    if i<4:
    Vo[i]=0
    elif i<10:
    Vo[i] = Va[i]*(1 + array_resistores[11]/array_resistores[10])
    else:
    Vo[i] = -2.5*V[i]*array_resistores[11]/array_resistores[10]
    Vo

    for i in range(len(Vl)):
    if i<4:
    Vl[i] = -V[i]*array_resistores[13]/array_resistores[i]
    else:
    Vl[i] = -Vo[i]*(array_resistores[13]/array_resistores[12])
    Vl

    #K1 = R13
    #K2 = (R13/R12)(1+(R11/R10))
    #S10 = (2.5*R11*R13)/(R10*R12)

\end{python}