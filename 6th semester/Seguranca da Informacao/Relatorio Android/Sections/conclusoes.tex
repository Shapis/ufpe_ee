\section{Conclusões}

Percebe-se que, embora o sistema de particionamento não seja uniforme em todos os dispositivos, existem partições recorrentes. Constatamos que o processo de destruição de dados pode ser realizado de maneira relativamente simples. No entanto, a recuperação de dados pode se revelar uma tarefa mais complexa, uma vez que sua eficácia depende das especificidades do dispositivo e da versão do sistema Android.

Essa diversidade no sistema de particionamento, aliada à crescente complexidade das versões do Android, demonstra a necessidade de abordagens flexíveis e adaptáveis tanto na destruição quanto na recuperação de dados. Enquanto a destruição pode ser conduzida com relativa simplicidade, a recuperação requer um conhecimento aprofundado das particularidades do dispositivo em questão, assim como da versão do Android que o equipa.

Dessa forma, torna-se fundamental considerar esses fatores ao planejar procedimentos de segurança de dados e ao lidar com situações que envolvem a eliminação ou a restauração de informações em dispositivos Android. Cada caso exigirá uma estratégia sob medida para garantir a proteção ou a recuperação eficiente dos dados, levando em consideração a complexidade do cenário em questão.