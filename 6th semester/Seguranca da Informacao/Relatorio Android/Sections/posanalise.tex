\newpage

\section{Análise dos resultados}

Os principais resultados obtidos a partir das medições e simulações realizadas no experimento foram analisados, incluindo comparações com os resultados numéricos e as conclusões relevantes.

\subsection{Exemplo 1}

No Exemplo 1, obtemos o $V_{D0}$ medindo as tensões sobre $R_5$ e $V_o$ no caso em que o LED está ligado, ou seja:

\begin{equation}
    V_{D0} = V_{om2} - V_{R5m2} = 9.22V - 7.14V = 2.08V
\end{equation}

Utilizando os valores de $V_{m1}$, $V_{m2}$ e o novo $V_{D0}$, procedemos ao recálculo dos seguintes parâmetros:

\begin{itemize}
    \item $T = 515.8 ms$
    \item $k = 0.501$
    \item $V_1 = 4.313V$
    \item $V_2 = 5.39V$
    \item $I_L = 12.95mA$
\end{itemize}

Com esses dados em mãos, é possível realizar uma comparação com os valores obtidos experimentalmente:

\begin{center}
    \begin{tabular}{ |c|c|c| }
        \hline
        Medidas & Experimental & Numerico   \\
        $T$     & $510.62 ms$  & $515.8 ms$ \\
        $k$     & $0.5$        & $0.501$    \\
        $V_1$   & $4.31V$      & $4.313V$   \\
        $V_2$   & $5.4V$       & $5.39V$    \\
        $I_L$   & $12.95mA$    & $12.95mA$  \\
        \hline
    \end{tabular}
\end{center}

Ao analisar os resultados, é notável que os valores obtidos após o recálculo se aproximam consideravelmente dos valores obtidos por meio de experimentação prática. Essa proximidade indica que a montagem do circuito foi executada de forma consistente e precisa, demonstrando a integridade das medições realizadas.

\subsection{Exemplo 2}

No Exemplo 1, obtemos o $V_{D0}$ medindo as tensões sobre $R_5$ e $V_o$ no caso em que o LED está ligado, ou seja:

\begin{equation}
    V_{D0} = V_{om2} - V_{R5m2} = 9.23V - 7.22V = 2.01V
\end{equation}

Utilizando os valores de $V_{m1}$, $V_{m2}$ e o novo $V_{D0}$, procedemos ao recálculo dos seguintes parâmetros:

\begin{itemize}
    \item $T = 6.6 ms$
    \item $k = 0.398$
    \item $V_1 = 1.38 V$
    \item $V_2 = 7.12 V$
    \item $I_L = 13.1mA$
\end{itemize}

Com esses dados em mãos, é possível realizar uma comparação com os valores obtidos experimentalmente:

\begin{center}
    \begin{tabular}{ |c|c|c| }
        \hline
        Medidas & Experimental & Numerico \\
        $T$     & $7.4422 ms$  & $6.6 ms$ \\
        $k$     & $0.39$       & $0.398$  \\
        $V_1$   & $1.35V$      & $1.38$   \\
        $V_2$   & $7.14V$      & $7.12$   \\
        $I_L$   & $13.1mA$     & $13.1mA$ \\
        \hline
    \end{tabular}
\end{center}

Ao analisar os resultados, é notável que os valores obtidos após o recálculo se aproximam consideravelmente dos valores obtidos por meio de experimentação prática. Essa proximidade indica que a montagem do circuito foi executada de forma consistente e precisa, demonstrando a integridade das medições realizadas.